\subsubsection*{Zadanie~8.24.}
Zdarzenia:
\begin{description}
    \item[\(Z_1\)] --- piernik wyprodukowany na pierwszej zmianie
    \item[\(Z_2\)] --- piernik wyprodukowany na drugiej zmianie
    \item[\(N\)] --- piernik uszkodzony
\end{description}
\begin{gather*}
    P\pars{Z_1} = \frac{2}{3}\\
    P\pars{Z_2} = \frac{1}{3}\\
    P\pars{N / Z_1} = \frac{95}{100} = \frac{19}{20}\\
    P\pars{N / Z_2} = \frac{97}{100}\\
    P\pars{N}
        = P\pars{N / Z_1} \cdot P\pars{Z_1} + P\pars{N / Z_2} \cdot P\pars{Z_2}
        = \frac{19}{20} \cdot \frac{2}{3} + \frac{97}{100} \cdot \frac{1}{3}
        = \frac{190}{300} + \frac{97}{300}
        = \frac{287}{300}
\end{gather*}
\subsubsection*{Zadanie~8.27.}
Przez \(D\) będziemy oznaczać zdarzenie polegające na wylosowaniu dobrego cukierka.
\begin{gather*}
    P\pars{F_1} = \frac{25}{100} = \frac{1}{4}\\
    P\pars{F_2} = \frac{35}{100} = \frac{7}{20}\\
    P\pars{F_3} = \frac{40}{100} = \frac{2}{5}\\
    P\pars{D / F_1} = \frac{98}{100} = \frac{49}{50}\\
    P\pars{D / F_2} = \frac{96}{100} = \frac{24}{25}\\
    P\pars{D / F_3} = \frac{95}{100} = \frac{19}{20}\\
    \begin{split}
    P\pars{F_3 / D}
        &= \frac{P\pars{D / F_3} \cdot P\pars{F_3}}{P\pars{D / F_1} \cdot P\pars{F_1} + P\pars{D / F_2} \cdot P\pars{F_2} + P\pars{D / F_3} \cdot P\pars{F_3}}
        = \frac{\frac{19}{20} \cdot \frac{2}{5}}{\frac{49}{50} \cdot \frac{1}{4} + \frac{24}{25} \cdot \frac{7}{20} + \frac{19}{20} \cdot \frac{2}{5}}\\
        &= \frac{\frac{19}{50}}{\frac{49}{200} + \frac{168}{500} + \frac{19}{50}}
        = \frac{\frac{380}{1000}}{\frac{245}{1000} + \frac{336}{1000} + \frac{380}{1000}}
        = \frac{380}{961}
    \end{split}
\end{gather*}
\subsubsection*{Zadanie~8.28.}
Zdarzenia:
\begin{description}
    \item[\(H\)] --- morela pochodząca z~Hiszpanii
    \item[\(W\)] --- morela pochodząca z~Włoch
    \item[\(N\)] --- morela niedojrzała
\end{description}
\begin{gather*}
    P\pars{H} = \frac{45}{100} = \frac{9}{20}\\
    P\pars{W} = \frac{55}{100} = \frac{11}{20}\\
    P\pars{N / H} = \frac{8}{1000} = \frac{1}{125}\\
    P\pars{N / W} = \frac{12}{1000} = \frac{3}{250}\\
    P\pars{N}
        = P\pars{N / H} \cdot P\pars{H} + P\pars{N / W} \cdot P\pars{W}
        = \frac{1}{125} \cdot \frac{9}{20} + \frac{3}{250} \cdot \frac{11}{20}
        = \frac{18}{5000} + \frac{33}{5000}
        = \frac{51}{5000}
\end{gather*}
\subsubsection*{Zadanie~8.31.}
Zdarzenia:
\begin{description}
    \item[\(K\)] --- kobieta
    \item[\(M\)] --- mężczyzna
    \item[\(C\)] --- osoba nierozróżniająca kolorów
\end{description}
\begin{gather*}
    P\pars{K} = P\pars{M} = \frac{1}{2}\\
    P\pars{C / K} = \frac{2}{1000} = \frac{1}{500}\\
    P\pars{C / M} = \frac{5}{100} = \frac{1}{20}\\
    P\pars{C}
        = P\pars{C / K} \cdot P\pars{K} + P\pars{C / M} \cdot P\pars{M}
        = \frac{1}{2}\pars{\frac{1}{500} + \frac{1}{20}}
        = \frac{1}{1000} + \frac{1}{40}
        = \frac{1}{1000} + \frac{25}{1000}
        = \frac{26}{1000}
\end{gather*}