\subsection*{Zestaw~XVII --- Planimetria (zadania otwarte)}
\subsubsection*{Zadanie~2.}
\begin{mathfigure*}
    \coordinate (A) at (-2, 0);
    \coordinate (B) at (2, 0);
    \coordinate (C) at (0, 5);
    \coordinate (X) at (-0.5, 0);
    \coordinate (H) at (0, 0);
    \draw[dashed] (C) -- node[right]{\(h\)} (H);
    \draw (A) node[below left]{\(A\)}
        -- node[below]{\(m\)} (B) node[below right]{\(B\)}
        -- node[above right]{\(n\)} (C) node[above]{\(C\)}
        -- node[above left]{\(n\)} cycle;
    \path (A)
        -- node[above]{\(x\)} (X)
        -- node[above]{\(m - x\)} (B);
    \draw (C) -- (X);
    \fillpoint*{X}[\(X\)][below];
\end{mathfigure*}
\begin{gather*}
    \frac{d\pars{X, AB} \cdot n}{2} = \area{XBC} = \frac{h\pars{m - x}}{2}\\
    \frac{d\pars{X, AC} \cdot n}{2} = \area{XAC} = \frac{hx}{2}
\end{gather*}
Po pomnożeniu obustronnie przez \(2\) i~dodaniu stronami tych równań otrzymujemy
\begin{gather*}
    n\pars{d\pars{X, AB} + d\pars{X, AC}} = h\pars{m - x} + hx\\
    n\pars{d\pars{X, AB} + d\pars{X, AC}} = hm - hx + hx\\
    d\pars{X, AB} + d\pars{X, AC} = \frac{hm}{n}
\end{gather*}
Widzimy, że wartość tego wyrażenia nie zależy od \(x\).
\qed
\subsubsection*{Zadanie~5.}
\begin{mathfigure*}
    \coordinate (A) at (0, 0);
    \coordinate (B) at (5, 0);
    \coordinate (C) at (0, 3);
    \coordinate (L) at (2.5, 1.5);
    \coordinate (K) at (2.5, 0);
    \coordinate (M) at (0, 1.5);
    \draw (A) node[below left]{\(A\)}
        -- (B) node[below right]{\(B\)}
        -- (C) node[above]{\(C\)}
        -- cycle;
    \draw (K) -- (L) -- (M);
    \fillpoint*{M}[\(M\)][left];
    \fillpoint*{K}[\(K\)][below];
    \fillpoint*{L}[\(L\)][above right];
\end{mathfigure*}
\subsubsection*{Zadanie~6.}
\begin{mathfigure*}
    \coordinate (A) at (-1, 0);
    \coordinate (B) at (0, -1);
    \coordinate (S) at (1, 1);
    \coordinate (T) at (-1, -1);
    \coordinate (N) at (3, 2);
    \coordinate (M) at (-1, -2);
    \drawrightangle[angle radius=0.3cm, ForestGreen]{N--B--A};
    \drawrightangle[angle radius=0.3cm, ForestGreen]{A--B--M};
    \draw[dashed, ForestGreen] (A) -- (B) -- (M);
    \draw[dashed, ForestGreen] (A) -- (B) -- (N);
    \draw (S) circle[radius=\fpeval{sqrt(5)}];
    \draw (T) circle[radius=1];
    \draw (A) -- (M);
    \draw (A) -- (N);
    \fillpoint{S};
    \fillpoint{T};
    \fillpoint*{A}[\(A\)][above left];
    \fillpoint*{B}[\(B\)][below right];
    \fillpoint*{N}[\(N\)][above right];
    \fillpoint*{M}[\(M\)][below left];
\end{mathfigure*}
Rozważmy kąty \(\angle{MBA}\) i~\(\angle{NBA}\). Są one proste, ponieważ są wpisane w~odpowiednie okręgi i~oparte na ich średnicach. Zatem
\begin{equation*}
    \mangle{MBN}
        = \mangle{MBA} + \mangle{NBA}
        = 90\degree + 90\degree
        = 180\degree
\end{equation*}
Kąt \(\angle{MBN}\) jest półpełny, więc punkty \(M\), \(B\), \(N\) są współliniowe.
\qed
\subsubsection*{Zadanie 7.}
\begin{mathfigure*}
    \coordinate (A) at (0, 0);
    \coordinate (B) at (3, 0);
    \coordinate (C) at (2, 3);
    \coordinate (D) at (-1, 2);
    \draw (A) node[below left]{\(A\)}
        -- (B) node[below right]{\(B\)}
        -- (C) node[above right]{\(C\)}
        -- (D) node[above left]{\(D\)}
        -- cycle;
    \drawangle*[ForestGreen, angle radius=0.5cm]{C--B--D}[\tiny\(58\degree\)];
    \draw (B) -- (D);
    \drawangle*[angle radius=0.5cm]{D--B--A}[\tiny\(44\degree\)];
    \drawangle*[angle radius=0.5cm, Orange]{A--D--C}[\tiny\(78\degree\)];
    \drawangle[angle radius=0.5cm, ForestGreen]{C--A--D};
    \draw (C) -- (A);
\end{mathfigure*}
Zauważmy, że
\begin{equation*}
    \mangle{ABD} + \mangle{CBD} + \mangle{ADC}
        = 44\degree + 58\degree + 78\degree
        = 180\degree
\end{equation*}
Skoro suma przeciwległych kątów wynosi \(180\degree\), to czworokąt \(ABCD\) jest wpisany w~okrąg. W~tym okręgu kąt \(\angle{CBD}\) jest oparty na tym samym łuku \(\arc{CD}\) co kąt \(\angle{CAD}\). Mają zatem takie same miary:
\begin{equation*}
    \mangle{CAD} = \mangle{CBD} = 54\degree
\end{equation*}
\subsubsection*{Zadanie~8.}
\begin{mathfigure*}
    \def\rt{\fpeval{sqrt(3)}}
    \coordinate (start) at (-6, 0);
    \coordinate (end) at (4, 0);
    \coordinate (S) at (0, 3);
    \coordinate (T) at (-2*\rt, 1);
    \coordinate (X) at (0, 0);
    \coordinate (Y) at (-2*\rt, 0);
    \coordinate (P) at (0, 1);
    \drawrightangle[angle radius=0.25cm]{X--Y--T};
    \drawrightangle[angle radius=0.25cm]{Y--T--P};
    \drawrightangle[angle radius=0.25cm]{S--P--T};
    \drawrightangle[angle radius=0.25cm]{P--X--Y};
    \drawangle[angle radius=0.4cm, Orange]{Y--T--S};
    \drawangle[angle radius=0.4cm, ForestGreen]{T--S--P};
    \draw (start) -- (end);
    \draw (S) circle[radius=3];
    \draw (T) circle[radius=1];
    \draw (X) -- (S) -- (T) -- (Y);
    \draw[dashed] (T) -- (P);
    \path (P) -- node[right]{\(2\)} (S);
    \path (P) -- node[right]{\(1\)} (X);
    \path (Y) -- node[left]{\(1\)} (T);
    \fillpoint*{S}[\(S\)][above];
    \fillpoint*{T}[\(T\)][above];
    \fillpoint*{X}[\(X\)][below];
    \fillpoint*{Y}[\(Y\)][below];
    \fillpoint*{P}[\(P\)][right];
\end{mathfigure*}
Aby obliczyć pole obszaru między okręgami i~ich styczną, musimy od pola trapezu \(YXST\) odjąć pole wycięte z~dużego koła zielonym kątem oraz pole wycięte z~małego koła pomarańczowym kątem. Zauważmy, że \(ST = 1 + 3 = 4 = 2 \cdot 2 = 2 \cdot SP\), a~kąt \(\angle{TPS}\) jest prosty, więc \(\triangle{TSP}\) to trójkąt \(30\degree\ 60\degree\ 90\degree\). Zatem odcinek \(TP\) będący wysokością trapezu \(YXST\) ma długość \(2\sqrt{3}\). Pole trapezu wynosi więc
\begin{equation*}
    \area{YXST}
        = \frac{\pars{YT + XP} \cdot TP}{2}
        = \frac{\pars{1 + 3} \cdot 2\sqrt{3}}{2}
        = 4\sqrt{3}
\end{equation*}
Skoro wiemy, że zielony kąt ma miarę \(60\degree\), to pole wycinka wyznaczonego nim w~dużym kole wynosi
\begin{equation*}
    \frac{60\degree}{360\degree} \cdot \pi \cdot 3^2
        = \frac{1}{6} \cdot 9\pi
        = \frac{3}{2}\pi
\end{equation*}
Wiemy też, że \(\mangle{STP} = 30\degree\) i~\(\mangle{PTY} = 90\degree\), więc pomarańczowy kąt ma miarę \(30\degree + 90\degree = 120\degree\). Wycinek małego koła ma więc pole równe
\begin{equation*}
    \frac{120\degree}{360\degree} \cdot \pi \cdot 1^2
        = \frac{\pi}{3}
\end{equation*}
Ostatecznie zatem, poszukiwane pole obszaru wynosi
\begin{equation*}
    4\sqrt{3} - \frac{3}{2}\pi - \frac{\pi}{3}
        = 4\sqrt{3} - \frac{11\pi}{6}
\end{equation*}
\subsubsection*{Zadanie~9.}
Oznaczmy długość przyprostokątnej tego trójkąta prostokątnego równoramiennego przez \(a\).
\begin{mathfigure*}
    \coordinate (A) at (0, 0);
    \coordinate (B) at (4, 0);
    \coordinate (C) at (0, 4);
    \coordinate (D) at (0, 2);
    \coordinate (E) at (2, 0);
    \coordinate (X) at (4/3, 4/3);
    \draw (B) -- (D);
    \draw (C) -- (E);
    \drawrightangle{B--A--C};
    \drawangle*[Orange]{B--X--C}[\(\alpha\)];
    \draw (A) node[below left]{\(A\)}
        -- (B) node[below right]{\(B\)}
        -- node[above right]{\(a\sqrt{2}\)} (C) node[above]{\(C\)}
        -- cycle;
    \path (A) -- node[below]{\(\frac{a}{2}\)} (E);
    \fillpoint*{X}[\(X\)][below left];
    \fillpoint*{D}[\(D\)][left];
    \fillpoint*{E}[\(E\)][below];
\end{mathfigure*}
\noindent
Możemy wyliczyć długość środkowej z~twierdzenia Pitagorasa:
\begin{equation*}
    BD = CE
        = \sqrt{a^2 + \frac{a^2}{4}}
        = \sqrt{\frac{5a^2}{4}}
        = \frac{a\sqrt{5}}{2}
\end{equation*}
Środkowe przecinają się w~stosunku \(2 : 1\), więc
\begin{equation*}
    BX = CX
        = \frac{2}{3} \cdot \frac{a\sqrt{5}}{2}
        = \frac{a\sqrt{5}}{3}
\end{equation*}
Z~twierdzenia cosinusów w~\(\triangle{BXC}\) mamy
\begin{gather*}
    CB^2
        = BX^2 + CX^2 - 2 \cdot BX \cdot CX \cdot \cos\alpha\\
    2a^2 = 2 \cdot \frac{5a^2}{9} - 2 \cdot \frac{5a^2}{9} \cdot \cos\alpha\\
    1 = \frac{5}{9}\pars{1 - \cos\alpha}\\
    1 - \cos\alpha = \frac{9}{5}\\
    \cos\alpha = -\frac{4}{5}
\end{gather*}
\subsubsection*{Zadanie~10.}
\begin{mathfigure*}
    \coordinate (S) at (0, 0);
    \coordinate (P) at (1.75, 0);
    \coordinate (A) at (-2.75, 0);
    \coordinate (B) at (2.75, 0);
    \coordinate (C) at (0, 2.75);
    \coordinate (D) at (2.49, -1.16);
    \draw (A) -- (B);
    \draw (C) -- (D);
    \draw (S) circle[radius=2.75];
    \fillpoint*{A}[\(A\)][left];
    \fillpoint*{B}[\(B\)][right];
    \fillpoint*{C}[\(C\)][above];
    \fillpoint*{D}[\(D\)][right];
    \fillpoint*{P}[\(P\)][below left];
    \fillpoint*{S}[\(S\)][above];
\end{mathfigure*}
Z~potęgi punktu mamy
\begin{gather*}
    CP \cdot PD
        = AP \cdot PB
        = \pars{AS + SP}\pars{BS - SP}
        = \pars{11 + 7}\pars{11 - 7}
        = 18 \cdot 4\\
    CP \cdot PD = 72
\end{gather*}
Skoro cięciwa \(CD\) ma długość \(18\), to mamy układ równań:
\begin{equation*}
    \begin{cases}
        CP + PD = 18\\
        CP \cdot PD = 72
    \end{cases}
\end{equation*}
Możemy wyliczyć \(CP\) z~drugiego równania i~podstawić do pierwszego:
\begin{gather*}
    CP = \frac{72}{PD}\\
    \frac{72}{PD} + PD = 18\\
    72 + PD^2 = 18 \cdot PD\\
    PD^2 - 18 \cdot PD + 72 = 0\\
    \pars{PD - 6}\pars{PD - 12} = 0
\end{gather*}
Zatem możliwe rozwiązania to:
\begin{equation*}
    \begin{cases}
        PD = 6\\
        CP = 12
    \end{cases}
    \lor
    \begin{cases}
        PD = 12\\
        CP = 6
    \end{cases}
\end{equation*}
W~obydwu przypadkach, stosunek w~którym \(P\) dzieli cięciwę wynosi
\begin{equation*}
    \frac{12}{6} = \frac{2}{1}
\end{equation*}
\subsubsection*{Zadanie~14.}
\begin{mathfigure*}
    \coordinate (A) at (-4, 0);
    \coordinate (B) at (4, 0);
    \coordinate (C) at (1, 4);
    \coordinate (D) at (-1, 4);
    \coordinate (H) at (-1, 0);
    \coordinate (I) at (0, 2);
    \draw[dashed] (D) -- node[right]{\(h\)} (H);
    \draw[ForestGreen] (I) circle[radius=2];
    \draw (A) node[below left]{\(A\)}
        -- node[below]{\(16\)} (B) node[below right]{\(B\)}
        -- node[above right]{\(\ell\)} (C) node[above right]{\(C\)}
        -- node[above]{\(4\)} (D) node[above left]{\(D\)}
        -- node[above left]{\(\ell\)} cycle;
    \path (A) -- node[above]{\(6\)} (H);
    \fillpoint*{H}[\(H\)][below];
\end{mathfigure*}
Skoro na trapezie można opisać okrąg, to jest on równoramienny.
\begin{equation*}
    AH
        = \frac{AB - CD}{2}
        = \frac{16 - 4}{2}
        = 6
\end{equation*}
Jeśli natomiast można w~niego wpisać okrąg, to sumy par przeciwległych boków są sobie równe:
\begin{gather*}
    AB + CD = AD + BC\\
    16 + 4 = 2\ell\\
    \ell = 10
\end{gather*}
Z~twierdzenia Pitagorasa w~\(\triangle{AHD}\) mamy
\begin{equation*}
    h
        = \sqrt{\ell^2 - 6^2}
        = \sqrt{100 - 36}
        = \sqrt{64}
        = 8
\end{equation*}
Wiemy, że \(h = 2r\), więc \(r = 4\).
