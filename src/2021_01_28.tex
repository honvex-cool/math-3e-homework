\subsubsection*{Zadanie~1.10.}
Załóżmy do dowodu nie wprost, że te trzy proste nie mają punktu wspólnego. Niech \(X\) będzie punktem przecięcia prostych \(AA'\) i~\(BB'\), \(Y\) punktem przecięcia \(BB'\) i~\(CC'\), a~\(Z\) punktem przecięcia \(CC'\) i~\(AA'\). Wtedy \(X \neq Y \neq Z\). Oznacza to, że te trzy punkty wyznaczają płaszczynę, w~której zawierają się proste \(AA'\), \(BB'\) i~\(CC'\). Tak jednak być nie może, ponieważ wtedy w~jednej płaszczyźnie leżałaby zarówno podstawa \(ABC\) jak i~punkty należąe do ścian bocznych ostrosłupa. Zatem założenie musiało być fałszywe, czyli proste \(AA'\), \(BB'\) i~\(CC'\) są współpękowe.
\qed
\subsubsection*{Zadanie~1.12.}
Oznaczmy przez \(V\) wierzchołek ostrosłupa, przez \(H\) spodek jego wysokości na podstawę, a~przez \(G\) przecięcie wysokości z~przekrojem \(ABCDEF\). Skoro mamy do czynienia z~ostrosłupem prawidłowym, to jego wysokość jest wysokością trójkąta \(\triangle{PSV}\), czyli leży w~płaszczyźnie \(PSV\), która jest tożsama z~płaszczyzną \(ADV\). Zatem prosta \(AD\) ma punkt wspólny z~wysokością, więc musi być nim punkt \(G\). Analogicznie pokazujemy, że \(G \in BE\) i~\(G \in CF\). Zatem proste \(AD\), \(BE\) i~\(CF\) są współpękowe.
\qed
\subsubsection*{Zadanie~1.13.}
\begin{enumerate}[label={\alph*)}]
    \item Rozważmy przeciwległe ściany boczne graniastosłupa. Ponieważ jest to graniastosłup prawidłowy, to płaszczyzny zawierające te ściany są równoległe. Zatem przeciwległe boki wyciętego sześciokąta zawierające się w~tych przeciwległych ścianach nie przecinają się. Ponieważ te przeciwległe boki leżą w~jednej płaszczyźnie --- tej, którą przecięto graniastosłup --- i~nie przecinają się, to mogą być tylko równoległe.
    \item Dowód dokładnie analogiczny jak w~zadaniu 1.12., przy czym zamiast dowolnej wysokości bierzemy prostą przechodzącą przez środki podstaw.
\end{enumerate}
\subsubsection*{Zadanie~1.14.}
Załóżmy, że proste \(AB\) i~\(LM\) się przecinają, i~oznaczmy ich punkt wspólny przez \(X\). Wtedy \(X \in ABD\) i~\(X \in KLM\), bo musi należeć do płaszczyzn, do których należą proste, których jest przecięciem. Jeżeli natomiast założymy, że \(AB\) i~\(LM\) są równoległe, to \(AB\) i~\(KN\) muszą także być równoległe, bo inaczej postępując analogicznie jak poprzednio pokazaliśmy, że \(AB\) i~\(LM\) mają punkt wspólny, a~założyliśmy że nie; natomiast \(AB\) i~\(KN\) nie mogą się nie przecinać i~nie być równoległe, bo leżą w~jednej płaszczyźnie \(ABC\).
\qed

