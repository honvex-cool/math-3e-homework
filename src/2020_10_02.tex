\subsection*{Zestaw~X (zadania otwarte)}
\subsubsection*{Zadanie~1.}
\begin{equation*}
    a_n = \frac{2n - 1}{n + 1}
\end{equation*}
Aby udowodnić, że ciąg jest rosnący, pokażemy, że różnica dwóch kolejnych wyrazów \(a_{n + 1} - a_n\) jest dodatnia dla każdego \(n \geq 0\):
\begin{equation*}
    \begin{split}
        a_{n + 1} - a_n
            &= \frac{2\pars{n + 1} - 1}{\pars{n + 1} + 1} - \frac{2n - 1}{n + 1}
            = \frac{2n + 1}{n + 2} - \frac{2n - 1}{n + 1}
            = \frac{\pars{2n + 1}\pars{n + 1}}{\pars{n + 2}\pars{n + 1}} - \frac{\pars{2n - 1}\pars{n + 2}}{\pars{n + 2}\pars{n + 1}}\\
            &= \frac{2n^2 + 3n + 1 - 2n^2 - 3n + 2}{\pars{n + 2}\pars{n + 1}}
            = \frac{3}{\pars{n + 2}\pars{n + 1}}
            > 0
    \end{split}
\end{equation*}
\qed
\subsubsection*{Zadanie~2.}
\begin{gather*}
    S_n = a_1 + a_2 + \ldots + a_n = 2n^2 - n\\
    \begin{split}
        a_{n + 2} - a_{n + 1}
            &= \pars{S_{n + 2} - S_{n + 1}} - \pars{S_{n + 1} - S_n}
            = S_{n + 2} - 2S_{n + 1} + S_n\\
            &= 2\pars{n + 2}^2 - \pars{n + 2} - 2\pars{2\pars{n + 1}^2 - \pars{n + 1}} + 2n^2 - n\\
            &= 2n^2 + 8n + 8 - n - 2 - 4n^2 - 8n - 4 + 2n + 2 + 2n^2 - n
            = 4
    \end{split}\\
    a_1 = S_1 = 2 \cdot 1^2 - 1 = 1\\
    a_2 = S_2 - S_1 = 2 \cdot 2^2 - 2 - 2 \cdot 1^2 + 1 = 8 - 2 - 2 + 1 = 5\\
    a_2 - a_1 = 5 - 1 = 4
\end{gather*}
Ponieważ różnica każdych dwóch kolejnych wyrazów wynosi \(4\), czyli jest stała, to jest to ciąg arytmetyczny.
\qed
\subsubsection*{Zadanie~3.}
\begin{gather*}
    S_n = a_1 + a_2 + \ldots + a_n = 4 \cdot \pars{5^n - 1}\\
    \begin{split}
        \frac{a_{n + 2}}{a_{n + 1}}
            &= \frac{S_{n + 2} - S_{n + 1}}{S_{n + 1} - S_n}
            = \frac{4 \cdot \pars{5^{n + 2} - 1} - 4 \cdot \pars{5^{n + 1} - 1}}{4 \cdot \pars{5^{n + 1} - 1} - 4 \cdot \pars{5^n - 1}}
            = \frac{5^{n + 2} - 1 - 5^{n + 1} + 1}{5^{n + 1} - 1 - 5^n + 1}
            = \frac{5^{n + 2} - 5^{n + 1}}{5^{n + 1} - 5^n}\\
            &= \frac{25 \cdot 5^n - 5 \cdot 5^n}{5 \cdot 5^n - 5^n}
            = \frac{20 \cdot 5^n}{4 \cdot 5^n}
            = 5
    \end{split}\\
    a_1 = S_1 = 4 \cdot \pars{5^1 - 1} = 16\\
    a_2 = S_2 - S_1 = 4 \cdot \pars{5^2 - 1} - 4 \cdot \pars{5^1 - 1} = 96 - 16 = 80\\
    \frac{a_2}{a_1} = \frac{80}{16} = 5
\end{gather*}
Ponieważ iloraz każdych dwóch kolejnych wyrazów wynosi \(5\), czyli jest stały, to jest to ciąg geometryczny.
\subsubsection*{Zadanie~4.}
\begin{equation*}
    x^2 + x^3 + x^4 + \ldots = \frac{4}{3}
\end{equation*}
Po lewej stronie mamy szereg geometryczny o~pierwszym wyrazie \(a_1 = x^2\) i~ilorazie \(q = x\). Jest zbieżny czyli \(\abs{q} = \abs{x} < 1\). Zatem dziedziną równania jest \(D = \open{-1}{1}\). W~tej sytuacji możemy zapisać równanie z~użyciem wzoru na sumę nieskończonego szeregu geometrycznego:
\begin{gather*}
    \frac{a_1}{1 - q} = \frac{4}{3}\\
    \frac{x^2}{1 - x} = \frac{4}{3}\\
    3x^2 = 4 - 4x\\
    3x^2 + 4x - 4 = 0\\
    \Delta = 4^2 - 4 \cdot 3 \cdot \pars{-4} = 16 + 48 = 64\\
    \sqrt{\Delta} = \sqrt{64} = 8\\
    x_1 = \frac{-4 - \sqrt{\Delta}}{2 \cdot 3} = \frac{-12}{6} = -2 \not\in D\\
    x_2 = \frac{-4 + \sqrt{\Delta}}{2 \cdot 3} = \frac{4}{6} = \frac{2}{3}
\end{gather*}
Zatem \(x = \frac{2}{3}\).
\subsubsection*{Zadanie~5.}
Na początek wyznaczmy granicę tego ciągu:
\begin{equation*}
    \limit \frac{4n + 1}{n + 2}
        = \frac{\cancel{n}\pars{4 + \converges{0}{\frac{1}{n}}}}{\cancel{n}\pars{1 + \converges{0}{\frac{2}{n}}}}
        = \frac{4}{1}
        = 4
\end{equation*}
Rozwiążmy teraz nierówności:
\begin{itemize}
    \item szacowanie górne:
        \begin{gather*}
            \frac{4n + 1}{n + 2} < 4{,}01\\
            4n + 1 < 4{,}01n + 8{,}02\\
            0{,}01n > -7{,}02\\
            n > -702
        \end{gather*}
    \item szacowanie dolne:
        \begin{gather*}
            \frac{4n + 1}{n + 2} > 3{,}99\\
            4n + 1 > 3{,}99n + 7{,}98\\
            0,01n > 6{,}98\\
            n > 698\\
            n \geq 699
        \end{gather*}
\end{itemize}
Wyrazy \(a_{699}\) i~wszystkie następne różnią się od granicy ciągu o~mniej niż \(0{,}01\).
\subsubsection*{Zadanie~6.}
\begin{equation*}
    a_n = 33n - n^3
\end{equation*}
\begin{equation*}
    a_{n + 1} - a_n
        = 33\pars{n + 1} - \pars{n + 1}^3 - 33n + n^3
        = 33n + 33 - n^3 - 3n^2 - 3n - 33n + n^3
        = -3n^2 - 3n + 33
\end{equation*}
\subsubsection*{Zadanie~7.}
Oznaczmy te trzy liczby przez \(a, b, c\). Skoro tworzą ciąg geometryczny, to \(b^2 = ac\). Ciąg arytmetyczny oznaczmy przez \(\sequence{a_n}\), a~jego różnicę przez \(r\). Wiemy, że
\begin{gather*}
    a_1 = a\\
    a_2 = b = a_1 + r = a + r\\
    a_5 = c = a_1 + 4r = a + 4r
\end{gather*}
Możemy podstawić te równości do równania wynikającego z~tego, że ciąg jest geometryczny:
\begin{gather*}
    b^2 = ac\\
    \pars{a + r}^2 = a \cdot \pars{a + 4r}\\
    a^2 + 2ar + r^2 = a^2 + 4ar\\
    r^2 = 2ar\\
    r = 2a
\end{gather*}
Możemy teraz podstawić następująco:
\begin{gather*}
    b = a_2 = a + r = a + 2a = 3a\\
    c = a_5 = a + 4r = a + 8a = 9a
\end{gather*}
Dodatkowo wiemy, że \(a + b + c = 91\), zatem
\begin{gather*}
    a + 3a + 9a = 91\\
    13a = 91\\
    a = 7\\
    b = 3a = 21\\
    c = 9a = 63
\end{gather*}
Te liczby to \(a = 7, b = 21, c = 63\).
\subsubsection*{Zadanie~8.}
Wiemy, że \(a_1 = 6\). Oznaczmy iloraz ciągu przez \(q\). Skoro suma nieskończonego ciągu jest zbieżna, to \(\abs{q} < 1\). Zauważmy też, że ciąg \(\sequence{b_n}\) kwadratów wyrazów ciągu \(\sequence{a_n}\) jest również ciągiem geometrycznym o~pierwszym wyrazie \(b_1 = a_1^2 = 6^2 = 36\) i~ilorazie \(q^2\). Tu również musi zachodzić \(\abs{q^2} < 1\), ponieważ suma tego nieskończonego ciągu jest zbieżna, ale to wynika z~faktu, że \(\abs{q} < 1\). Teraz ze wzorów na sumę szeregu geometrycznego:
\begin{gather*}
    \series a_n = \frac{a_1}{1 - q} = \frac{6}{1 - q}\\
    \series b_n = \frac{b_1}{1 - q^2} = \frac{36}{1 - q^2}
\end{gather*}
Wiemy, że
\begin{gather*}
    \series a_n = \frac{1}{8}\series b_n\\
    \frac{6}{1 - q} = \frac{1}{8} \cdot \frac{36}{1 - q^2}\\
    \frac{4}{1 - q} = \frac{3}{1 - q^2}\\
    4 - 4q^2 = 3 - 3q\\
    4q^2 - 3q - 1 = 0\\
    \pars{q - 1}\pars{4q + 1}\\
    q = 1 \wlor q = -\frac{1}{4}
\end{gather*}
Jednak wynik \(q = 1\) odrzucamy, ponieważ nie spełnia warunku \(\abs{q} < 1\). Zatem iloraz tego ciągu wynosi \(\frac{1}{4}\).
\subsubsection*{Zadanie~9.}
\begin{mathfigure*}
    \def\rt{\fpeval{sqrt(3)}}
    \coordinate (center) at (0, 0);
    \coordinate (A) at (9 / \rt, -3);
    \coordinate (B) at (-9 / \rt, -3);
    \coordinate (C) at (0, 6);
    \coordinate (a) at (4.5 / \rt, -1.5);
    \coordinate (b) at (-4.5 / \rt, -1.5);
    \coordinate (c) at (0, 3);
    \draw (center) circle [radius=6];
    \draw (A) -- node[below]{\(a_1 = 6\sqrt{3}\)} (B) -- (C) -- cycle;
    \draw (center) circle [radius=3];
    \draw (a) -- node[below]{\(a_2 = 3\sqrt{3}\)} (b) -- (c) -- cycle;
    \draw (center) circle [radius=1.5];
\end{mathfigure*}
Promień \(r_1\) pierwszego okręgu jest równy \(\frac{2}{3}\) wysokości trójkąta \(h_1\). Oznacza to, że \(h_1 = 9\). Ponadto \(h_1 = \frac{a_1\sqrt{3}}{2}\), zatem \(a_1 = 6\sqrt{3}\). Okrąg wpisany ma promień równy \(\frac{1}{3}\) wysokości trójkąta, czyli \(\frac{1}{2}\) promienia okręgu opisanego. Zatem \(r_2 = 3\), czyli \(\frac{3}{2}r_2 = \frac{9}{2} = h_2 = \frac{a_2\sqrt{3}}{2}\), więc \(a_2 = 3\sqrt{3}\). Widzimy, że boki kolejnych trójkątów tworzą ciąg geometryczny o~pierwszym wyrazie równym \(a_1 = 6\sqrt{3}\) i~ilorazie \(q = \frac{1}{2}\) (\(\abs{q} < 1\), więc szereg jest zbieżny). Pole trójkąta równobocznego o~boku \(a\) wynosi \(S = \frac{a^2\sqrt{3}}{4}\). Zatem jeśli obliczymy sumę kwadratów boków nieskończenie wielu trójkątów, a~następnie pomnożymy ją przez \(\frac{\sqrt{3}}{4}\), otrzymamy sumę pól nieskończenie wielu trójkątów. Ciąg kwadratów wyrazów ciągu geometrycznego sam jest ciągiem geometrycznym \(\sequence{b_n}\) o~pierwszym wyrazie \(b_1 = a_1^2 = 108\) i~ilorazie \(q^2 = \frac{1}{4}\). Możemy teraz obliczyć ze wzoru na sumę szeregu geometrycznego:
\begin{equation*}
    \series b_n
        = \frac{b_1}{1 - q^2}
        = \frac{108}{1 - \frac{1}{4}}
        = \frac{4}{3} \cdot 108
        = 144
\end{equation*}
Zatem suma pól nieskończenie wielu trójkątów wynosi \(\frac{144\sqrt{3}}{4} = 36\sqrt{3}\).
\subsubsection*{Zadanie~10.}
Oznaczmy ten ciąg geometryczny przez \(\sequence{a_n}\). Skoro jest zbieżny, to \(\abs{q} < 1\), gdzie \(q\) jest ilorazem tego ciągu. Zauważmy, że ciąg sześcianów wyrazów zbieżnego ciągu geometrycznego sam jest zbieżnym ciągiem geometrycznym \(\sequence{b_n}\) o~pierwszym wyrazie \(b_1 = a_1^3\) i~ilorazie \(q^3\). Możemy teraz zapisać ze wzoru na sumę szeregu geometrycznego:
\begin{gather*}
    \series a_n = \frac{a_1}{1 - q} = 4 \implies a_1 = 4 - 4q \implies a_1^2 = 16 - 32q + 16q^2\\
    \series b_n = \frac{b_1}{1 - q^3} = \frac{a_1^3}{1 - q^3} = 192\\
    \frac{a_1^3}{1 - q^3} = 48 \cdot \frac{a_1}{1 - q}\\
    \frac{a_1^2}{1 + q + q^2} = 48\\
    \frac{16 - 32q + 16q^2}{1 + q + q^2} = 48\\
    16q^2 - 32q + 16 = 48q^2 + 48q + 48\\
    q^2 - 2q + 1 = 3q^2 + 3q + 3\\
    2q^2 + 5q + 2 = 0\\
    \Delta = 5^2 - 4 \cdot 2 \cdot 2 = 25 - 16 = 9\\
    \sqrt{\Delta} = \sqrt{9} = 3\\
    q_1 = \frac{-5 - \sqrt{\Delta}}{2 \cdot 2} = \frac{-5 - 3}{4} = -2\\
    q_2 = \frac{-5 + \sqrt{\Delta}}{2 \cdot 2} = \frac{-5 + 3}{4} = -\frac{1}{2}
\end{gather*}
Odrzucamy \(q = -2\), ponieważ nie spełnia warunku \(\abs{q} < 1\). Zatem \(q = -\frac{1}{2}\), natomiast \(a_1 = 4 - 4q = 4 - 4 \cdot \pars{-\frac{1}{2}} = 6\). Zatem ciąg geometryczny jest zdefiniowany następująco:
\begin{equation*}
    \begin{cases}
        a_1 = 6\\
        q = -\frac{1}{2}
    \end{cases}
\end{equation*}
\subsubsection*{Zadanie~11.}
Oznaczmy te trzy liczby przez \(a, b, c\). Skoro tworzą ciąg arytmetyczny, to \(2b = a + c\). Ponadto ciąg \(\sequence{a + 1, b + 4, c + 19}\) jest geometryczny, czyli \(\pars{b + 4}^2 = \pars{a + 1}\pars{c + 19}\). Mamy zatem układ równań:
\begin{equation*}
    \begin{cases}
        a + b + c = 15\\
        2b = a + c \implies 2b - a - c = 0\\
        \pars{b + 4}^2 = \pars{a + 1}\pars{c + 19}
    \end{cases}
\end{equation*}
Możemy stronami dodać pierwsze i~drugie równanie, aby uzyskać:
\begin{gather*}
    3b = 15\\
    b = 5
\end{gather*}
Zatem \(a + c = 15 - b = 10\), czyli \(c = 10 - a\). Podstawmy to zatem do ostatniego równania:
\begin{gather*}
    \pars{5 + 4}^2 = \pars{a + 1}\pars{10 - a + 19}\\
    \pars{a + 1}\pars{29 - a} = 9^2\\
    29a - a^2 + 29 - a = 81\\
    a^2 - 28a + 52 = 0\\
    \Delta = \pars{-28}^2 - 4 \cdot 1 \cdot 52 = 784 - 208 = 576\\
    \sqrt{\Delta} = \sqrt{576} = 24\\
    a_1 = \frac{-\pars{-28} + \sqrt{\Delta}}{2 \cdot 1} = \frac{28 + 24}{2} = 26 \implies c_1 = -16\\
    a_2 = \frac{-\pars{-28} - \sqrt{\Delta}}{2 \cdot 1} = \frac{28 - 24}{2} = 2 \implies c_2 = 8
\end{gather*}
Mamy zatem dwa potencjalne ciągi:
\begin{equation*}
    \begin{cases}
        a = 26\\
        b = 5\\
        c = -16
    \end{cases}
    \wlor
    \begin{cases}
        a = 2\\
        b = 5\\
        c = 8
    \end{cases}
\end{equation*}
W~pierwszym przypadku różnica ciągu arytmetycznego wynosi \(-21\), a~w drugim \(3\). Po dodaniu odpowiednich wartości:
\begin{equation*}
    \begin{cases}
        a + 1 = 27\\
        b + 4 = 9\\
        c + 19 = 3
    \end{cases}
    \wlor
    \begin{cases}
        a + 1 = 3\\
        b + 4 = 9\\
        c + 19 = 27
    \end{cases}
\end{equation*}
otrzymamy ciągi geometryczne o~ilorazach odpowiednio \(\frac{1}{3}\) i~\(3\). Zatem wszystko się zgadza i~są dwa rozwiązania:
\begin{equation*}
    \sequence{a, b, c} \in \set{\sequence{26, 5, -16}, \sequence{2, 5, 8}}
\end{equation*}
