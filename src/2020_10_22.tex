\subsubsection*{Zestaw~XIII (trygonometria)}
\subsubsection*{Zadanie~1.}
\begin{gather*}
    \begin{split}
        \sin47\degree + \cos47\degree + \sin43\degree + \cos43\degree
            &= \sin47\degree + \sin43\degree + \sin43\degree + \sin47\degree
            = 2\pars{\sin47\degree + \sin43\degree}\\
            &= 2\pars{2\sin\frac{47\degree + 43\degree}{2}\cos\frac{47\degree - 43\degree}{2}}
            = 2 \cdot 2\sin\frac{90\degree}{2}\cos\frac{4\degree}{2}\\
            &= 2 \cdot 2\sin45\degree\cos2\degree
            = 2 \cdot 2 \cdot \frac{\sqrt{2}}{2} \cdot \cos2\degree
            = 2\cos2\degree
    \end{split}
\end{gather*}
\qed
\subsubsection*{Zadanie~2.}
\begin{gather*}
    \frac{\cos2\alpha}{1 + \sin2\alpha} = \frac{1 - \tan\alpha}{1 + \tan\alpha}\\
    1 + \sin\alpha \neq 0 \implies \sin\alpha \neq -1 \implies \alpha \neq 2k\pi + \frac{3\pi}{2}, k \in \integer\\
    1 + \tan\alpha \neq 0 \implies \tan\alpha \neq -1 \implies \alpha \neq k\pi + \frac{3\pi}{4}, k \in \integer\\
    R
        = \frac{1 - \tan\alpha}{1 + \tan\alpha} = \frac{1 - \frac{\sin\alpha}{\cos\alpha}}{1 + \frac{\sin\alpha}{\cos\alpha}}
        = \frac{\frac{\cos\alpha - \sin\alpha}{\cancel{\cos x}}}{\frac{\cos\alpha + \sin\alpha}{\cancel{\cos\alpha}}}
        = \frac{\cos\alpha - \sin\alpha}{\cos\alpha + \sin\alpha}\\
    L
        = \frac{\cos2\alpha}{1 + \sin2\alpha} = \frac{\cos^2\alpha - \sin^2\alpha}{\sin^2\alpha + \cos^2\alpha + 2\sin\alpha\cos\alpha}
        = \frac{\cancel{\pars{\cos\alpha + \sin\alpha}}\pars{\cos\alpha - \sin\alpha}}{\pars{\cos\alpha + \sin\alpha}^{\cancel{2}}}
        = \frac{\cos\alpha - \sin\alpha}{\cos\alpha + \sin\alpha}\\
    L = R
\end{gather*}
\qed
\subsubsection*{Zadanie~3.}
\begin{gather*}
    x, y \in \open{0}{\frac{\pi}{2}}\\
    \sin x + \sin y = \frac{4}{3} \wland \cos x + \cos y = \frac{2\sqrt{5}}{3}\\
    2\sin\frac{x + y}{2}\cos\frac{x - y}{2} = \frac{4}{3} \wland 2\cos\frac{x + y}{2}\cos\frac{x - y}{2} = \frac{2\sqrt{5}}{3}\\
    \sin\frac{x + y}{2}\cos\frac{x - y}{2} = \frac{2}{3} \wland \cos\frac{x + y}{2}\cos\frac{x - y}{2} = \frac{\sqrt{5}}{3}\\
    \frac{\sqrt{5}}{2}\sin\frac{x + y}{2}\cos\frac{x - y}{2} = \cos\frac{x + y}{2}\cos\frac{x - y}{2}
\end{gather*}
Skoro \(x, y \in \open{0}{\frac{\pi}{2}}\), to również \(\frac{x + y}{2} \in \open{0}{\frac{\pi}{2}}\). W~tym przedziale funkcja \(\sin\alpha\) nie ma miejsc zerowych, więc \(\sin\frac{x + y}{2} \neq 0\). Zatem możemy podzielić obustronnie równanie przez \(\sin\frac{x + y}{2}\):
\begin{gather*}
    \frac{\sqrt{5}}{2}\sin\frac{x + y}{2} = \cos\frac{x + y}{2}\\
    \frac{5}{4}\sin^2\pars{\frac{x + y}{2}} = \cos^2\pars{\frac{x + y}{2}}\\
    \frac{5}{4}\sin^2\pars{\frac{x + y}{2}} = 1 - \sin^2\pars{\frac{x + y}{2}}\\
    \frac{9}{4}\sin^2\pars{\frac{x + y}{2}} = 1\\
    \sin^2\pars{\frac{x + y}{2}} = \frac{4}{9}\\
    \sin\frac{x + y}{2} = \frac{2}{3}
\end{gather*}
Wiemy, że \(\sin\frac{x + y}{2}\cos\frac{x - y}{2} = \frac{2}{3}\). Podstawiając, mamy:
\begin{gather*}
    \frac{2}{3}\cos\frac{x - y}{2} = \frac{2}{3}\\
    \cos\frac{x - y}{2} = 1\\
    \frac{x - y}{2} = 2k\pi \qquad k \in \integer
    x - y = 4k\pi
\end{gather*}
Skoro \(x, y \in \open{0}{\frac{\pi}{2}}\), to \(x - y = 0\), czyli \(x = y\).
\qed
\subsubsection*{Zadanie~4.}
\begin{gather*}
    \begin{split}
        \frac{1}{\sin10\degree} - \frac{\sqrt{3}}{\cos10\degree}
            &= \frac{\cos10\degree - \sqrt{3}\sin10\degree}{\sin10\degree\cos10\degree}
            = \frac{2\pars{\frac{1}{2}\cos10\degree - \frac{\sqrt{3}}{2}\sin10\degree}}{\sin10\degree\cos10\degree}
            = \frac{2\pars{\sin30\degree - \cos30\degree\sin10\degree}}{\sin10\degree\cos10\degree}\\
            &= \frac{2\sin\pars{30\degree - 10\degree}}{\sin10\degree\cos10\degree}
            = \frac{2\cancel{\sin20\degree}}{\frac{1}{2}\cancel{\sin20\degree}}
            = \frac{2}{\frac{1}{2}}
            = 4
    \end{split}
\end{gather*}
\qed
\subsubsection*{Zadanie~5.}
\begin{gather*}
    \sin2x = 2\cos x\\
    2\sin x\cos x = 2\cos x\\
    2\sin x\\
    \sin x\cos x = \cos x
\end{gather*}
Jeśli \(\cos x = 0\), to równanie jest spełnione i~mamy rozwiązania \(x = k\pi + \frac{\pi}{2}\), gdzie \(k \in \integer\). Jeśli \(\cos x \neq 0\), to dzielimy obustronnie przez \(\cos x\):
\begin{gather*}
    \sin x = 1\\
    x = 2k\pi + \frac{\pi}{2} \qquad k \in \integer
\end{gather*}
Zatem po połączeniu wyników wnioskujemy, że \(x = k\pi + \frac{\pi}{2}\), gdzie \(k \in \integer\). Rozwiążmy teraz nierówność
\begin{gather*}
    x^2 - 4x - 32 < 0\\
    \pars{x + 4}\pars{x - 8} < 0\\
    \upparabola{-4}{8}\\
    x \in \open{-4}{8}
\end{gather*}
Poszukajmy teraz rozwiązań równania \(\sin2x = 2\cos x\) zawierających się w~otrzymanym przedziale (będziemy używać szacowania \(3{,}14 < \pi < 3{,}15\)):
\begin{gather*}
    -2k\pi + \frac{\pi}{2} < -6{,}28 + 1{,}6 = -4{,}68 < -4 \text{ (za małe)}\\
    -4 = -3{,}15 + 1{,15} < -\pi + \frac{\pi}{2} = -\frac{\pi}{2} < 0\\
    -4 < 0 < \frac{\pi}{2} < 3 < 8\\
    -4 < 0 < \frac{3\pi}{2} < 6 < 8\\
    -4 < 0 < \frac{5\pi}{2} < 6{,}3 + 1{,}6 = 7{,}9 < 8\\
    \frac{7\pi}{2} > 9 \text{ (za duże)}
\end{gather*}
Zatem rozwiązania znajdujące się w~rozważanym przedziale to \(x \in \set{-\frac{\pi}{2}, \frac{\pi}{2}, \frac{3\pi}{2}, \frac{5\pi}{2}}\).
\subsubsection*{Zadanie~6.}
\begin{equation*}
    \cos\alpha = \frac{m^2 - 4m - 4}{m^2 + 1} \qquad \alpha \in \open{0}{\frac{\pi}{3}}
\end{equation*}
Funkcja \(\cos\alpha\) przyjmuje w~tym przedziale wszystkie wartości z~przedziału \(\open{\frac{1}{2}}{1}\) (bo \(\cos0 = 1\), \(\cos\frac{\pi}{3} = \frac{1}{2}\), funkcja jest w~tym przedziale malejąca i~jest ciągła). Zatem
\begin{gather*}
    \frac{1}{2} < \frac{m^2 - 4m - 4}{m^2 + 1} < 1\\
    \frac{m^2 - 4m - 4}{m^2 + 1} > \frac{1}{2}
\end{gather*}
Liczby \(2\) i~\(m^2 + 1\) są dodatnie, więc możemy pomnożyć nierówność stronami przez mianowniki:
\begin{gather*}
    2m^2 - 8m - 8 > m^2 + 1\\
    m^2 - 8m - 9 > 0\\
    \pars{m + 1}\pars{m - 9} > 0\\
    \upparabola{-1}{9}\\
    x \in \open{-\infty}{-1} \cup \open{9}{+\infty}
\end{gather*}
Rozwiążmy drugą część nierówności:
\begin{equation*}
    \frac{m^2 - 4m - 4}{m^2 + 1} < 1
\end{equation*}
Możemy pomnożyć przez \(m^2 + 1\), ponieważ to wyrażenie jest zawsze dodatnie:
\begin{gather*}
    m^2 - 4m - 4 < m^2 + 1\\
    -4m - 5 < 0\\
    4m + 5 > 0\\
    x \in \open{-\frac{5}{4}}{+\infty}
\end{gather*}
Po wyznaczeniu części wspólnej uzyskanych przedziałów otrzymujemy
\begin{equation*}
    x \in \open{-\frac{5}{4}}{-1} \cup \open{9}{+\infty}
\end{equation*}
\subsubsection*{Zadanie~7.}
\begin{gather*}
    f\pars{x} = 3\sin^2x - 6\sin x + 1 \qquad x \in \real\\
    t \coloneqq \sin x \qquad t \in \closed{-1}{1}\\
    3t^2 - 6t + 1
\end{gather*}
Jest to funkcja kwadratowa zmiennej \(t\). Ramiona paraboli są skierowane w~stronę rosnących współrzędnych \(y\), więc istnieje globalna wartość najmniejsza przyjmowana dla \(t = \frac{-b}{2a} = \frac{-\pars{-6}}{2 \cdot 3} = 1 \in \closed{-1}{1}\) i~wynosząca \(\frac{-\Delta}{4a} = \frac{-\pars{\pars{-6}^2 - 4 \cdot 3 \cdot 1}}{4 \cdot 3} = \frac{-\pars{36 - 12}}{12} = -2\). Globalna wartość największa nie istnieje, ale wartość na drugiej krawędzi przedziału domkniętego, czyli \(t = -1\), jest wartością największą w~tym przedziale:
\begin{equation*}
    3 \cdot \pars{-1}^2 - 6 \cdot \pars{-1} + 1 = 10
\end{equation*}
Zatem wartość najmniejsza funkcji \(f\) jest przyjmowana dla \(\sin x = t_\p{min} = 1\), czyli \(x = 2k\pi + \frac{\pi}{2}\), gdzie \(k \in \integer\), i~wynosi ona \(-2\). Natomiast wartość największa jest przyjmowana dla \(\sin x = t_\p{max} = -1\), czyli \(x = 2k\pi + \frac{3\pi}{2}\), gdzie \(k \in \integer\), i~wynosi ona \(10\).
\subsubsection*{Zadanie~8.}
\begin{gather*}
    f\pars{x} = \cos^2x + \abs{\sin x} \cdot \sin x\\
    x \in \closed{-\frac{5}{2}\pi}{\frac{5}{2}\pi}
\end{gather*}
Rozważmy dwa przypadki:
\begin{proofcases}
    \item \(\sin x \leq 0 \implies x \in \closed{-2\pi}{-\pi} \cup \closed{0}{\pi} \cup \closed{2\pi}{\frac{5}{2}\pi}\)
        \begin{equation*}
            f\pars{x} = \cos^2x + \sin^2x = 1
        \end{equation*}
        W~tych przedziałach jest to funkcja stała równa \(1\).
        \begin{mathfigure*}
            \drawcoordsystem{-5*pi/2, -1.5}{5*pi/2, 1.5};
            \draw[ForestGreen, thick, smooth, domain=-5*pi/2:5*pi/2] plot (\x, {1});
        \end{mathfigure*}
    \item \(\sin x < 0 \implies x \in \leftclosed{-\frac{5}{2}\pi}{-2\pi} \cup \open{-\pi}{0} \cup \open{\pi}{2\pi}\)
        \begin{equation*}
            f\pars{x} = \cos^2x - \sin^2x = \cos2x
        \end{equation*}
        Jest to przekształcenie wykresu \(y = \cos x\) w~powinowactwie o~skali \(\frac{1}{2}\) względem osi \(Oy\):
        \begin{mathfigure*}
            \drawcoordsystem{-5*pi/2, -1.5}{5*pi/2, 1.5};
            \draw[RoyalBlue, thick, smooth, domain=-5*pi/2:5*pi/2, samples=150] plot (\x, {cos(2*\x r)});
        \end{mathfigure*}
\end{proofcases}
Łączymy te wykresy ze sobą w~odpowiednich przedziałach i~uzyskujemy ostateczny wygląd wykresu:
\begin{mathfigure*}
    \drawcoordsystem{-5*pi/2, -1.5}{5*pi/2, 1.5};
    \draw[RoyalBlue, thick, smooth, domain=-5*pi/2:-2*pi, samples=20] plot (\x, {cos(2*\x r)});
    \draw[RoyalBlue, thick, smooth, domain=pi:2*pi, samples=20] plot (\x, {cos(2*\x r)});
    \draw[RoyalBlue, thick, smooth, domain=-pi:0, samples=150] plot (\x, {cos(2*\x r)});
    \draw[ForestGreen, thick, smooth, domain=-2*pi:-pi] plot (\x, {1});
    \draw[ForestGreen, thick, smooth, domain=0:pi] plot (\x, {1});
    \draw[ForestGreen, thick, smooth, domain=2*pi:5*pi/2] plot (\x, {1});
    \fillpoint*{-pi/4, 0}[\(-\frac{\pi}{4}\)][below left];
    \fillpoint*{-3*pi/4, 0}[\(-\frac{3\pi}{4}\)][below right];
    \fillpoint*{-9*pi/4, 0}[\(-\frac{9\pi}{4}\)][above left];
    \fillpoint*{5*pi/4, 0}[\(\frac{5\pi}{4}\)][below right];
    \fillpoint*{7*pi/4, 0}[\(\frac{7\pi}{4}\)][below left];
    \fillpoint*[1][ForestGreen][ForestGreen]{5*pi/2, 1}[\(\pars{\frac{5\pi}{2}; 1}\)];
    \fillpoint*[1][RoyalBlue][RoyalBlue]{-5*pi/2, -1}[\(\pars{-\frac{5\pi}{2}; -1}\)][below];
    \draw[densely dotted] (-pi/2, -1) -- (-pi/2, 0) node[above]{\(-\frac{\pi}{2}\)};
    \draw[densely dotted] (-pi, 1) -- (-pi, 0) node[below]{\(-\pi\)};
    \draw[densely dotted] (-2*pi, 1) -- (-2*pi, 0) node[below]{\(-2\pi\)};
    \draw[densely dotted] (pi, 1) -- (pi, 0) node[below]{\(\pi\)};
    \draw[densely dotted] (2*pi, 1) -- (2*pi, 0) node[below]{\(2\pi\)};
    \draw[densely dotted] (3*pi/2, -1) -- (3*pi/2, 0) node[above]{\(\frac{3\pi}{2}\)};
\end{mathfigure*}
\subsubsection*{Zadanie~9.}
\begin{gather*}
    \cos\frac{\pi}{3} \cdot \sin4x - \sin\frac{\pi}{3} \cdot \cos4x = \frac{\sqrt{3}}{2} \qquad x \in \real\\
    \sin\pars{4x - \frac{\pi}{3}} = \frac{\sqrt{3}}{2} = \sin\frac{\pi}{3}\\
    4x - \frac{\pi}{3} = 2k\pi + \frac{\pi}{3} \wlor 4x - \frac{\pi}{3} = 2k\pi + \frac{2\pi}{3} \qquad k \in \integer\\
    4x = 2k\pi + \frac{2\pi}{3} \wlor 4x = 3k\pi \qquad k \in \integer\\
    x = \frac{k\pi}{2} + \frac{\pi}{6} \wlor x = \frac{3k\pi}{4} \qquad k \in \integer
\end{gather*}
Najmniejsze rozwiązanie dodatnie otrzymamy, podstawiając \(k = 0\) lub \(k = 1\) (ponieważ zauważamy, że dla \(k < 0\) rozwiązania są ujemne):
\begin{gather*}
    x = \frac{\pi}{6} \lor x = \frac{3\pi}{4}
\end{gather*}
Pierwsze z~tych rozwiązań jest mniejsze. Zatem najmniejsze rozwiązanie dodatnie to \(x = \frac{\pi}{6}\).
\subsubsection*{Zadanie~10.}
\begin{gather*}
    2\sin\pars{x - \frac{\pi}{2}} = 2\abs{k - \frac{1}{2}} - 5\\
    x \in \real\\
    -2 \leq 2\sin\pars{x - \frac{\pi}{2}} \leq 2
\end{gather*}
Równanie z~funkcją \(\sin\) na pewno będzie miało rozwiązanie, jeśli wartość po prawej stronie będzie znajdować się w~odpowiednim przedziale, ponieważ jest to funkcja siągła.
\begin{gather*}
    2\abs{k - \frac{1}{2}} - 5 \geq -2\\
    2\abs{k - \frac{1}{2}} \geq 3\\
    \abs{k - \frac{1}{2}} \geq \frac{3}{2}\\
    k \in \closed{-\infty}{-1} \cup \closed{2}{+\infty}\\
    2\abs{k - \frac{1}{2}} - 5 \leq 2\\
    2\abs{k - \frac{1}{2}} \leq 7\\
    \abs{k - \frac{1}{2}} \leq \frac{7}{2}\\
    k \in \closed{-3}{4}
\end{gather*}
Rozwiązaniem jest część wspólna uzyskanych przedziałów:
\begin{equation*}
    k \in \closed{-3}{-1} \cup \closed{2}{4}
\end{equation*}
\subsubsection*{Zadanie~11.}
\begin{gather*}
    \pars{2\sin\alpha - 1}x^2 - 2x + \sin\alpha = 0\\
    \alpha \in \closed{-\frac{\pi}{2}}{\frac{\pi}{2}}
\end{gather*}
Aby to równanie miało dwa pierwiastki, musi być kwadratowe. Wyraz wolny nie może być zerem, ponieważ wtedy jednym z~pierwiastków byłoby \(0\), czyli nie miałoby odwrotności. Zatem \(2\sin\alpha - 1 \neq 0\), czyli \(\alpha \neq \frac{\pi}{6}\), i~\(\sin\alpha \neq 0\), czyli \(\alpha \neq 0\). Wyróżnik musi być nieujemny:
\begin{gather*}
    \Delta
        = \pars{-2}^2 - 4\pars{2\sin\alpha - 1}\sin\alpha
        =  4 - 8\sin^2\alpha + 4\sin\alpha\\
    -8\sin^2\alpha + 4\sin\alpha + 4 \geq 0\\
    -2\sin^2\alpha + \sin\alpha + 1 \geq 0\\
    2\sin^2\alpha - \sin\alpha - 1 \leq 0\\
    t \coloneqq \sin\alpha\\
    2t^2 - t - 1 \leq 0\\
    \pars{2t + 1}\pars{t - 1} \leq 0\\
    \upparabola{-\frac{1}{2}}{1}\\
    t \in \closed{-\frac{1}{2}}{1}\\
    \sin\alpha \in \closed{-\frac{1}{2}}{1} \implies \alpha \in \closed{-\frac{\pi}{6}}{\frac{\pi}{2}}
\end{gather*}
Ostatecznie zatem \(\alpha \in \leftclosed{-\frac{\pi}{6}}{0} \cup \open{0}{\frac{\pi}{6}} \cup \rightclosed{\frac{\pi}{6}}{\frac{\pi}{2}}\). Teraz możemy zastosować wzory Viete'a:
\begin{gather*}
    \frac{1}{x_1} + \frac{1}{x_2} = \frac{x_1 + x_2}{x_1x_2} = \frac{\frac{-\pars{-2}}{2\sin\alpha - 1}}{\frac{\sin\alpha}{2\sin\alpha - 1}} = \frac{2}{\sin\alpha}\\
    \frac{2}{\sin\alpha} = 4\cos\alpha\\
    4\sin\alpha\cos\alpha = 2\\
    2\sin2\alpha = 2\\
    \sin2\alpha = 1\\
    2\alpha = \frac{\pi}{2}\\
    \alpha = \frac{\pi}{4} \in \leftclosed{-\frac{\pi}{6}}{0} \cup \open{0}{\frac{\pi}{6}} \cup \rightclosed{\frac{\pi}{6}}{\frac{\pi}{2}}
\end{gather*}
Zatem rozwiązaniem zadania jest \(\alpha = \frac{\pi}{4}\).
\subsubsection*{Zadanie~12.}
\begin{gather*}
    \cos\alpha = \frac{9}{41}\\
    \alpha \in \open{0}{\frac{\pi}{2}}
\end{gather*}
Z~użyciem jedynki trygonometrycznej możemy wyliczyć \(\sin\alpha\):
\begin{gather*}
    \sin^2\alpha + \cos^2\alpha = 1\\
    \sin^2\alpha = 1 - \cos^2\alpha = 1 - \pars{\frac{9}{41}}^2 = 1 - \frac{81}{1681} = \frac{1600}{1681}
\end{gather*}
Ponieważ wiemy, że \(\alpha \in \open{0}{\frac{\pi}{2}}\), to mamy pewność, że \(\sin\alpha\) jest dodatni. Zatem możemy spierwiastkować obie strony równania:
\begin{equation*}
    \sin\alpha = \sqrt{\frac{1600}{1681}} = \frac{40}{41}
\end{equation*}
Wyrażenie \(\tan\pars{\alpha - \frac{\pi}{4}}\) możemy inaczej zapisać jako
\begin{equation*}
    \frac{\sin\pars{\alpha - \frac{\pi}{4}}}{\cos\pars{\alpha - \frac{\pi}{4}}}
        = \frac{\sin\alpha\cos\frac{\pi}{4} - \sin\frac{\pi}{4}\cos\alpha}{\cos\alpha\cos\frac{\pi}{4} + \sin\alpha\sin\frac{\pi}{4}}
        = \frac{\frac{40}{41} \cdot \frac{\sqrt{2}}{2} - \frac{\sqrt{2}}{2} \cdot \frac{9}{41}}{\frac{9}{41} \cdot \frac{\sqrt{2}}{2} + \frac{40}{41} \cdot \frac{\sqrt{2}}{2}}
        = \frac{\cancel{\frac{\sqrt{2}}{2}} \cdot \frac{31}{\cancel{41}}}{\cancel{\frac{\sqrt{2}}{2}} \cdot \frac{49}{\cancel{41}}}
        = \frac{31}{49}
\end{equation*}
Zatem \(\tan\pars{\alpha - \frac{\pi}{4}} = \frac{31}{49}\).
\subsubsection*{Zadanie~13.}
\begin{gather*}
    \sin x\pars{\cos x - \frac{1}{2}} < 0\\
    x \in \closed{0}{2\pi}\\
    \sin x\cos x - \frac{1}{2}\sin x < 0\\
    \frac{1}{2}\sin2x < \frac{1}{2}\sin x\\
    \sin2x < \sin x
\end{gather*}
Na początek rozwiążmy równość:
\begin{gather*}
    \sin2x = \sin x\\
    2x = 2k\pi + x \wlor 2x = 2k\pi + \pi - x\\
    x = 2k\pi \wlor 3x = 2k\pi + \pi\\
    x = 2k\pi \wlor x = \frac{2k\pi}{3} + \frac{\pi}{3}\\
    x \in \set{0, \frac{\pi}{3}, \pi, 2\pi}
\end{gather*}
Teraz możemy narysować wykres i~rozwiązać nierówność graficznie. Najpierw rysujemy wykres \(y = \sin x\):
\begin{mathfigure*}
    \drawcoordsystem{0, -1.5}{2*pi + 0.5, 1.5};
    \draw[ForestGreen, thick, domain=0:2*pi, smooth] plot (\x, {sin(\x r)});
    \fillpoint*{pi, 0}[\(\pi\)][below];
    \fillpoint*{2*pi, 0}[\(2\pi\)][above];
\end{mathfigure*}
\noindent
Następnie nakładamy na niego wykres \(y = \sin2x\), czyli wykres \(y = \sin x\) przekształcony w~powinowactwie o~skali \(\frac{1}{2}\) względem osi \(Oy\):
\begin{mathfigure*}
    \drawcoordsystem{0, -1.5}{2*pi + 0.5, 1.5};
    \draw[ForestGreen, thick, domain=0:2*pi, smooth] plot (\x, {sin(\x r)});
    \draw[RoyalBlue, thick, domain=0:2*pi, smooth] plot (\x, {sin(2*\x r)});
    \fillpoint*{pi/2, 0}[\(\frac{\pi}{2}\)][below];
    \fillpoint*{pi, 0}[\(\pi\)][below];
    \fillpoint*{3*pi/2, 0}[\(\frac{3\pi}{2}\)][below];
    \fillpoint*{2*pi, 0}[\(2\pi\)][above];
    \fillpoint*{pi/3, \fpeval{sqrt(3)/2}}[\(\pars{\frac{\pi}{3}, \frac{\sqrt{3}}{2}}\)][above];
    \fillpoint*{5*pi/3, -\fpeval{sqrt(3)/2}}[\(\pars{\frac{5\pi}{3}, \frac{-\sqrt{3}}{2}}\)][below];
    \draw[densely dotted] (pi/4, 1) -- (pi/4, 0) node[below]{\(\frac{\pi}{4}\)};
    \draw[densely dotted] (3*pi/4, -1) -- (3*pi/4, 0) node[above]{\(\frac{3\pi}{4}\)};
    \draw[densely dotted] (5*pi/4, 1) -- (5*pi/4, 0) node[below]{\(\frac{5\pi}{4}\)};
    \draw[densely dotted] (7*pi/4, -1) -- (7*pi/4, 0) node[above]{\(\frac{7\pi}{4}\)};
\end{mathfigure*}
Teraz możemy odczytać z~wykresu zbiór rozwiązań nierówności \(\sin2x < \sin x\):
\begin{equation*}
    x \in \open{\frac{\pi}{3}}{\pi} \cup \open{\frac{5\pi}{3}}{2\pi}
\end{equation*}
