% \subsection*{Zestaw~I (zadania otwarte)}
% \subsubsection*{Zadanie~1.}
% Oznaczmy przez \(n\) tę liczbę parzystą podzielną przez \(4\). Możemy to zapisać jako \(n \equiv 2 \pmod{4}\). Rozważmy, jakie reszty \(\bmod\ 4\) może dawać kwadrat liczby całkowitej \(x\):
% \begin{gather*}
%     x \equiv 0 \pmod{4} \implies x^2 \equiv 0 \pmod{4}\\
%     x \equiv 1 \pmod{4} \implies x^2 \equiv 1 \pmod{4}\\
%     x \equiv 2 \pmod{4} \implies x^2 \equiv 4 \equiv 0 \pmod{4}\\
%     x \equiv 3 \pmod{4} \implies x^2 \equiv 9 \equiv 1 \pmod{4}
% \end{gather*}
% Zatem możliwe reszty z~dzielenia kwadratu liczby całkowitej przez \(4\) to \(0\) i~\(1\). Załóżmy zatem, że \(n = p^2 - q^2\), gdzie \(p, q \in \natural\), i~rozważmy to równanie \(\bmod\ 4\):
% \begin{equation*}
%     n \equiv 2 \equiv p^2 - q^2 \pmod{4}
% \end{equation*}
% Możemy teraz spojrzeć na możliwe przypadki:
% \begin{gather*}
%     p^2 \equiv 0 \pmod{4} \land q^2 \equiv 0 \pmod{4} \implies 2 \equiv 0 - 0 \equiv 0 \pmod{4}\\
%     p^2 \equiv 1 \pmod{4} \land q^2 \equiv 0 \pmod{4} \implies 2 \equiv 1 - 0 \equiv 1 \pmod{4}\\
%     p^2 \equiv 0 \pmod{4} \land q^2 \equiv 1 \pmod{4} \implies 2 \equiv 0 - 1 \equiv -1 \equiv 3 \pmod{4}\\
%     p^2 \equiv 1 \pmod{4} \land q^2 \equiv 1 \pmod{4} \implies 2 \equiv 1 - 1 \equiv 0 \pmod{4}
% \end{gather*}
% W~każdym z~przypadków otrzymaliśmy oczywiste sprzeczności, zatem założenie, że \(n = p^2 - q^2\) musiało być fałszywe.
% \qed
% \subsubsection*{Zadanie~2.}
% Niech liczby \(n - 1, n, n + 1\) będą tymi trzema kolejnymi liczbami naturalnymi (\(n > 0\)). Zapiszmy teraz sumę ich kwadratów:
% \begin{equation*}
%     \tag{\(\star\)} (n - 1)^2 + n^2 + (n + 1)^2 = n^2 - 2n + 1 + n^2 + n^2 + 2n + 1 = 3n^2 + 2 \label{2020_09_21:sum_of_squares}
% \end{equation*}
% Rozważmy teraz, jakie reszty \(\bmod\ 3\) może dawać kwadrat liczby całkowitej \(x\):
% \begin{gather*}
%     x \equiv 0 \pmod{3} \implies x^2 \equiv 0 \pmod{3}\\
%     x \equiv 1 \pmod{3} \implies x^2 \equiv 1 \pmod{3}\\
%     x \equiv 2 \pmod{3} \implies x^2 \equiv 4 \equiv 1 \pmod{3}
% \end{gather*}
% Teraz rozważmy równanie (\ref{2020_09_21:sum_of_squares}) \(\bmod\ 3\):
% \begin{equation*}
%     3n^2 + 2 \equiv 2 \pmod{3}
% \end{equation*}
% Ponieważ suma kwadratów trzech kolejnych liczb całkowitych daje resztę \(2 \pmod{3}\), to nie może być kwadratem liczby całkowitej, ponieważ, jak ustaliliśmy wyżej, kwadrat liczby całkowitej nigdy nie daje reszty \(2 \pmod{3}\).
% \qed
% \subsubsection*{Zadanie~3.}
% \begin{equation*}
%     \frac{n^3}{6} + \frac{n^2}{2} + \frac{n}{3} = \frac{n^3}{6} + \frac{3n^2}{6} + \frac{2n}{6}
%         = \frac{n^3 + 3n^2 + 2n}{6}
%         = \frac{n \parens{n^2 + 3n + 2}}{6}
%         = \frac{n(n + 1)(n + 2)}{6}
% \end{equation*}
% Liczby \(n, n + 1, n + 2\) to trzy kolejne liczby naturlane, zatem przynajmniej jedna z nich jest podzielna przez \(2\) i~przynajmniej jedna z~nich jest podzielna przez \(3\). Zatem ich iloczyn jest podzielny przez \(6\), a~zatem cały ułamek jest liczbą całkowitą.
% \qed
% \subsubsection*{Zadanie~4.}
% \begin{equation*}
%     \begin{split}
%         \parens{5 - 3\sqrt{2}} &\cdot \sqrt{43 + 5\sqrt{2}} = \parens{5 - 3\sqrt{2}} \cdot \sqrt{25 + 5 \cdot \sqrt{36} \cdot \sqrt{2} + 18}
%             = \parens{5 - 3\sqrt{2}} \cdot \sqrt{5^2 + 30\sqrt{2} + \parens{3\sqrt{2}}^2}\\
%             &= \parens{5 - 3\sqrt{2}} \cdot \sqrt{\parens{5 + 3\sqrt{2}}^2}
%             = \parens{5 - 3\sqrt{2}} \cdot \parens{5 + 3\sqrt{2}}
%             = 5^2 - \parens{3\sqrt{2}}^2 = 25 - 18
%             = 7 \in \natural
%     \end{split}
% \end{equation*}
% \qed
% \subsubsection*{Zadanie~5.}
% \begin{gather*}
%     \log_{30}{3} = a\\
%     \log_{30}{5} = b\\
%     \begin{split}
%         3 \cdot (1 - a - b) &= 3 \cdot (\log_{30}{30} - \log_{30}{3} - \log_{30}{5})
%             = 3 \cdot \parens{\log_{30}{\parens{\frac{30}{3}}} - \log_{30}{5}}
%             = 3 \cdot \parens{\log_{30}{10} - \log_{30}{5}}\\
%             &= 3 \cdot \log_{30}{\parens{\frac{10}{5}}}
%             = 3 \cdot \log_{30}2
%             = \log_{30}{\parens{2^3}}
%             = \log_{30}{8}
%     \end{split}
% \end{gather*}
% \qed
% \subsubsection*{Zadanie~6.}
% \begin{equation*}
%     \begin{split}
%         3 + 3^2 + 3^3 + 3^4 + \ldots + 3^{66} &= 3 \cdot \parens{1 + 3 + 3^2} + 3^4 \cdot \parens{1 + 3 + 3^2} + \ldots + 3^{64} \cdot \parens{1 + 3 + 3^2}\\
%             &= \parens{1 + 3 + 3^2} \cdot \parens{3 + 3^4 + 3^7 + 3^{10} + \ldots + 3^{64}}
%             = 13 \cdot \underbrace{\parens{3 + 3^4 + 3^7 + 3^{10} + \ldots + 3^{64}}}_{\text{całkowite}}
%     \end{split}
% \end{equation*}
% \qed
% \subsubsection*{Zadanie~7.}
% Liczba cyfr w~zapisie dziesiętnym liczby \(x\) jest większa lub równa \(\log_{10}{x}\). Pokażmy najpierw, że \(\log_{10}{2} > 0{,}3\):
% \begin{gather*}
%     \frac{3}{10} < \log_{10}{2}\\
%     \qquad \verticaliff\\
%     10^{\frac{3}{10}} < 10^{\log_{10}{2}}\\
%     \sqrt[10]{10^3} < 2
% \end{gather*}
% Ostatnia nierówność jest prawdziwa, ponieważ \(\sqrt[10]{10^3} = \sqrt[10]{1000} < \sqrt[10]{1024} = 2\). Skoro wszystkie przejścia były równoważne, to \(\log_{10}{2} > 0{,}3\). Oszacujmy teraz liczbę cyfr liczby \(2^{2015}\):
% \begin{equation*}
%     \text{liczba cyfr} \geq \log_{10}{2^{2015}} = 2015 \cdot \log_{10}{2} > 2015 \cdot 0{,}3 = 604{,}5
% \end{equation*}
% Liczba cyfr musi być całkowita, więc wynosi przynajmniej \(605\).
% \qed
% \subsubsection*{Zadanie~8.}
% \begin{gather*}
%     2^a = 3 \iff a = \log_{2}{3}\\
%     24^{\frac{2}{3 + a}} = 24^{\frac{2}{\log_{2}{8} + \log_{2}{3}}}
%         = 24^{\frac{2}{\log_{2}{\parens{8 \cdot 3}}}}
%         = 24^{\frac{\log_{2}{4}}{\log_{2}{24}}}
%         = 24^{\log_{24}{4}}
%         = 4
% \end{gather*}
% \subsubsection*{Zadanie~9.}
% \begin{gather*}
%     a = \log_{2}{5} > 0\\
%     2{,}25 < a < 2{,}5\\
%     \frac{9}{4} < \log_{2}{5} < \frac{5}{2}\\
%     \qquad\verticaliff\\
%     2^{\frac{9}{4}} < 2^{\log_{2}{5}} < 2^{\frac{5}{2}}\\
%     \sqrt[4]{2^9} < 5 < \sqrt{2^5}
% \end{gather*}
% Zauważmy, że
% \begin{equation*}
%     \sqrt{2^5} = \sqrt{32} > \sqrt{25} = 5
% \end{equation*}
% oraz
% \begin{equation*}
%     \sqrt[4]{2^9} = \sqrt[4]{512} < \sqrt[4]{625} = 5
% \end{equation*}
% Ostatecznie otrzymujemy
% \begin{equation*}
%     \sqrt[4]{2^9} < \sqrt[4]{625} = 5 = \sqrt{25} < \sqrt{2^5}
% \end{equation*}
% a~ponieważ wszystkie przejścia były równoważne, otrzymujemy tezę:
% \begin{gather*}
%     \frac{9}{4} < \log_{2}{5} < \frac{5}{2}\\
%     2{,}25 < a < 2{,}5\\
% \end{gather*}
% \qed
% \subsubsection*{Zadanie~10.}
% Chcemy pokazać, że
% \begin{gather*}
%     \sqrt{4^{100} - 1} + \sqrt{4^{100} + 1} < 2^{100} + 2^{100}\\
%     \sqrt{2^{200} - 1} + \sqrt{2^{200} - 1} < 2 \cdot 2^{100}
% \end{gather*}
% Z~nierówności między średnią arytmetyczną a~średnią kwadratową otrzymujemy:
% \begin{gather*}
%     \frac{\sqrt{2^{200} - 1} + \sqrt{2^{200} + 1}}{2} \leq \sqrt{\frac{\parens{\sqrt{2^{200} - 1}}^2 + \parens{\sqrt{2^{200} + 1}}^2}{2}}
%         = \sqrt{\frac{2^{200} - 1 + 2^{200} + 1}{2}} = \sqrt{\frac{2 \cdot 2^{200}}{2}} = \sqrt{2^{200}} = 2^{100}\\
%     \frac{\sqrt{2^{200} - 1} + \sqrt{2^{200} + 1}}{2} \leq 2^{100}\\
%     \sqrt{2^{200} - 1} + \sqrt{2^{200} + 1} \leq 2 \cdot 2^{100}\\
%     \sqrt{4^{100} - 1} + \sqrt{4^{100} + 1} \leq 2^{100} + 2^{100}
% \end{gather*}
% Ponieważ w~nierówności między średnią arytmetyczną a~średnią kwadratową równość zachodzi tylko wtedy, gdy wszystkie uśredniane liczby są sobie równe, a~wiadomo, że \(\sqrt{4^{100} - 1} \neq \sqrt{4^{100} + 1}\), to udowodniona powyżej nierówność okazuje się być ostra:
% \begin{equation*}
%     \sqrt{4^{100} - 1} + \sqrt{4^{100} + 1} < 2^{100} + 2^{100}
% \end{equation*}
% \qed
% \subsubsection*{Zadanie~11.}
% \begin{equation*}
%     100! = 1 \cdot 2 \cdot 3 \cdot \ldots \cdot 100
% \end{equation*}
% Silnia ma na końcu tyle zer, ile jest w~niej przemnożonych czynników \({} \cdot 10\), a~każdy czynnik \({} \cdot 10\) bierze się z~czynników \({} \cdot 2 \cdot 5\). Jeśli zatem wśród przemnożonych liczb od \(1\) do \(100\) policzymy czynniki \({} \cdot 2 \cdot 5\), a~bez straty ogólności tylko czynniki \({} \cdot 5\) (ponieważ czynniki \({} \cdot 2\) występują częściej, więc na pewno jest ich więcej), to otrzymamy liczbę zer na końcu liczby \(100!\)\ :
% \begin{equation*}
%     \text{liczba zer na końcu } 100! = \floor{\frac{100}{5}} + \floor{\frac{100}{5^2}} = 20 + 4 = 24
% \end{equation*}
% \subsubsection*{Zadanie~12.}
% \begin{equation*}
%     \begin{split}
%         \frac{1}{1 \cdot 3} &+ \frac{1}{3 \cdot 5} + \frac{1}{5 \cdot 7} + \ldots + \frac{1}{2013 \cdot 2015}\\
%             &= \frac{1}{2} \cdot \parens{\frac{1}{1} - \frac{1}{3}} + \frac{1}{2} \cdot \parens{\frac{1}{3} - \frac{1}{5}} + \frac{1}{2} \cdot \parens{\frac{1}{5} - \frac{1}{7}} + \ldots + \frac{1}{2} \cdot \parens{\frac{1}{2011} - \frac{1}{2013}} + \frac{1}{2} \cdot \parens{\frac{1}{2013} - \frac{1}{2015}}\\
%             &= \frac{1}{2} \cdot \parens{\frac{1}{1} - \frac{1}{3} + \frac{1}{3} - \frac{1}{5} + \frac{1}{5} - \frac{1}{7} + \ldots - \frac{1}{2011} + \frac{1}{2011} - \frac{1}{2013} + \frac{1}{2013} - \frac{1}{2015}}\\
%             &= \frac{1}{2} \cdot \parens{1 - \frac{1}{2015}}
%             = \frac{1}{2} \cdot \parens{\frac{2015}{2015} - \frac{1}{2015}}
%             = \frac{1}{2} \cdot \frac{2014}{2015}
%             = \frac{1007}{2015}
%     \end{split}
% \end{equation*}
\subsubsection*{Zadanie~1.3.}
\begin{itemize}
    \item[d)]
        \begin{equation*}
            \begin{split}
                \limit[x \to 1] \frac{x^4 + x^3 + x^2 - 4}{x - 1} &= \limit[x \to 1] \parens{\frac{x^4 + x^3 + x^2 - 3}{x - 1} - \frac{1}{x - 1}}
                    = \limit[x \to 1] \parens{\frac{\cancel{(x - 1)} \parens{x^3 + 2x^2 + 3x + 3}}{\cancel{x - 1}} - \frac{1}{x - 1}}\\
                    &= \limit[x \to 1] \parens{x^3 + 2x^2 + 3x + 3 - \frac{1}{x - 1}}
            \end{split}
        \end{equation*}
        Granica tego wyrażenia nie może istnieć, ponieważ granice jednostronne są różne:
        \begin{gather*}
            \limit[x \to 1^{-}] \parens{x^3 + 2x^2 + 3x + 3 - \frac{1}{x - 1}} = 1 + 2 + 3 + 3 - \frac{1}{0^{-}} = 9 - (-\infty)\\
            \limit[x \to 1^{+}] \parens{x^3 + 2x^2 + 3x + 3 - \frac{1}{x - 1}} = 1 + 2 + 3 + 3 - \frac{1}{0^{+}} = 9 - (+\infty)\\
        \end{gather*}
    \item[g)]
        \begin{equation*}
            \begin{split}
                \limit[x \to 1] \parens{\frac{3}{1 - x^3} + \frac{1}{x - 1}}
                    = \indeterminate{\frac{0}{0}}
                    &= \limit[x \to 1] \parens{\frac{3}{1 - x^3} - \frac{1}{1 - x}}
                    = \limit[x \to 1] \parens{\frac{3}{1 - x^2} - \frac{1 + x + x^2}{1 - x^3}}\\
                    &= \limit[x \to 1] \frac{-x^2 - x + 2}{1 - x^3} = \limit[x \to 1] \frac{x^2 + x - 2}{x^3 - 1}
                    = \limit[x \to 1] \frac{\cancel{(x - 1)}(x + 2)}{\cancel{(x - 1)} \parens{x^2 + x + 1}}\\
                    &= \limit[x \to 1] \frac{x + 2}{x^2 + x + 1}
                    = \frac{1 + 2}{1 + 1 + 1}
                    = 1
            \end{split}
        \end{equation*}
\end{itemize}
\subsubsection*{Zadanie~1.4.}
\begin{itemize}
    \item[g)]
        \begin{equation*}
            \begin{split}
                \limit[x \to 0] \frac{\sqrt[3]{1 - x} - \sqrt[3]{1 + x}}{x}
                = \indeterminate{\frac{0}{0}}
                &= \limit[x \to 0]{\frac{1 - x - 1 - x}{x \parens{\sqrt[3]{(1 - x)^2} + \sqrt[3]{1 - x^2} + \sqrt[3]{(1 + x)^2}}}}\\
                &= \limit[x \to 0] \frac{-2\cancel{x}}{\cancel{x} \parens{\sqrt[3]{(1 - x)^2} + \sqrt[3]{1 - x^2} + \sqrt[3]{(1 + x)^2}}}\\
                &= \limit[x \to 0] \frac{-2}{\sqrt[3]{(1 - x)^2} + \sqrt[3]{1 - x^2} + \sqrt[3]{(1 + x)^2}}\\
                &= \frac{-2}{\sqrt[3]{(1 - 0)^2} + \sqrt[3]{1 - 0^2} + \sqrt[3]{(1 + 0)^2}}
                = -\frac{2}{3}
            \end{split}
        \end{equation*}
    \item[d)]
        \begin{equation*}
            \begin{split}
                \limit[x \to 4] \frac{\sqrt{1 + 2x} - 3}{\sqrt{x} - 2}
                    = \indeterminate{\frac{0}{0}}
                    &= \limit[x \to 4] \frac{\parens{2x - 8} \parens{\sqrt{x} + 2}}{(x - 4) \parens{\sqrt{1 + 2x} + 3}}
                    = \limit[x \to 4] \frac{2\cancel{(x - 4)}\parens{\sqrt{x} + 2}}{\cancel{(x - 4)} \parens{\sqrt{1 + 2x} + 3}}\\
                    &= \limit[x \to 4] \frac{2\parens{\sqrt{x} + 2}}{\sqrt{1 + 2x} + 3}
                    = \frac{2\parens{\sqrt{4} + 2}}{\sqrt{1 + 2 \cdot 4} + 3}
                    = \frac{8}{6}
                    = \frac{4}{3}
            \end{split}
        \end{equation*}
\end{itemize}
\subsubsection*{Zadanie~1.5.}
\begin{itemize}
    \item[j)]
        \begin{equation*}
            \begin{split}
                \limit[x \to 1] \frac{\sqrt[3]{x} - x^{15}}{x - 1} = \indeterminate{\frac{0}{0}}
                    &= \limit[x \to 1] \frac{x - x^{45}}{(x - 1)\parens{\sqrt[3]{x^2} + \sqrt[3]{x} \cdot x^{15} + x^{30}}}
                    = \limit[x \to 1] \frac{-x \parens{x^{44} - 1}}{(x - 1)\parens{\sqrt[3]{x^2} + \sqrt[3]{x} \cdot x^{15} + x^{30}}}\\
                    &= \limit[x \to 1] \frac{-x\cancel{(x - 1)}\parens{x^{43} + x^{42} + \ldots + x^2 + x + 1}}{\cancel{(x - 1)}\parens{\sqrt[3]{x^2} + \sqrt[3]{x} \cdot x^{15} + x^{30}}}
                    = \limit[x \to 1] \frac{-x\parens{x^{43} + x^{42} + \ldots + x^2 + x + 1}}{\parens{\sqrt[3]{x^2} + \sqrt[3]{x} \cdot x^{15} + x^{30}}}\\
                    &= \frac{-44}{3}
            \end{split}
        \end{equation*}
\end{itemize}
\subsubsection*{Zadanie~1.19.}
\begin{itemize}
    \item[a)]
        \begin{equation*}
            \begin{split}
                \limit[x \to 2] \frac{\sqrt{x} - \sqrt{2} + \sqrt{x - 2}}{\sqrt{x^2 - 4}}
                    = \indeterminate{\frac{0}{0}}
                    &= \parens{\limit[x \to 2] \frac{\sqrt{x} - \sqrt{2}}{\sqrt{x^2 - 4}}} + \parens{\limit[x \to 2] \frac{\sqrt{x - 2}}{\sqrt{x^2 - 4}}}\\
                    &= \parens{\limit[x \to 2] \frac{x - 2}{\sqrt{x^2 - 4} \cdot \parens{\sqrt{x} + \sqrt{2}}}} + \parens{\limit[x \to 2] \frac{\cancel{\sqrt{x - 2}}}{\cancel{\sqrt{x - 2}} \cdot \sqrt{x + 2}}}\\
                    &= \parens{\limit[x \to 2] \frac{x - 2}{\sqrt{x - 2} \cdot \sqrt{x + 2} \cdot \parens{\sqrt{x} + \sqrt{2}}}} + \frac{1}{\sqrt{2 + 2}}\\
                    &= \parens{\limit[x \to 2] \frac{\sqrt{x - 2}}{\sqrt{x + 2} \cdot \parens{\sqrt{x} + \sqrt{2}}}} + \frac{1}{2}\\
                    &= \frac{\sqrt{2 - 2}}{\sqrt{2 + 2} \cdot \parens{\sqrt{2} + \sqrt{2}}} + \frac{1}{2} = 0 + \frac{1}{2} = \frac{1}{2}
            \end{split}
        \end{equation*}
\end{itemize}
