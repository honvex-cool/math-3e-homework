\subsubsection*{Zadanie~1.21.}
\begin{gather*}
    V \coloneqq \text{przegłosowano ustawę}\\
    T \coloneqq \text{Tiger zabrał przynajmniej połowę otrzymanych kopert}
\end{gather*}
Uchwalenie ustawy mogło nastąpić na \(4\) sposoby. Rozważmy, ile możliwości jest w~każdym przypadku:
\begin{itemize}
    \item wszyscy \(3\) za:
        \begin{description}
            \item[możliwości w~\(V\):] \(2^3 = 8\)
            \item[możliwości w~\(V \cap T\):] \(1\)
        \end{description}
    \item \(2\) za, \(1\) przeciw
        \begin{description}
            \item[możliwości w~\(V\):] \(3 \cdot 2^2 = 12\)
            \item[możliwośći w~\(V \cap T\):] \(3\)
        \end{description}
    \item \(2\) za, \(1\) wstrzymuje się od głosu:
        \begin{description}
            \item[możliwości w~\(V\):] \(3 \cdot 2^2 = 12\)
            \item[możliwości w~\(V \cap T\):] \(3 \cdot 3 = 9\)
        \end{description}
    \item \(1\) za, \(2\) wstrzymuje się od głosu:
        \begin{description}
            \item[możliwości w~\(V\):] \(3 \cdot 2 = 6\)
            \item[możliwości w~\(V \cap T\):] \(3 \cdot 2 = 6\)
        \end{description}
\end{itemize}
Możemy teraz obliczyć prawdopodobieństwo warunkowe:
\begin{gather*}
    \card V = 8 + 12 + 12 + 6 = 38\\
    \card\pars{V \cap T} = 1 + 3 + 9 + 6 = 19\\
    P\pars{T / V}
    = \frac{P\pars{T \cap V}}{P\pars{V}}
    = \frac{\frac{\card\pars{V \cap T}}{\cancel{\card\Omega}}}{\frac{\card V}{\cancel{\card\Omega}}}
    = \frac{\card\pars{V \cap T}}{\card V}
    = \frac{19}{38}
    = \frac{1}{2}
\end{gather*}
Zatem Tiger zabrał przynajmniej połowę kopert dla siebie z~prawdopodobieństwem \(\frac{1}{2}\).
\subsubsection*{Zadanie~1.22.}
Wprowadźmy oznaczenia:
\begin{gather*}
    K \coloneqq \text{Książę wybrał staw z~księżniczkami}\\
    F \coloneqq \text{wylosował zwykłą żabę}
\end{gather*}
Wiemy z~treści zadania, że
\begin{gather*}
    P\pars{F / K} = \frac{2}{3}\\
    P\pars{K} = \frac{1}{4}
\end{gather*}
Korzystamy ze wzoru Bayesa:
\begin{equation*}
    P\pars{K / F}
    = \frac{P\pars{F / K} \cdot P\pars{K}}{P\pars{F}}
    = \frac{\frac{2}{3} \cdot \frac{1}{4}}{\frac{2}{3} \cdot \frac{1}{4} + \frac{3}{4}}
    = \frac{\frac{2}{12}}{\frac{11}{12}}
    = \frac{2}{11}
\end{equation*}
Aby otrzymać prawdopodobieństwo, że następna wylosowana żaba będzie księżniczką, mnożymy prawdopodobieństwo, że wybraliśmy dobry staw, przez prawdopodobieństwo, że w~dobrym stawie wylosujemy księżniczkę:
\begin{equation*}
    P\pars{\text{następna to księżniczka}}
    = \frac{1}{3} \cdot \frac{2}{11}
    = \frac{2}{33}
\end{equation*}
\subsubsection*{Zadanie~1.26.}
Obliczamy sumaryczną skuteczność leku:
\begin{gather*}
    P\pars{\text{przeżył} / \text{był leczony}}
    = \frac{95 + 1000}{95 + 5 + 1000 + 9000}
    = \frac{1095}{10100}
    = \frac{219}{2020}
    \approx 0{,}108\\
    P\pars{\text{przeżył} / \text{nie był leczony}}
    = \frac{5000 + 50}{5000 + 5000 + 1000}
    = \frac{5050}{10100}
    \approx 0{,}459
\end{gather*}
Zatem w~ogólności lek nie jest zbyt skuteczny.

