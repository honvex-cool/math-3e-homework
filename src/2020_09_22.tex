\subsection*{Zestaw~II (zadania otwarte)}
\subsubsection*{Zadanie~1.}
Najpierw udowodnijmy, że dla każdej liczby \(n \in \natural\) liczba \(n^3 + 5n\) jest podzielna przez \(2\):
\begin{gather*}
    n \equiv 0 \pmod{2} \implies n^3 + 5n \equiv 0^3 + 5 \cdot 0 \equiv 0 \pmod{2}\\
    n \equiv 1 \pmod{2} \implies n^3 + 5n \equiv 1^3 + 5 \cdot 1 \equiv 6 \equiv 0 \pmod{2}
\end{gather*}
Udowodnimy też, że dla każdej liczby \(n \in \natural\) liczba \(n^3 + 5n\) jest podzielna przez \(3\):
\begin{gather*}
    n \equiv 0 \pmod{3} \implies n^3 + 5n \equiv 0^3 + 5 \cdot 0 \equiv 0 \pmod{3}\\
    n \equiv 1 \pmod{3} \implies n^3 + 5n \equiv 1^3 + 5 \cdot 1 \equiv 6 \equiv 0 \pmod{3}\\
    n \equiv 2 \pmod{3} \implies n^3 + 5n \equiv 2^3 + 5 \cdot 2 \equiv 18 \equiv 0 \pmod{3}
\end{gather*}
Skoro dla każdego \(n \in \natural\) liczba \(n^3 + 5n\) jest podzielna przez liczby pierwsze \(3\) i~\(2\), to jest też podzielna przez \(3 \cdot 2 = 6\).
\qed
\subsubsection*{Zadanie~2.}
\begin{equation*}
    \begin{split}
        M &= \frac{1}{\sqrt{1} + \sqrt{4}} + \frac{1}{\sqrt{4} + \sqrt{7}} + \ldots + \frac{1}{\sqrt{97} + \sqrt{100}}\\
            &= \frac{\sqrt{1} - \sqrt{4}}{\parens{\sqrt{1} + \sqrt{4}} \parens{\sqrt{1} - \sqrt{4}}} + \frac{\sqrt{4} - \sqrt{7}}{\parens{\sqrt{4} + \sqrt{7}} \parens{\sqrt{4} - \sqrt{7}}} + \ldots + \frac{\sqrt{97} - \sqrt{100}}{\parens{\sqrt{97} + \sqrt{100}} \parens{\sqrt{97} - \sqrt{100}}}\\
            &= \frac{\sqrt{1} - \sqrt{4}}{1 - 4} + \frac{\sqrt{4} - \sqrt{7}}{4 - 7} + \ldots + \frac{\sqrt{97} - \sqrt{100}}{97 - 100}\\
            &= \frac{\sqrt{1} - \sqrt{4} + \sqrt{4} - \sqrt{7} + \sqrt{7} - \ldots - \sqrt{97} + \sqrt{97} - \sqrt{100}}{-3}\\
            &= \frac{\sqrt{1} - \sqrt{100}}{-3}
            = \frac{1 - 10}{-3}
            = \frac{-9}{-3}
            = 3 \in \integer
    \end{split}
\end{equation*}
\qed
\subsubsection*{Zadanie~3.}
Chcemy pokazać, że
\begin{equation*}
    \sqrt{\parens{x^2 + 1}^{10} - 1} + \sqrt{\parens{x^2 + 1}^{10} + 1} \leq 2\parens{x^2 + 1}^5
\end{equation*}
Z~nierówności między średnią arytmetyczną a~średnią kwadratową mamy:
\begin{gather*}
    \begin{split}
        \frac{\sqrt{\parens{x^2 + 1}^{10} - 1} + \sqrt{\parens{x^2 + 1}^{10} + 1}}{2} &\leq \sqrt{\frac{\parens{\sqrt{\parens{x^2 + 1}^{10} - 1}}^2 + \parens{\sqrt{\parens{x^2 + 1}^{10} + 1}}^2}{2}}\\
            &= \sqrt{\frac{\parens{x^2 + 1}^{10} - 1 + \parens{x^2 + 1}^{10} + 1}{2}}\\
            &= \sqrt{\frac{\cancel{2}\parens{x^2 + 1}^{10}}{\cancel{2}}}\\
            &= \parens{x^2 + 1}^5
    \end{split}\\
    \frac{\sqrt{\parens{x^2 + 1}^{10} - 1} + \sqrt{\parens{x^2 + 1}^{10} + 1}}{2} \leq \parens{x^2 + 1}^5\\
    \sqrt{\parens{x^2 + 1}^{10} - 1} + \sqrt{\parens{x^2 + 1}^{10} + 1} \leq 2\parens{x^2 + 1}^5
\end{gather*}
\qed
\subsubsection*{Zadanie~4.}
\begin{gather*}
    A = \sqrt[3]{2 + \sqrt{5}} + \sqrt[3]{2 - \sqrt{5}}\\
    \begin{split}
        A^3 &= \parens{\sqrt[3]{2 + \sqrt{5}} + \sqrt[3]{2 - \sqrt{5}}}^3\\
            &= \parens{\sqrt[3]{2 + \sqrt{5}}}^3 + 3 \cdot \parens{\sqrt[3]{2 + \sqrt{5}}}^2 \cdot \parens{\sqrt[3]{2 - \sqrt{5}}} + 3 \cdot \parens{\sqrt[3]{2 + \sqrt{5}}} \cdot \parens{\sqrt[3]{2 - \sqrt{5}}}^2 + \parens{\sqrt[3]{2 - \sqrt{5}}}^3\\
            &= 2 + \sqrt{5} + 3\sqrt[3]{9 + 4\sqrt{5}} \cdot \sqrt[3]{2 - \sqrt{5}} + 3\sqrt[3]{9 - 4\sqrt{5}} \cdot \sqrt[3]{2 + \sqrt{5}} + 2 - \sqrt{5}\\
            &= 2 + \sqrt{5} + 2 - \sqrt{5} + 3\sqrt[3]{\parens{9 + 4\sqrt{5}} \parens{2 - \sqrt{5}}} + 3\sqrt[3]{\parens{9 - 4\sqrt{5}} \parens{2 + \sqrt{5}}}\\
            &= 4 + 3\sqrt[3]{18 + 8\sqrt{5} - 9\sqrt{5} - 20} + 3\sqrt[3]{18 - 8\sqrt{5} + 9\sqrt{5} - 20}\\
            &= 4 + 3\sqrt[3]{-2 - \sqrt{5}} + 3\sqrt[3]{-2 + \sqrt{5}}\\
            &= 4 - 3\sqrt[3]{2 + \sqrt{5}} - 3\sqrt[3]{2 - \sqrt{5}}\\
            &= 4 - 3 \cdot \parens{\sqrt[3]{2 + \sqrt{5}} + \sqrt[3]{2 - \sqrt{5}}}
            = 4 - 3A
    \end{split}\\
    A^3 = 4 - 3A\\
    A^3 + 3A - 4 = 0\\
    (A - 1)\underbrace{\overset{\Delta = 1^2 - 4 \cdot 4 \cdot 1 < 0}{(A^2 + A + 4)}}_{\text{brak pierwiastków}} = 0 \implies A = 1 \in \integer
\end{gather*}
Liczba \(A\) jest równa \(1\), czyli jest całkowita.
\subsubsection*{Zadanie~5.}
\begin{gather*}
    \log_{3}{2} = a\\
    \log_{3}{11} = b\\
    \log_{33}{12}
        = \frac{\log_{3}{12}}{\log_{3}{33}}
        = \frac{\log_{3}{\parens{2 \cdot 2 \cdot 3}}}{\log_{3}{\parens{11 \cdot 3}}}
        = \frac{\log_{3}{2} + \log_{3}{2} + \log_{3}{3}}{\log_{3}{11} + \log_{3}{3}}
        = \frac{a + a + 1}{b + 1}
        = \frac{2a + 1}{b +1}\\
    \log_{33}{12} = \frac{2a + 1}{b + 1}
\end{gather*}
\subsubsection*{Zadanie~6.}
\begin{gather*}
    a \equiv 3 \pmod{7}\\
    a^3 \equiv 3^3 \equiv 27 \equiv 21 + 6 \equiv 7 \cdot 3 + 6 \pmod{7}
\end{gather*}
Reszta z~dzielenia sześcianu liczby \(a\) przez \(7\) wynosi \(6\).
\subsubsection*{Zadanie~7.}
\begin{gather*}
    a, b, c \in \real \setminus \set{0}\\
    a + b + c = 0
\end{gather*}
Wprowadźmy oznaczenia na wyrażenia pewnych typów:
\begin{gather*}
    S_n = a^n + b^n + c^n\\
    \text{w~szczególności } S_0 = a^0 + b^0 + c^0 = 1 + 1 + 1 = 3\\
    \sigma_1 = S_1 = a + b + c = 0\\
    \sigma_2 = ab + bc + ca\\
    \sigma_3 = abc
\end{gather*}
Wykorzystamy następujący wzór (wzór Girarda-Newtona):
\begin{gather*}
    S_{n+3} = S_{n+2}\sigma_1 - S_{n+1}\sigma_2 + S_{n}\sigma_3\\
    S_3 = S_2\sigma_1 - S_1\sigma_2 + S_0\sigma_3 = S_2 \cdot 0 + 0 \cdot \sigma_2 + 3\sigma_3\\
    S_3 = 3\sigma_3\\
    a^3 + b^3 + c^3 = 3abc
\end{gather*}
Ponieważ żadna z~liczb \(a, b, c\) nie jest równa \(0\), to \(abc \neq 0\). Możemy teraz przekształcić wyrażenie podane w~zadaniu i~podstawić wyliczoną wartość:
\begin{gather*}
    \frac{a^2}{bc} + \frac{b^2}{ca} + \frac{c^2}{ab}
        = \frac{a^3}{abc} + \frac{b^2}{abc} + \frac{c^2}{abc}
        = \frac{a^3 + b^3 + c^3}{abc}
        = \frac{3\cancel{abc}}{\cancel{abc}}
        = 3
\end{gather*}
Wartość tej sumy wynosi \(3\).
\subsubsection*{Zadanie~8.}
Aby logarytm z~liczby był zdefiniowany, nie może ona być równa \(0\). Zatem \(a \neq 0 \land b \neq 0\).
\begin{gather*}
    \begin{cases}
        \log_{4}{ab} = 3 \implies \log_{2}{ab} = \frac{\log_{4}{ab}}{\log_{4}{2}} = \frac{3}{\frac{1}{2}} = 6\\
        \log_{2}{a} - \log_{2}{b} = -2 = \log_{2}{\frac{a}{b}}
    \end{cases}\\
    \begin{cases}
        \log_{2}{ab} = 6\\
        \log_{2}{\frac{a}{b}} = -2
    \end{cases}\\
    \begin{cases}
        ab = 2^6\\
        \frac{a}{b} = 2^{-2}
    \end{cases}
\end{gather*}
Ponieważ żadna z~liczb \(a, b\) nie jest równa \(0\), to możemy bezpiecznie przemnożyć równania stronami:
\begin{gather*}
    ab \cdot \frac{a}{b} = 2^6 \cdot 2^{-2}\\
    a^2 = 2^4 = 16\\
    \begin{cases}
        a = 4\\
        b = \frac{64}{a} = 16
    \end{cases}
    \lor
    \begin{cases}
        a = -4\\
        b = \frac{64}{a} = -16
    \end{cases}
\end{gather*}
Zatem \((a, b) \in \set{(4, 16), (-4, -16)}\).
\subsubsection*{Zadanie~9.}
\begin{gather*}
    a + b = 100 \qquad 0 \leq a, b \leq 100\\
    F(a, b) = a^2 + b^2\\
    b = 100 - a\\
    f(a) = a^2 + (100 - a)^2 \qquad D = \closed{0}{100}\\
    f(a) = a^2 + 10000 - 200a + a^2 = 2a^2 - 200a + 10000
\end{gather*}
\begin{enumerate}[label={\alph*)}]
    \item suma kwadratów najmniejsza z~możliwych: współczynnik przy \(a^2\) jest dodatni, więc ramiona paraboli są skierowane w~górę. Oznacza to, że najmniejszą wartość funkcja funkcja \(f\) osiąga dla argumentu \(p = \frac{-b}{2a} = \frac{200}{4} = 50 \in D\) i~wynosi ona \(q = \frac{-\Delta}{4a} = \frac{-(40000 - 80000)}{8} = 5000\), a~\(b = 100 - a\) wynosi wtedy \(50\). Zatem zapis liczby \(100\), o~którym mowa w~treści zadania, to \(100 = 50 + 50\). Suma kwadratów użytych liczb jest minimalna i~wynosi \(5000\).
    \item suma kwadratów największa z~możliwych: współczynnik przy \(a^2\) jest dodatni, więc ramiona paraboli są skierowane w~górę. Oznacza to, że taka funkcja kwadratowa globalnie nie ma wartości największej, ale może istnieć wartość największa w~dziedzinie \(D\). Dziedzina \(D\) jest dwustronnie domknięta, więc istnieją takie wartości na końcach rozważanego przedziału \(\closed{1}{100}\):
        \begin{gather*}
            a = 0 \qquad b = 100 \longrightarrow 100 = 0 + 100 \text{, suma kwadratów użytych liczb wynosi } 10000\\
            a = 100 \qquad b = 0 \longrightarrow 100 = 100 + 0 \text{, suma kwadratów użytych liczb wynosi } 10000
        \end{gather*}
\end{enumerate}
\subsubsection*{Zadanie~10.}
Cyfra jedności liczby jest resztą, do której przystaje ta liczba \({} \bmod 10\). Rozważmy reszty \({} \bmod 10\) kolejnych potęg liczby \(2012\):
\begin{align*}
    2012 &\equiv 2 \pmod{10}\\
    2012^2 &\equiv 4 \pmod{10}\\
    2012^3 &\equiv 8 \pmod{10}\\
    2012^4 &\equiv 16 \equiv 6 \pmod{10}\\
    2012^5 &\equiv 12 \equiv 2 \pmod{10}\\
    2012^6 &\equiv 4 \pmod{10}\\
    &\vdots
\end{align*}
Dalej wszystkie reszty powtarzają się co \(4\) potęgi liczby \(2012\). Jeśli wykładnik jest podzielny przez \(4\), to resztą \({} \bmod 10\), a~zatem ostatnią cyfrą liczby, jest \(6\). Ponieważ \(4 \vert 2012\), to ostatnią cyfrą liczby \(2012^{2012}\) jest \(6\).
\subsubsection*{Zadanie~1.7.}
\begin{itemize}
    \item[f)]
        \begin{equation*}
            \limit[x \to +\infty] \frac{x^2 + 4x - 7}{3x^2 - 2x + 3}
                = \indeterminate{\frac{\infty}{\infty}}
                = \limit[x \to +\infty] \frac{\cancel{x^2} \parens{1 + \converges{0}{\frac{4}{x}} - \converges{0}{\frac{7}{x^2}}}}{\cancel{x^2} \parens{3 - \converges*{0}{\frac{2}{x}} + \converges*{0}{\frac{3}{x^2}}}}
                = \frac{1}{3}
        \end{equation*}
    \item[i)]
        \begin{equation*}
            \limit[x \to -\infty] \frac{1 - \sqrt{x^2 + 1}}{x}
                = \indeterminate{\frac{-\infty}{-\infty}}
                = \limit[x \to -\infty] \frac{\abs{x} \cdot \parens{\converges{0}{\frac{1}{\abs{x}}} - \sqrt{1 + \converges*{0}{\frac{1}{x^2}}}}}{x}
        \end{equation*}
        Zauważmy, że gdy \(x \to -\infty\), to \(\abs{x} = -x\).
        \begin{equation*}
            \limit[x \to -\infty] \frac{\abs{x} \cdot \parens{\converges{0}{\frac{1}{\abs{x}}} - \sqrt{1 + \converges*{0}{\frac{1}{x^2}}}}}{x}
                = \limit[x \to -\infty] \frac{-\cancel{x} \cdot \parens{\converges{0}{\frac{1}{\abs{x}}} - \sqrt{1 + \converges*{0}{\frac{1}{x^2}}}}}{\cancel{x}} = 1
        \end{equation*}
    \item[l)]
        \begin{equation*}
            \limit[x \to +\infty] \frac{(x - 1)^{20} (2x + 15)^{15}}{(3x + 17)^{35}}
                = \indeterminate{\frac{\infty}{\infty}}
                = \limit[x \to +\infty] \frac{\cancel{x^{35}} \parens{2^{15} + \overbrace{\ldots}^{\text{wyrażenia z~\(x\) stopnia mniejszego niż \(35\), więc zbiegają do \(0\)}}}}{\cancel{x^{35}} \parens{3^{35} + \underbrace{\ldots}_{\text{wyrażenia z~\(x\) stopnia mniejszego niż \(35\), więc zbiegają do \(0\)}}}}
                = \frac{2^{15}}{3^{35}}
        \end{equation*}
\end{itemize}
\subsubsection*{Zadanie~1.8.}
\begin{itemize}
    \item[a)]
        \begin{equation*}
            \limit[x \to +\infty] \frac{\sqrt{x + \sqrt{x + \sqrt{x}}}}{\sqrt{x + 1}}
                = \indeterminate{\frac{\infty}{\infty}}
                = \limit[x \to +\infty] \frac{\cancel{\sqrt{x}}\parens{\sqrt{1 + \sqrt{\converges{0}{\frac{1}{x}} + \sqrt{\converges{0}{\frac{1}{x^3}}}}}}}{\cancel{\sqrt{x}} \cdot \sqrt{1 + \converges*{0}{\frac{1}{x}}}}
                = 1
        \end{equation*}
    \item[c)]
        \begin{equation*}
            \begin{split}
                \limit[x \to +\infty] \frac{\sqrt{x^2 + 2} - \sqrt{x^2 + 1}}{\sqrt[4]{x^4 + 8}}
                    = \limit[x \to +\infty] \frac{1}{\sqrt[4]{x^4 + 8} \cdot \parens{\sqrt{x^2 + 2} + \sqrt{x^2 + 1}}}
                    = 0
            \end{split}
        \end{equation*}
    \item[i)]
        \begin{equation*}
            \begin{split}
                \limit[x \to +\infty] \frac{(1 + x)(1 + 2x) \ldots (1 + 10x)}{x^{10} + 1}
                    &= \indeterminate{\frac{\infty}{\infty}}\\
                    &= \limit[x \to +\infty] \frac{\cancel{x^{10}} \parens{10! + \overbrace{\ldots}^{\text{wyrażenia z~\(x\) stopnia mniejszego niż \(10\), więc zbiegają do \(0\)}}}}{\cancel{x^{10}}\parens{1 + \converges*{0}{\frac{1}{x^{10}}}}}
                    = 10!
            \end{split}
        \end{equation*}
\end{itemize}
\subsubsection*{Zadanie~1.9.}
\begin{itemize}
    \item[b)] Zauważmy, że \(\limit[x \to -\infty] \sqrt{x^2 + x + 2} = +\infty\), ponieważ wyrażenie pod pierwiastkiem nie ma miejsc zerowych \(\Delta =~1^2 - 4 \cdot 1 \cdot 2 < 0\), a~współczynnik przy \(x^2\) jest dodatni, więc ramiona paraboli są skierowane w~górę.
        \begin{equation*}
            \limit[x \to -\infty] \parens{x - \sqrt{x^2 + x + 2}}
                = -\infty - \infty
                = -\infty
        \end{equation*}
\end{itemize}
