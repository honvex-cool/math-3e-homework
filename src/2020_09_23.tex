\subsection*{Zestaw~III (zadania otwarte)}
\subsubsection*{Zadanie~1.}
\begin{equation*}
    \frac{a^2}{b} + \frac{b^2}{a} > a + b
\end{equation*}
Skoro liczby \(a\) i~\(b\) są dodatnie, to możemy pomnożyć nierówność stronami przez \(ab\):
\begin{gather*}
    a^3 + b^3 > a^2b + ab^2\\
    a^3 - a^2b + b^3 - ab^2\\
    a^2(a - b) + b^2(b - a) > 0\\
    a^2(a - b) - b^2(a - b) > 0\\
    \parens{a^2 - b^2}(a - b) > 0\\
    (a + b)(a - b)(a - b) > 0\\
    (a + b)(a - b)^2 > 0
\end{gather*}
Skoro \(a \neq b\), to \(a - b \neq 0\), zatem \((a - b)^2 > 0\). Ponadto, suma liczb dodatnich \(a + b\) jest większa od \(0\). Mamy zatem iloczyn dwóch liczb dodatnich, więc jest on większy od \(0\). Wszystkie przejścia były równoważne, zatem teza jest prawdziwa.
\qed
\subsubsection*{Zadanie~2.}
\begin{gather*}
    a^2 + b^2 + 16 \geq ab + 4a + 4b\\
    2a^2 + 2b^2 + 32 \geq 2ab + 8a + 8b\\
    a^2 - 2ab + b^2 + a^2 - 8a + 16 + b^2 - 8b + 16 \geq 0\\
    \parens{a^2 - 2ab + b^2} + \parens{a^2 - 8a + 4^2} + \parens{b^2 - 8b + 4^6} \geq 0\\
    (a - b)^2 + (a - 4)^2 + (b - 4)^2 \geq 0
\end{gather*}
Suma kwadratów jest zawsze nieujemna, więc ostatnia nierówność jest prawdziwa. Wszystkie przejścia były równoważne, więc teza jest prawdziwa.
\qed
\subsubsection*{Zadanie~3.}
\begin{equation*}
    \frac{n^2 + 2}{3} = \frac{n^2 - 1 + 3}{3} = \frac{(n + 1)(n - 1) + 3}{3} = \frac{(n + 1)(n - 1)}{3} + 1
\end{equation*}
Skoro liczba \(n\) nie jest podzielna przez \(3\), to daje przy dzieleniu przez \(3\) resztę \(1\) lub \(2\). Zatem któraś z~liczb \(n + 1, n - 1\) jest podzielna przez \(3\), w~związku z~czym cały ułamek \(\frac{(n + 1)(n - 1)}{3}\) jest liczbą naturalną. Wynika z~tego, że całe wyrażenie jest liczbą naturalną.
\qed
\subsubsection*{Zadanie~4.}
\begin{gather*}
    9x^4 + y^4 + 6 \geq 12xy\\
    9x^4 + y^4 - 12xy + 6 \geq 0\\
    9x^4 - 6x^2y^2 + y^4 + 6x^2y^2 - 12xy + 6 \geq 0\\
    \parens{\parens{3x^2}^2 - 2 \cdot 3 \cdot x^2y^2 + \parens{y^2}^2} + 6\parens{x^2y^2 - 2xy + 1} \geq 0\\
    \parens{9x^2 - y^2}^2 + 6(xy - 1)^2 \geq 0
\end{gather*}
Suma dodatnich wielokrotności kwadratów jest zawsze liczbą nieujemną, więc ostatnia nierówność jest prawdziwa. Wszystkie przejścia były równoważne, więc teza również jest prawdziwa.
\qed
\subsubsection*{Zadanie~5.}
\begin{equation*}
    (n - 1)n(n + 1)(n + 2) + 1 = (n^2 - 1)n(n + 2) + 1 = (n^2 - 1)(n^2 + 2n) + 1 = n^4 + 2n^3 - n^2 - 2n + 1
\end{equation*}
\subsubsection*{Zadanie~6.}
\begin{equation*}
    n^4 + n^2 + 1 = n^4 + 2n^2 + 1 - n^2 = \parens{n^2 + 1}^2 - n^2 = \parens{n^2 + 1 + n} \parens{n^2 + 1 - n}
\end{equation*}
Zauważmy, że gdy \(n > 2\), obydwa powyższe czynniki, na które rozłożyliśmy liczbę \(n^4 + n^2 + 1\), są większe od \(1\). Zatem liczba \(n^4 + n^2 + 1\) jest złożona.
\qed
\subsubsection*{Zadanie~7.}
\begin{gather*}
    abc = 1\\
    \begin{split}
        \frac{1}{1 + a^2b} + \frac{1}{1 + bc^2}
            &= \frac{1 + a^2b}{\parens{1 + a^2b} \parens{1 + ab^2}} + \frac{1 + ab^2}{\parens{1 + a^2b} \parens{1 + ab^2}}
            = \frac{2 + a^2b + ab^2}{1 + a^2b^2c^2 + a^2b + ab^2}\\
            &= \frac{2 + a^2b + ab^2}{1 + (abc)^2 + a^2b + ab^2}
            = \frac{2 + a^2b + ab^2}{1 + 1^2 + a^2b + ab^2}
            = \frac{2 + a^2b + ab^2}{2 + a^2b + ab^2}
            = 1
    \end{split}
\end{gather*}
\qed
\subsubsection*{Zadanie~8.}
\begin{equation*}
    \frac{n^3 + 2n^2 - 3n - 6}{n^2 + 3n + 2}
        = \frac{\cancel{(n + 2)}(n^2 - 3)}{(n + 1)\cancel{(n + 2)}}
        = \frac{n^2 - 1 - 2}{n + 1}
        = \frac{\cancel{(n + 1)}(n - 1)}{\cancel{n + 1}} - \frac{2}{n + 1}
        = \underbrace{n - 1}_{\text{całkowite}} - \frac{2}{n + 1}
\end{equation*}
W~tej postaci wyrażenia widać, że \(n + 1\) musi dzielić liczbę \(2\). Istnieją cztery potencjalne możliwości:
\begin{itemize}
    \item \(n + 1 = -1 \implies n = -2\)
    \item \(n + 1 = -2 \implies n = -3\)
    \item \(n + 1 = 1 \implies n = 0\)
    \item \(n + 1 = 2 \implies n = 1\)
\end{itemize}
Jednak \(n \neq -2\), ponieważ inaczej zerowałby się mianownik. Zatem ostatecznie \(n \in \set{-3, 0, 1}\).
\subsubsection*{Zadanie~9.}
Oznaczenia:
\begin{gather*}
    v = \text{ prędkość w \(\frac{\km}{\h}\)} \neq 0\\
    t = \text{ czas przejazdu trasy w \(\h\)}\\
    s = \text{ długość przebytej trasy } = 300\km\\
    15\text{ minut} = \frac{1}{4}\text{ godziny}
\end{gather*}
Przy stałej prędkości drogę, prędkość i~czas łączy zależność:
\begin{equation*}
    v = \frac{s}{t} \implies s = v \cdot t
\end{equation*}
Na podstawie danych z~zadania możemy zapisać układ równań:
\begin{gather*}
    \begin{cases}
        s = v \cdot t \implies t = \frac{s}{v}\\
        v \cdot t = (v + 5) \cdot (t - \frac{1}{4})
    \end{cases}
\end{gather*}
Przekształcamy drugie równanie:
\begin{gather*}
    v \cdot t = v \cdot t - \frac{1}{4}v + 5t - \frac{5}{4}\\
    \frac{1}{4}v - 5t + \frac{5}{4} = 0
\end{gather*}
Teraz za \(t\) możemy podstawić \(\frac{s}{v}\):
\begin{gather*}
    \frac{1}{4}v - 5\frac{s}{v} + \frac{5}{4} = 0\\
    \frac{1}{4}v^2 + \frac{5}{4}v - 5s = 0\\
    \Delta = \frac{25}{16} + 5s = 1500 + \frac{25}{16}\\
    \sqrt{\Delta} = 38 + \frac{3}{4}\\
    v = \frac{-\frac{5}{4} + 38 + \frac{3}{4}}{2 \cdot \frac{1}{4}} = 75 \lor v = \frac{-\frac{5}{4} - 38 - \frac{3}{4}}{2 \cdot \frac{1}{4}} < 0 \text{ (niemożliwe)}\\
    v = 75\frac{\km}{\h}
\end{gather*}
Kierowca jechał z~prędkością \(75\frac{\km}{\h}\).
\subsubsection*{Zadanie~10.}
\begin{gather*}
    \frac{a^2 - 6b^2}{ab} = -1
\end{gather*}
\subsubsection*{Zadanie~11.}
Skoro \(a \neq 0\) i~\(b \neq 0\), to \(ab \neq 0\). Warunek dany w~zadaniu możemy przekształcić:
\begin{gather*}
    \frac{a}{b} + a = \frac{b}{a} + b\\
    b - a = \frac{a}{b} - \frac{b}{a}
\end{gather*}
Ułamki po prawej stronie możemy sprowadzić do wspólnego mianownika i~odjąć:
\begin{gather*}
    b - a = \frac{a^2}{ab} - \frac{b^2}{ab}\\
    b - a = \frac{a^2 - b^2}{ab}\\
    -(a - b) = \frac{(a + b)(a - b)}{ab}
\end{gather*}
Ponieważ \(a \neq b\), to \(a - b \neq 0\). Zatem obie strony równania wolno nam podzielić przez \((a - b)\).
\begin{gather*}
    -\cancel{(a - b)} = \frac{(a + b)\cancel{(a - b)}}{ab}\\
    \frac{a + b}{ab} = -1
\end{gather*}
W~zadaniu mamy wyznaczyć \(\frac{1}{a} + \frac{1}{b}\). Teraz możemy to zrobić łatwo:
\begin{equation*}
    \frac{1}{a} + \frac{1}{b} = \frac{a}{ab} + \frac{b}{ab} = \frac{a + b}{ab} = -1
\end{equation*}
Zatem \(\frac{1}{a} + \frac{1}{b}\) wynosi \(-1\).
\subsubsection*{Zadanie~12.}
\begin{equation*}
    \begin{split}
        \frac{a}{(a - b)(a - c)} &+ \frac{b}{(b - c)(b - a)} + \frac{c}{(c - a)(c - b)}
            = -\frac{a}{(a - b)(c - a)} - \frac{b}{(a - b)(b - c)} - \frac{c}{(b - c)(c - a)}\\
            &= -\frac{a(b - c)}{(a - b)(b - c)(c - a)} - \frac{b(c - a)}{(a - b)(b - c)(c - a)} - \frac{c(a - b)}{(a - b)(b - c)(c - a)}\\
            &= -\frac{a(b - c) + b(c - a) + c(a - b)}{(a - b)(b - c)(c - a)}
            = -\frac{ab - ca + bc - ab + ca - bc}{(a - b)(b - c)(c - a)}\\
            &= -\frac{0}{(a - b)(b - c)(c - a)}
            = 0
    \end{split}
\end{equation*}
\subsubsection*{Zadanie~1.10.}
\begin{enumerate}[label={\alph*)}]
    \item
        \begin{equation*}
            \limit[x \to 0] (\sin{x} + 2\cos{x}) = \sin{0} + 2\cos{0} = 0 + 2 \cdot 1 = 2
        \end{equation*}
    \item
        \begin{equation*}
            \limit[x \to \pi] \frac{\sin{x}}{x} = \frac{\sin{\pi}}{\pi} = \frac{0}{\pi} = 0
        \end{equation*}
    \item
        \begin{equation*}
            \limit[x \to 0] \frac{\sin{5x}}{x}
                = \indeterminate{\frac{0}{0}}
                = \limit[x \to 0] \frac{\sin{(x + 4x)}}{x}
                = \limit[x \to 0] \frac{\sin{x}\cos{4x} + \sin{4x}\cos{x}}{x}
        \end{equation*}
        Możemy obliczyć dwie pomocnicze granice:
        \begin{gather*}
            \limit[x \to 0] \frac{\sin{x}\cos{4x}}{x}
                = \indeterminate{\frac{0}{0}}
                = \limit[x \to 0] \frac{\sin{x}}{x} \cdot \cos{4x}
                = \parens{\limit[x \to 0] \frac{\sin{x}}{x}} \cdot \parens{\limit[x \to 0] \cos{4x}}
                = 1 \cdot 1
                = 1\\
            \begin{split}
                \limit[x \to 0] \frac{\sin{4x}\cos{x}}{x}
                    &= \indeterminate{\frac{0}{0}}
                    = \limit[x \to 0] \frac{2\sin{2x}\cos{2x}\cos{x}}{x}
                    = \limit[x \to 0] \frac{2 \cdot 2\sin{x}\cos{x}\cos{2x}\cos{x}}{x}\\
                    &= \parens{\limit[x \to 0] \frac{4\sin{x}}{x}} \cdot \parens{\limit[x \to 0] \cos{2x}\cos^2{x}}
                    = 4 \cdot 1
                    = 4
            \end{split}
        \end{gather*}
        Teraz na mocy praw działań na granicach mamy:
        \begin{equation*}
            \limit[x \to 0] \frac{\sin{5x}}{x}
                = \limit[x \to 0] \frac{\sin{x}\cos{4x} + \sin{4x}\cos{x}}{x}
                = \parens{\limit[x \to 0] \frac{\sin{x}\cos{4x}}{x}} + \parens{\limit[x \to 0] \frac{\sin{4x}\cos{x}}{x}}
                = 1 + 4
                = 5
        \end{equation*}
    \item
        \begin{equation*}
            \begin{split}
                \limit[x \to 0] \frac{\sin{6x}}{\sin{2x}}
                    &= \indeterminate{\frac{0}{0}}
                    = \limit[x \to 0] \frac{\sin{(4x + 2x)}}{\sin{2x}}
                    = \limit[x \to 0] \frac{\sin{4x}\cos{2x} + \sin{2x}\cos{4x}}{\sin{2x}}
                    = \limit[x \to 0] \frac{2\cancel{\sin{2x}}\cos{2x}\cos{2x} + \cancel{\sin{2x}}\cos{4x}}{\cancel{\sin{2x}}}\\
                    &= \limit[x \to 0] 2\cos^2{2x} + \cos{4x}
                    = 2\cos^2{0} + \cos{0}
                    = 2 + 1
                    = 3
            \end{split}
        \end{equation*}
    \item
        \begin{equation*}
            \begin{split}
                \limit[x \to 0] \frac{\sin^2{2x}}{\sin^2{3x}}
                    &= \indeterminate{\frac{0}{0}}
                    = \limit[x \to 0] \frac{4\sin^2{x}\cos^2{x}}{\sin^2{(2x + x)}}
                    = \limit[x \to 0] \frac{4\sin^2{x}\cos^2{x}}{(\sin{2x}\cos{x} + \sin{x}\cos{2x})^2}
                    = \limit[x \to 0] \frac{4\sin^2{x}\cos^2{x}}{(2\sin{x}\cos{x}\cos{x} + \sin{x}\cos{2x})^2}\\
                    &= \limit[x \to 0] \frac{4\cancel{\sin^2{x}}\cos^2{x}}{4\cancel{\sin^2{x}}\cos^4{x} + 4\cancel{\sin^2{x}}\cos^2{x}\cos{2x} + \cancel{\sin^2{x}}\cos^2{2x}}
                    = \frac{4\cos^2{0}}{4\cos^4{0} + 4\cos^2{0}\cos{(2 \cdot 0)} + \cos^2{(2 \cdot 0)}}\\
                    &= \frac{4}{4 + 4 + 1}
                    = \frac{2}{9}
            \end{split}
        \end{equation*}
    \item
    \item
        \begin{equation*}
            \limit[x \to 0] \frac{\sqrt{1 - \cos^2{x}}}{1 - \cos{x}}
                = \indeterminate{\frac{0}{0}}
                = \limit[x \to 0] \frac{1 - \cos^2{x}}{(1 - \cos{x}) \cdot \sqrt{1 - \cos^2{x}}}
                = \limit[x \to 0] \frac{\cancel{(1 - \cos{x})}\bconverges{2}{(1 + \cos{x})}}{\cancel{(1 - \cos{x})} \cdot \bconverges*{0^{+}}{\sqrt{1 - \converges*{1}{\cos^2{x}}}}}
                = +\infty
        \end{equation*}
    \item
        \begin{equation*}
            \limit[x \to 0] \frac{\sin{x}}{\sin{6x} - \sin{7x}}
                = \indeterminate{\frac{0}{0}}
                = \limit[x \to 0] \frac{2\sin{\frac{x}{2}}\cos{\frac{x}{2}}}{2\sin{\frac{6x - 7x}{2}}\cos{\frac{6x + 7x}{2}}}
                = \limit[x \to 0] \frac{2\sin{\frac{x}{2}}\cos{\frac{x}{2}}}{2\sin{\frac{-x}{2}}\cos{\frac{13x}{2}}}
                = \limit[x \to 0] \frac{\cancel{2}\cancel{\sin{\frac{x}{2}}}\cos{\frac{x}{2}}}{-\cancel{2}\cancel{\sin{\frac{x}{2}}}\cos{\frac{13x}{2}}}
                = -\frac{\cos{0}}{\cos{0}} = -1
        \end{equation*}
    \item
        \begin{equation*}
            \begin{split}
                \limit[x \to 0] \frac{1 - \cos{4x}}{\sin{3x}}
                    &= \indeterminate{\frac{0}{0}}
                    = \limit[x \to 0] \frac{\sin^2{2x} + \cos^2{2x} - \parens{\cos^2{2x} - \sin^2{2x}}}{\sin{(2x + x)}}
                    = \limit[x \to 0] \frac{2\sin^2{2x}}{\sin{2x}\cos{x} + \sin{x}\cos{2x}}\\
                    &= \limit[x \to 0] \frac{\cancel{2} \cdot 2\sin^{\cancel{2}}{x}\cos^{\bcancel{2}}{x}}{\cancel{2}\cancel{\sin{x}}\cos{x}\bcancel{\cos{x}} + \cancel{\sin{x}}\bcancel{\cos{x}}}
                    = \frac{2\sin{0}\cos{0}}{\cos{0} + 1}
                    = \frac{0}{2}
                    = 0
            \end{split}
        \end{equation*}
    \item
    \item
        \begin{equation*}
            \limit[x \to \frac{\pi}{2}] \frac{\cos{x}}{x - \frac{\pi}{2}}
                = \indeterminate{\frac{0}{0}}
                = \limit[x \to \frac{\pi}{2}] -\frac{\sin{\converges{0}{\parens{\frac{\pi}{2} - x}}}}{\converges*{0}{\frac{\pi}{2} - x}}
                = -1
        \end{equation*}
\end{enumerate}
