% \subsection*{Zestaw~VII (zadania otwarte)}
% \subsubsection*{Zadanie~1.}
% \begin{equation*}
%     y = 4x^2 - 6x - 9
% \end{equation*}
% Wiemy, że pierwiastkami są liczby \(x_1, x_2\). Możemy zatem skorzystać z~wzorów Viete'a:
% \begin{gather*}
%     x_1 + x_2 = \frac{-\pars{-6}}{4} = \frac{3}{2}\\
%     x_1x_2 = \frac{-9}{4} = -\frac{9}{4}
% \end{gather*}
% Możemy teraz obliczyć podaną liczbę, aby pokazać, że jest całkowita:
% \begin{equation*}
%     \frac{x_1}{x_2} + \frac{x_2}{x_1}
%         = \frac{x_1^2 + x_2^2}{x_1x_2}
%         = \frac{\pars{x_1 + x_2}^2 - 2x_1x_2}{x_1x_2}
%         = \frac{\pars{\frac{3}{2}}^2 - 2 \cdot \pars{-\frac{9}{4}}}{-\frac{9}{4}}
%         = \frac{\frac{9}{4} + 2 \cdot \frac{9}{4}}{-\frac{9}{4}}
%         = \frac{3 \cdot \cancel{\frac{9}{4}}}{-\cancel{\frac{9}{4}}}
%         = -3 \in \integer
% \end{equation*}
% \qed
% \subsubsection*{Zadanie~2.}
% \begin{gather*}
%     f\pars{x} = \pars{x - a}\pars{x - b}\pars{x - c}\\
%     a < b < c
% \end{gather*}
% Zauważmy kilka faktów:
% \begin{gather*}
%     \frac{a + b}{2} > \frac{a + a}{2} = \frac{2a}{2} = a\\
%     \frac{a + b}{2} - a > a - a > 0\\
%     \frac{a + b}{2} < \frac{b + b}{2} = \frac{2b}{2} = b\\
%     \frac{a + b}{2} - b < b - b < 0\\
%     \frac{a + b}{2} - c < b - c < 0
% \end{gather*}
% Teraz możemy użyć uzyskanych rezultatów do oszacowania wartości funkcji:
% \begin{equation*}
%     f\pars{\frac{a + b}{2}} = \underset{> 0}{\pars{\frac{a + b}{2} - a}}\underset{< 0}{\pars{\frac{a + b}{2} - b}}\underset{< 0}{\pars{\frac{a + b}{2} - c}} > 0
% \end{equation*}
% \qed
% \subsubsection*{Zadanie~4.}
\subsubsection*{Zadanie~29.09.1.}
\begin{enumerate}[label={\alph*)}]
    \item
        \begin{equation*}
            \limit[x \to 2^-] \frac{\abs{x - 2}}{2 - x}
                = \limit[x \to 2^-] \frac{2 - x}{2 - x}
                = 1
        \end{equation*}
    \item
        \begin{equation*}
            \limit[x \to 2^+] \frac{\abs{x - 2}}{2 - x}
                = \limit[x \to 2^+] \frac{x - 2}{2 - x}
                = -1
        \end{equation*}
    \item
        \begin{equation*}
            \limit[x \to -2^-] \frac{\abs{x^2 - 4}}{x + 2}
                = \limit[x \to -2^-] \frac{\abs{\overset{< 0}{\pars{x - 2}}\overset{< 0}{\pars{x + 2}}}}{x + 2}
                = \limit[x \to -2^-] \frac{\pars{x - 2}\cancel{\pars{x + 2}}}{\cancel{x + 2}}
                = \limit[x \to -2^-] \pars{x - 2}
                = -4
        \end{equation*}
    \item
        \begin{equation*}
            \limit[x \to 0^-] \pars{\sgn x \cdot x}
                = \limit[x \to 0^-] -x
                = 0
        \end{equation*}
    \item
        \begin{equation*}
            \limit[x \to 1^+] \floor{\overset{1 < x < 2}{x} + 2}
                = 1 + 2
                = 3
        \end{equation*}
    \item
        \begin{equation*}
            \limit[x \to 1^-] \floor{\overset{0 < x < 1}{x} + 2}
                = 0 + 2
                = 2
        \end{equation*}
    \item
        \begin{equation*}
            \limit[x \to 1^-] \frac{3x^2 + 2x - 5}{\abs{2x^2 - 5x + 3}}
                = \limit[x \to 1^-] \frac{\pars{x - 1}\pars{3x + 5}}{\abs{\underset{< 0}{\pars{x - 1}}\underset{< 0}{\pars{2x - 3}}}}
                = \limit[x \to 1^-] \frac{\cancel{\pars{x - 1}}\pars{3x + 5}}{\cancel{\pars{x - 1}}\pars{2x - 3}}
                = \limit[x \to 1^-] \frac{3x + 5}{2x - 3}
                = \frac{8}{-1}
                = -8
        \end{equation*}
\end{enumerate}
\subsubsection*{Zadanie~29.09.2.}
\begin{enumerate}[label={\alph*)}]
    \item
        \begin{equation*}
            \limit[x \to 0^-] \frac{x^2 - 1}{x}
                = \frac{\pars{0^-}^2 - 1}{0^-}
                = \frac{-1}{0^-}
                = +\infty
        \end{equation*}
    \item
        \begin{equation*}
            \limit[x \to 0^+] \frac{x^2 - 1}{x}
                = \frac{\pars{0^+}^2 - 1}{0^+}
                = \frac{-1}{0^+}
                = -\infty
        \end{equation*}
    \item
        \begin{equation*}
            \limit[x \to 3^-] \frac{5x + 1}{3 - x}
                = \frac{5 \cdot 3 + 1}{0^+}
                = +\infty
        \end{equation*}
    \addtocounter{enumi}{1}
    \item
        \begin{equation*}
            \limit[x \to -2^+] \frac{x^2 - 3x - 10}{x^2 + 4x + 4}
                = \limit[x \to -2^+] \frac{\cancel{\pars{x + 2}}\pars{x - 5}}{\pars{x + 2}^{\cancel{2}}}
                = \limit[x \to -2^+] \frac{x - 5}{x + 2}
                = \frac{-7}{0^+}
                = -\infty
        \end{equation*}
    \item
        \begin{equation*}
            \limit[x \to -1^-] \frac{3 - 2x}{3x^2 - 4x - 7}
                = \limit[x \to -1^-]\frac{3 - 2x}{\pars{x + 1}\pars{3x - 1}}
                = \frac{3 - 2 \cdot \pars{-1}}{0^- \cdot -4}
                = +\infty
        \end{equation*}
    \item
        \begin{equation*}
            \limit[x \to -2^+] \frac{x^2 - 3x}{x^3 + 7x^2 + 16x + 12}
                = \limit[x \to -2^+] \frac{x^2 - 3x}{\pars{x + 2}\pars{x^2 + 5x + 6}}
                = \limit[x \to -2^+] \frac{x\pars{x - 3}}{\pars{x + 2}^2\pars{x + 3}}
                = \frac{-2 \cdot \pars{-5}}{\pars{0^+}^2 \cdot 1}
                = +\infty
        \end{equation*}
    \item
        \begin{equation*}
            \limit[x \to 4^-] \frac{x^2 - 2x - 8}{x^3 - 7x + 8x + 16}
                = \limit[x \to 4^-] \frac{\pars{x - 4}\pars{x + 2}}{\pars{x - 4}\pars{x^2 - 3x - 4}}
                = \limit[x \to 4^-] \frac{\cancel{\pars{x - 4}}\pars{x + 2}}{\pars{x - 4}^{\cancel{2}}\pars{x + 1}}
                = \frac{6}{0^- \cdot 5}
                = -\infty
        \end{equation*}
\end{enumerate}
