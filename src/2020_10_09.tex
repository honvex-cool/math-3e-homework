\subsubsection*{Zadanie~2.16.}
\begin{gather*}
    f\pars{x} = \begin{cases}
        \frac{x^2 - 4}{x - 2} & \iff x \neq 2\\
        a \iff x = 2
    \end{cases}\\
    D_f = \real\\
    x_0 = 2
\end{gather*}
Ponieważ funkcja jest określona na całym przedziale liczb rzeczywistych, to jest ona ciągła w~punkcie \(x_0\), gdy istnieje w~nim granica funkcji i~jest ona równa wartości funkcji w~tym punkcie. Sprawdźmy to zatem:
\begin{equation*}
    \limit[x \to 2^\pm] f\pars{x}
        = \limit[x \to 2^\pm] \frac{x^2 - 4}{x - 2}
        = \indeterminate{\frac{0}{0}}
        = \limit[x \to 2^\pm] \frac{\cancel{\pars{x - 2}}\pars{x + 2}}{\cancel{x - 2}}
        = \limit[x \to 2^\pm] \pars{x + 2}
        = 4
\end{equation*}
Istnieją obustronne granice i~są one sobie równe, zatem granica w~punkcie \(x_0 = 2\) istnieje. Zatem wystarczy, aby \(f\pars{x_0} = f\pars{2} = a\) było równe \(\limit[x \to 2] f\pars{x}\). Zatem jedyną wartością parametru jest \(a = 4\).
\subsubsection*{Zadanie~3.2.}
\begin{equation*}
    f\colon \real \mapsto \real \text{ jest ciągła}
\end{equation*}
Funkcja jest ograniczona, czyli istnieją takie \(v_\ell, v_h \in \real\), że
\begin{equation*}
    \forall x \in \real\colon v_\ell \leq f\pars{x} \leq v_h
\end{equation*}
Oznacza to, w~szczególności, że
\begin{gather*}
    \exists x_\ell \in \real\colon f\pars{x_\ell} > x_\ell\\
    \wland\\
    \exists x_h \in \real\colon f\pars{x_h} < x_h
\end{gather*}
Zdefiniujmy funkcję
\begin{equation*}
    g\pars{x} = x - f\pars{x}
\end{equation*}
Wiemy, że funkcja \(x\) jest ciągła oraz z~założenia mamy ciągłość funkcji \(f\pars{x}\). Zatem funkcja \(g\pars{x}\) jako różnica funkcji ciągłych sama jest ciągła. Zauważmy, że
\begin{gather*}
    g\pars{x_\ell} = x_\ell - f\pars{x_\ell} < 0\\
    g\pars{x_h} = x_h - f\pars{x_h} > 0
\end{gather*}
Skoro funkcja ciągła \(h\pars{x} = x - f\pars{x}\) przyjmuje w~swojej dziedzinie wartości różnych znaków, to na mocy własności Darboux musi mieć miejsce zerowe.
\qed
\subsubsection*{Zadanie~3.3.}
Dziedziną podanego w~zadaniu wyrażenia jest \(\real\).
\begin{equation*}
    f\pars{x} = x^3 + 3x - 2
        = x^3 - 3x^2 + 3x - 1 - 1 + 3x^2
        = \pars{x - 1}^3 - 1 + 3x^2
\end{equation*}
Wykres funkcji \(\pars{x - 1}^3 - 1\) jest translacją wykresu funkcji rosnącej \(x^3\) o~wektor \(\brackets{1; -1}\), co nie zmienia kształtu wykresu, czyli nadal jest on rosnący. Do funkcji tej dodajemy zawsze nieujemne wyrażenie \(3x^2\), więc nie zaburza ono rośnięcia funkcji. Zatem wyrażenie dane w~zadaniu jest funkcją rosnącą w~\(\real\). Jest również funkcją ciągłą, ponieważ jest sumą funkcji ciągłych \(x^3, 3x, -2\). Zauważmy, że
\begin{gather*}
    f\pars{0} = -2 < 0\\
    f\pars{1} = 3 + 1 - 2 = 2 > 0
\end{gather*}
Skoro funkcja ciągła przyjmuje na końcach przedziału \(\closed{0}{1}\) wartości różnych znaków, to na mocy twierdzenia Darboux ma w~tym przedziale pierwiastek. Ponieważ jest rosnąca, to jest to jej jedyny pierwiastek.
\qed
\subsubsection*{Wyznaczenie pierwiastka równania z~zadania~3.3. z~dokładnością do \(\frac{1}{1000}\)}
Pierwiastek będziemy oznaczać przez \(x_0\), a~jego aproksymację przez \(\xi\). Widzimy, że \(x_0 \neq 0 \land x_0 \neq 1\). Wyznaczymy \(\xi\) metodą bisekcji:
\begin{proofcases}
    \item pierwiastek jest w~przedziale \(\open{0}{1}\), więc \(\xi = 0\) z~dokładnością do \(1 > \abs{\xi - x_0}\)
    \item dzielimy przedział na pół:
        \begin{equation*}
            f\pars{\frac{0 + 1}{2}}
                = f\pars{\frac{1}{2}}
                = \pars{\frac{1}{2}}^3 + 3 \cdot \frac{1}{2} - 2
                = \frac{1}{8} + \frac{3}{2} - 2 = -\frac{3}{8}
                < 0
        \end{equation*}
        \(\xi = \frac{1}{2}\) z~dokładnością do \(\frac{1}{2} > \abs{\xi - x_0}\)
    \item dzielimy przedział na pół:
        \begin{equation*}
            f\pars{\frac{\frac{1}{2} + 1}{2}}
                = f\pars{\frac{3}{4}}
                = \pars{\frac{3}{4}}^3 + 3 \cdot \frac{3}{4} - 2
                = \frac{27}{64} + \frac{144}{64} - 2
                = \frac{43}{64}
                > 0
        \end{equation*}
        \(\xi = \frac{3}{4}\) z~dokładnością do \(\frac{1}{4} > \abs{\xi - x_0}\)
    \item dzielimy przedział na pół:
        \begin{equation*}
            f\pars{\frac{\frac{1}{2} + \frac{3}{4}}{2}}
                = f\pars{\frac{5}{8}}
                = \pars{\frac{5}{8}}^3 + 3 \cdot \frac{5}{8} - 2
                = \frac{125}{512} + \frac{960}{512} - 2
                = \frac{1085}{512} - 2
                = \frac{61}{512}
                > 0
        \end{equation*}
        \(\xi = \frac{5}{8}\) z~dokładnością do \(\frac{1}{8} > \abs{\xi - x_0}\)
    \item dzielimy przedział na pół:
        \begin{equation*}
            f\pars{\frac{\frac{1}{2} + \frac{5}{8}}{2}}
                = f\pars{\frac{9}{16}}
                = \pars{\frac{9}{16}}^3 + 3 \cdot \frac{9}{16} - 2
                = \frac{729}{4096} + \frac{6912}{4096} - 2
                = \frac{7641}{4096} - 2
                = -\frac{551}{4096}
                < 0
        \end{equation*}
        \(\xi = \frac{9}{16}\) z~dokładnością do \(\frac{1}{16} > \abs{\xi - x_0}\)
    \item dzielimy przedział na pół:
        \begin{equation}
            f\pars{\frac{\frac{9}{16} + \frac{5}{8}}{2}}
                = f\pars{\frac{19}{32}}
                = \pars{\frac{19}{32}}^3 + 3 \cdot \frac{19}{32} - 2
                = \frac{6859}{32768} + \frac{58368}{32768} - 2
                = \frac{65227}{32768} - 2
                = -\frac{309}{32768}
                < 0
        \end{equation}
        \(\xi = \frac{19}{32}\) z~dokładnością do \(\frac{1}{32} > \abs{\xi - x_0}\)
    \item dzielimy przedział na pół:
        \begin{equation*}
            f\pars{\frac{\frac{19}{32} + \frac{5}{8}}{2}}
                = f\pars{\frac{39}{64}}
                = \pars{\frac{39}{64}}^3 + 3 \cdot \frac{39}{64} - 2
                = \frac{59319}{262144} + \frac{479232}{262144} - 2
                = \frac{538551}{262144} - 2
                = \frac{14263}{262144}
                > 0
        \end{equation*}
        \(\xi = \frac{39}{64}\) z~dokładnością do \(\frac{1}{64} > \abs{\xi - x_0}\)
    \item dzielimy przedział na pół:
        \begin{equation*}
            f\pars{\frac{\frac{19}{32} + \frac{39}{64}}{2}}
                = f\pars{\frac{77}{128}}
                = \pars{\frac{77}{128}}^3 + 3 \cdot \frac{77}{128} - 2
                = \frac{456533}{2097152} + \frac{3784704}{2097152} - 2
                = \frac{4241237}{2097152} - 2
                = \frac{46933}{2097152}
                > 0
        \end{equation*}
        \(\xi = \frac{77}{128}\) z~dokładnością do \(\frac{1}{128} > \abs{\xi - x_0}\)
    \item dzielimy przedział na pół:
        \begin{equation*}
            f\pars{\frac{\frac{19}{32} + \frac{77}{128}}{2}}
                = f\pars{\frac{153}{256}}
                = \pars{\frac{153}{256}}^3 + 3 \cdot \frac{153}{256} - 2
                = \frac{3581577}{16777216} + \frac{30081024}{16777216} - 2
                = \frac{33662601}{16777216} - 2
                = \frac{108169}{16777216}
                > 0
        \end{equation*}
        \(\xi = \frac{153}{256}\) z~dokładnością do \(\frac{1}{256} > \abs{\xi - x_0}\)
    \item dzielimy przedział na pół:
        \begin{equation*}
            f\pars{\frac{\frac{19}{32} + \frac{153}{256}}{2}}
                = f\pars{\frac{305}{512}}
                = \pars{\frac{305}{512}}^3 + 3 \cdot \frac{305}{512} - 2
                = \frac{28372625}{134217728} + \frac{239861760}{134217728} - 2
                = \frac{268234385}{134217728} - 2
                = -\frac{201071}{134217728}
                < 0
        \end{equation*}
        \(\xi = \frac{305}{512}\) z~dokładnością do \(\frac{1}{512} > \abs{\xi - x_0}\)
    \item dzielimy przedział na pół:
        \begin{equation*}
            \begin{split}
                f\pars{\frac{\frac{305}{512} + \frac{153}{256}}{2}}
                    &= f\pars{\frac{611}{1024}}
                    = \pars{\frac{611}{1024}}^3 + 3 \cdot \frac{611}{1024} - 2
                    = \frac{228099131}{1073741824} + \frac{1922039808}{1073741824} - 2\\
                    &= \frac{2150138939}{1073741824} - 2
                    = \frac{2655291}{1073741824}
                    > 0
            \end{split}
        \end{equation*}
        \(\xi = \frac{611}{1024}\) z~dokładnością do \(\frac{1}{1000} > \frac{1}{1024} > \abs{\xi - x_0}\)
\end{proofcases}
\subsubsection*{Zadanie~3.6.}
\begin{gather*}
    f\colon \real \mapsto \real \text{ jest ciągła}\\
    g\colon \real \mapsto \real \text{ jest ciągła}\\
    \tag{\(\ast\)} \forall x \in \real\colon f\pars{g\pars{x}} = g\pars{f\pars{x}} \label{2020_10_09:3_6:equality}\\
    \tag{\(\star\)} \forall x \in \real\colon f\pars{x} \neq g\pars{x} \label{2020_10_09:3_6:inequality}
\end{gather*}
Chcemy pokazać, że z tych założeń wynika
\begin{equation*}
    \tag{T} \forall x \in \real\colon f\pars{f\pars{x}} \neq g\pars{g\pars{x}} \label{2020_10_09:3_6:theorem}
\end{equation*}
Przeprowadzimy dowód nie wprost. Załóżmy zatem, że
\begin{equation*}
    \tag{!} \exists x_0 \in \real\colon f\pars{f\pars{x_0}} = g\pars{g\pars{x_0}} \label{2020_10_09:3_6:assumption}
\end{equation*}
Dokonajmy podstawienia \(a = f\pars{x_0}\) i~\(b = g\pars{x_0}\). Na mocy (\ref{2020_10_09:3_6:equality}) mamy
\begin{equation*}
    f\pars{b} = g\pars{a}
\end{equation*}
Natomiast równość z~(\ref{2020_10_09:3_6:assumption}) zapisujemy inaczej jako
\begin{equation*}
    f\pars{a} = g\pars{b}
\end{equation*}
Jeśli \(a = b\), to natychmiast mamy sprzeczność z~(\ref{2020_10_09:3_6:inequality}), więc teza (\ref{2020_10_09:3_6:theorem}) musi być prawdziwa. Załóżmy że \(a < b\) (w~odwrotnej konfiguracji dowód przebiega analogicznie). Jeśli \(f\pars{a} = f\pars{b}\), to \(f\pars{a} = g\pars{a}\), bo wiemy że \(f\pars{b} = g\pars{a}\). Od razu dostajemy zatem sprzeczność z~(\ref{2020_10_09:3_6:inequality}), czyli teza (\ref{2020_10_09:3_6:theorem}) musi być prawdziwa. Przyjmijmy zatem bez straty ogólności, że \(f\pars{a} < f\pars{b}\), (w~odwrotnej konfiguracji dowód przebiega analogicznie). Mamy zatem
\begin{equation*}
    f\pars{a} = g\pars{b} < f\pars{b} = g\pars{a}
\end{equation*}
Rozważmy teraz funkcję
\begin{equation*}
    h\pars{x} = f\pars{x} - g\pars{x}
\end{equation*}
Jest ona różnicą funkcji z~założenia ciągłych w~\(\real\), więc sama jest ciągła w~\(\real\). Zbadajmy jej zachowanie w~przedziale \(\closed{a}{b}\).
\begin{gather*}
    h\pars{a} = f\pars{a} - g\pars{a} < 0\\
    h\pars{b} = f\pars{b} - g\pars{b} > 0
\end{gather*}
Skoro ciągła funkcja \(h\pars{x}\) przyjmuje na końcach przedziału wartości wartości różnych znaków, to na mocy twierdzenia Darboux musi mieć w~tym przedziale miejsce zerowe. Oznacza to, że istnieje w~tym przedziale takie \(x_0\), że
\begin{gather*}
    h\pars{x_0} = 0\\
    f\pars{x_0} - g\pars{x_0} = 0\\
    f\pars{x_0} = g\pars{x_0}
\end{gather*}
Doszliśmy zatem do sprzeczności z~(\ref{2020_10_09:3_6:inequality}), więc ostatecznie teza (\ref{2020_10_09:3_6:theorem}) musi być prawdziwa.
\qed
