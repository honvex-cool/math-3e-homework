\subsection*{Reguła de l'Hospitala}
\subsubsection*{Zadanie~37.}
\begin{itemize}
    \item[a)]
        \begin{itemize}
        \end{itemize}
        \begin{equation*}
            \begin{split}
                \limit[x \to 0] \frac{x\cot x - 1}{x^2}
                    &= \limit[x \to 0] \frac{\converges{1}{\frac{x}{\tan x}} - 1}{x^2}
                    = \indeterminate{\frac{0}{0}}
                    = \limit[x \to 0] \frac{\frac{x\cos x - \sin x}{\sin x}}{x^2}
                    = \limit[x \to 0] \frac{x\cos x - \sin x}{x^2\sin x}
                    \lhospitaleq \limit[x \to 0] \frac{\cos x - x\sin x - \cos x}{2x\sin x + x^2\cos x}\\
                    &= \limit[x \to 0] \frac{-\cancel{x}\sin x}{2\cancel{x}\sin x + x^{\cancel{2}}\cos x}
                    = \limit[x \to 0] \frac{-\sin x}{2\sin x + x\cos x}
                    \lhospitaleq \limit[x \to 0] \frac{-\converges{1}{\cos x}}{2\converges*{1}{\cos x} + \converges*{1}{\cos x} - \converges*{0}{x\sin x}}
                    = -\frac{1}{3}
            \end{split}
        \end{equation*}
    \item[b)]
        \begin{equation*}
            \begin{split}
                \limit[x \to \frac{\pi}{4}] \frac{\sqrt[3]{\tan x} - 1}{2\sin^2x - 1}
                    = \indeterminate{\frac{0}{0}}
                    \lhospitaleq \frac{\frac{1}{3\sqrt[3]{\tan^2x}} \cdot \frac{1}{\cos^2x}}{4\sin x\cos x}
                    = \frac{\frac{1}{3\sqrt[3]{\tan^2\frac{\pi}{4}}} \cdot \frac{1}{\cos^2\frac{\pi}{4}}}{2\sin\pars{2 \cdot \frac{\pi}{4}}}
                    = \frac{\frac{2}{3}}{2}
                    = \frac{1}{3}
            \end{split}
        \end{equation*}
    \item[d)]
        \begin{equation*}
            \limit[x \to 0] \frac{e^{\converges{-\infty}{-1 / x^2}}}{x^{100}}
                = \indeterminate{\frac{0}{0}}
                = \limit[x \to 0] \frac{\frac{1}{e^{\frac{1}{x^2}}}}{x^{100}}
                = \limit[x \to 0] \frac{\frac{1}{x^{100}}}{e^{\frac{1}{x^2}}}
                \lhospitaleq \limit[x \to 0] \frac{\frac{-100}{x^{101}}}{-\frac{2}{x^3}e^{\frac{1}{x^2}}}
                = \limit[x \to 0] \frac{\frac{50}{x^{98}}}{e^{\frac{1}{x^2}}}
                = \limit[x \to 0] \frac{50e^{-\frac{1}{x^2}}}{x^{98}}
                \lhospitaleq \ldots
                \lhospitaleq \limit[x \to 0] \frac{ce^{-\frac{1}{x^2}}}{x^0}
        \end{equation*}
        Gdzie \(c\) jest pewną wartością stałą. Zatem
        \begin{equation*}
            \limit[x \to 0] \frac{e^{-\frac{1}{x^2}}}{x^{100}} = \frac{ce^{-\infty}}{1} = 0
        \end{equation*}
    \item[e)]
        \begin{equation*}
            \limit[x \to 0^+] x^{\frac{k}{1 + \ln x}}
                = \indeterminate{0^0}
                = \limit[x \to 0^+] \pars{e^{\ln x}}^{\frac{k}{1 + \ln x}}
                = \limit[x \to 0^+] e^{\pars{\frac{k}{1 + \ln x}\ln x}}
        \end{equation*}
        Ponieważ funkcja wykładnicza jest ciągła, to \(\limit[x \to 0^+] e^{\pars{\frac{k}{1 + \ln x}\ln x}} = e^{\pars{\limit[x \to 0^+] \frac{k}{1 + \ln x}\ln x}}\). Wystarczy zatem, jeśli obliczymy granicę wykładnika:
        \begin{equation*}
            \limit[x \to 0^+] \frac{k\ln x}{1 + \ln x}
        \end{equation*}
        Gdy \(k = 0\) jest to po prostu funkcja stała równa \(1\). Wtedy \(\limit[x \to 0^+] x^{\frac{k}{1 + \ln x}} = \limit[x \to 0^+] x^0 = 1\). Rozważmy teraz przypadek, gdy \(k \neq 0\):
        \begin{equation*}
            \limit[x \to 0^+] \frac{k\ln x}{1 + \ln x}
                = \indeterminate{\frac{\mp\infty}{-\infty}}
                \lhospitaleq \limit[x \to 0^+] \frac{\frac{k}{x}}{\frac{1}{x}}
                = k
        \end{equation*}
        Zatem
        \begin{equation*}
            \limit[x \to 0^+] x^{\frac{k}{1 + \ln x}} = e^k
        \end{equation*}
    \item[f)]
        \begin{equation*}
            \limit[x \to \frac{\pi}{4}] \pars{\tan x}^{\tan2x}
                = \indeterminate{1^{\pm\infty}}
                = \limit[x \to \frac{\pi}{4}] \pars{e^{\ln\pars{\tan x}}}^{\tan2x}
                = \limit[x \to \frac{\pi}{4}] e^{\pars{\tan2x\ln\pars{\tan x}}}
        \end{equation*}
        Ponieważ funkcja wykładnicza jest ciągła, to \(\limit[x \to \frac{\pi}{4}] e^{\pars{\tan2x\ln\pars{\tan x}}} = e^{\pars{\limit[x \to \frac{\pi}{3}] \tan2x\ln\pars{\tan x}}}\). Wystarczy zatem, jeśli obliczmy granicę wykładnika:
        \begin{equation*}
            \begin{split}
                \limit[x \to \frac{\pi}{4}^\pm] \tan2x\ln\pars{\tan x}
                    &= \limit[x \to \frac{\pi}{4}] \frac{\ln\pars{\converges{1}{\tan x}}}{\frac{1}{\converges*{\pm\infty}{\tan 2x}}}
                    = \indeterminate{\frac{0}{0}}
                    = \limit[x \to \frac{\pi}{4}] \frac{\ln\pars{\converges{1}{\tan x}}}{\cot2x}
                    \lhospitaleq \limit[x \to \frac{\pi}{4}] \frac{\frac{1}{\tan x} \cdot \frac{1}{\cos^2x}}{-\frac{2}{\sin^22x}}\\
                    &= \frac{\frac{1}{\tan\frac{\pi}{4}} \cdot \frac{1}{\cos^2\frac{\pi}{4}}}{-\frac{2}{\sin^2\pars{2 \cdot \frac{\pi}{4}}}}
                    = \frac{1 \cdot \frac{1}{2}}{-2}
                    = -1
            \end{split}
        \end{equation*}
        Zatem
        \begin{equation*}
            \limit[x \to \frac{\pi}{4}] \pars{\tan x}^{\tan2x} = e^{-1} = \frac{1}{e}
        \end{equation*}
    \item[g)]
        \begin{equation*}
            \limit[x \to 0^+] \pars{\ln\converges{+\infty}{\frac{1}{x}}}^x
                = \indeterminate{\pars{+\infty}^0}
                = \limit[x \to 0^+] \pars{e^{\ln\pars{\ln\frac{1}{x}}}}^x
                = \limit[x \to 0^+] e^{x\ln\pars{\ln\frac{1}{x}}}
        \end{equation*}
        Ponieważ funkcja wykładnicza jest ciągła, to \(\limit[x \to 0^+] e^{x\ln\pars{\ln\frac{1}{x}}} = e^{\pars{\limit[x \to 0^+] x\ln\pars{\ln\frac{1}{x}}}}\). Wystarczy zatem, jeśli obliczymy granicę z~wykładnika:
        \begin{equation*}
            \limit[x \to 0^+] x\ln\frac{1}{x}
                = \limit[x \to 0^+] \frac{\ln\pars{\converges{+\infty}{\ln\frac{1}{x}}}}{\frac{1}{\converges*{0}{x}}}
                = \indeterminate{\frac{+\infty}{+\infty}}
                \lhospitaleq \limit[x \to 0^+] \frac{\frac{1}{\ln\frac{1}{x}} \cdot \frac{1}{\frac{1}{x}} \cdot \cancel{\pars{-\frac{1}{x^2}}}}{\cancel{-\frac{1}{x^2}}}
                = \limit[x \to 0^+] \frac{\converges{0}{x}}{\converges*{+\infty}{\ln\frac{1}{x}}}
                = 0
        \end{equation*}
        Zatem
        \begin{equation*}
            \limit[x \to 0^+] \pars{\ln\frac{1}{x}}^x = e^0 = 1
        \end{equation*}
\end{itemize}
\subsection*{Zadania~optymalizacyjne}
\subsubsection*{Zadanie~1.}
\begin{equation*}
    P = \pars{\frac{1}{2}m + \frac{5}{2}; m} \qquad m \in \closed{-1}{7}\\
    Q = \pars{\frac{55}{2}; 0}
\end{equation*}
Z~twierdzenia Pitagorasa:
\begin{equation*}
    PQ^2 = \pars{\frac{1}{2}m + \frac{5}{2} - \frac{55}{2}}^2 + \pars{m - 0}^2
        = \pars{\frac{1}{2}m - 25} + m^2
        = \frac{1}{4}m^2 - 25m + 625 + m^2
        = \frac{5}{4}m^2 - 25m + 625
\end{equation*}
Zdefiniujmy funkcję odległości \(P\) od \(Q\) w~zależności od \(m\):
\begin{equation*}
    f\pars{m} = \frac{5}{4}m^2 - 25m + 625
        = 5\pars{\frac{1}{4}m^2 - 5m + 125}
\end{equation*}
Taka funkcja kwadratowa przyjmuje globalną wartość najmniejszą, ponieważ współczynnik kierujący jest dodatni, czyli ramiona paraboli są skierowane w~stronę rosnących współrzędnych \(y\). Globalna wartość najmniejsza jest przyjmowana dla \(m = \frac{-\pars{-25}}{2 \cdot \frac{5}{4}} = 10 \not\in \closed{-1}{7}\). Zatem w~przedziale domkniętym \(\open{-1}{7}\) funkcja jest malejąca, a~ekstrema znajdują się na końcach przedziału:
\begin{gather*}
    f_\p{max} = f\pars{-1} = \frac{5}{4} + 650 = 651{,}25\\
    f_\p{min} = f\pars{7} = \frac{245}{4} - 175 + 625 = \frac{245}{4} + 450 = 511{,}25
\end{gather*}
Zatem największa możliwa wartość \(PQ^2\) to \(651{,}25\), a~najmniejsza to \(511{,}25\).
\subsubsection*{Zadanie~2.}
Zauważmy, że jeśli \(x > 1\), to z~nierówności między średnimi:
\begin{gather*}
    \frac{x^2 + 2}{2} \geq \sqrt{x^2 \cdot 2}\\
    x^2 + 2 \geq 2\sqrt{2} \cdot \sqrt{x^2} = 2\sqrt{2} \cdot x \geq \sqrt{x}\\
    x^2 + 2 \geq \sqrt{x}
\end{gather*}
Natomiast jeśli \(x \in \closed{0}{1}\), to \(\sqrt{x} \in \closed{0}{1}\) oraz \(x^2 + 2 \geq 2 > 1\). Zatem
\begin{equation*}
    \forall x \in \leftclosed{0}{+\infty}\colon f\pars{x} \geq g\pars{x}
\end{equation*}
Zatem, aby znaleźć długość pionowego odcinka dla pewnego \(x\), wystarczy, że odejmiemy wartość funkcji \(g\pars{x}\) (mniejszą) od wartości \(f\pars{x}\) (większej). Zdefiniujmy funkcję długości takiego odcinka w~zależności od \(x\):
\begin{equation*}
    \ell\pars{x}
        = x^2 + 2 - \sqrt{x} \qquad x \in \leftclosed{0}{+\infty}
\end{equation*}
Obliczmy jej pochodną:
\begin{equation*}
    \ell'\pars{x}
        = 2x - \frac{1}{2\sqrt{x}}
        = \frac{4x\sqrt{x} - 1}{2\sqrt{x}}
        = \frac{4\pars{\sqrt{x}}^3 - 1}{2\sqrt{x}}
        = \frac{\pars{\sqrt[3]{4} \cdot \sqrt{x} - 1}\overbrace{\pars{\sqrt[3]{16} \cdot \pars{\sqrt{x}}^2 + \sqrt[3]{4} \cdot \sqrt{x} + 1}}^{> 0}}{2\sqrt{x}} \qquad x \in \open{0}{+\infty}
\end{equation*}
Mianownik jest dodatni, więc znak pochodnej zależy tylko od licznika, konkretnie od pierwszego wyrażenia. Wykres znaku pochodnej:
\begin{gather*}
    \begin{tikzpicture}
        \drawvec (-2, 0) -- (2, 0) node[below]{\(\sqrt{x}\)};
        \draw[thick] (-1.5, -1.5) -- (1.5, 1.5);
        \fillpoint*{0, 0}[\(\frac{1}{\sqrt[3]{4}}\)][below right];
    \end{tikzpicture}\\
    \begin{tikzpicture}
        \drawvec (-2, 0) -- (2, 0) node[below]{\(x\)};
        \draw[thick] (-1.5, -1.5) -- (1.5, 1.5);
        \fillpoint*{0, 0}[\(\frac{1}{\sqrt[3]{16}}\)][below right];
    \end{tikzpicture}
\end{gather*}
Interesuje nas tylko przedział \(\leftclosed{0}{+\infty}\). Ponieważ dla \(x = 0\) pochodna nie istnieje, ale dąży do \(-\infty\) a~w~przedziale \(\open{0}{\frac{1}{\sqrt[3]{16}}}\) jest ujemna, to znaczy, że funkcja \(\ell\) jest w~malejąca w~przedziale \(\open{0}{\frac{1}{\sqrt[3]{16}}}\), więc dla \(x = 0\) przyjmuje maksimum lokalne:
\begin{equation*}
    \ell\pars{0} = 2
\end{equation*}
Pochodna jest ujemna w~przedziale \(\open{0}{\frac{1}{\sqrt[3]{16}}}\), dla \(x = \frac{1}{\sqrt[3]{16}}\) przyjmuje wartość \(0\), a~w przedziale \(\open{\frac{1}{\sqrt[3]{16}}}{+\infty}\) jest dodatnia. Zatem funkcja \(\ell\) jest malejąca w~przedziale \(\open{0}{\frac{1}{\sqrt[3]{16}}}\) i~rosnąca w~przedziale \(\open{\frac{1}{\sqrt[3]{16}}}{+\infty}\), więc dla \(x = \frac{1}{\sqrt[3]{16}}\) przyjmuje globalną wartość najmniejszą:
\begin{equation*}
    \ell\pars{\frac{1}{\sqrt[3]{16}}}
        = \frac{1}{4\sqrt[3]{4}} + 2 - \frac{1}{\sqrt[3]{4}}
        = \frac{1}{4\sqrt[3]{4}} + \frac{8\sqrt[3]{4}}{4\sqrt[3]{4}} - \frac{4}{4\sqrt[3]{4}}
        = \frac{8\sqrt[3]{4} - 3}{4\sqrt[3]{4}}
\end{equation*}
\subsubsection*{Zadanie~3.}
\begin{mathfigure*}
    \drawcoordsystem{-7, -7}{7, 7};
    \draw[domain=-1.318:5.318, smooth, thick, ForestGreen] plot (\x, {-\x*\x+4*\x});
    \draw[thick, RoyalBlue] (2, -7) -- (2, 7) node[below left]{\(x = 2\)};
    \draw[domain=-4:3, smooth, thick, Red] plot (\x, {2*\x + 1});
    \fillpoint*{2, 4}[\(\pars{2; 4}\)][above right];
    \fillpoint{1, 3};
    \fillpoint*{2, 5}[\(C\)][above left];
    \fillpoint*{2, 0}[\(B\)][below right];
    \fillpoint*{0, 0}[\(A\)][below right];
    \fillpoint*{0, 1}[\(D\)][above left];
\end{mathfigure*}
Jest to trapez \(ABCD\), przy czym kąty \(\angle{ABC}\) i~\(\angle{DAB}\) są proste. Zatem wysokością tego trapezu jest odcinek \(AB\) od długości \(2\). Pole wyraża się zatem wzorem
\begin{equation*}
    \frac{\pars{AD + BC} \cdot AB}{2}
        = \frac{\pars{AD + BC} \cdot 2}{2}
        = AD + BC
\end{equation*}
Skoro \(y\pars{x} = -x^2 + 4x\), to \(y'\pars{x} = -2x + 4\). Styczna do paraboli w~punkcie \(\pars{p; y\pars{p}}\) ma zatem równanie
\begin{equation*}
    t\pars{x} = y'\pars{p}\pars{x - p} + y\pars{p}
        = \pars{-2p + 4}\pars{x - p} - p^2 + 4p
        = -2px + 2p^2 + 4x - 4p - p^2 + 4p
        = -2px + 4x + p^2 
\end{equation*}
Zatem
\begin{gather*}
    AD = t\pars{0} = p^2\\
    BC = t\pars{2} = -4p + 8 + p^2
\end{gather*}
Zdefiniujmy funkcję pola powierzchni trapezu w zależności od \(p\):
\begin{equation*}
    S\pars{p}
        = AD + BC
        = p^2 - 4p + 8 + p^2
        = 2p^2 - 4p + 8 \qquad p \in \real
\end{equation*}
Jest to funkcja kwadratowa z~dodatnim współczynnikiem kierującym, zatem ramiona paraboli są skierowane w~stronę rosnących współrzędnych \(y\), czyli funkcja przyjmuje globalną wartość najmniejszą dla argumentu \(\frac{-\pars{-4}}{2 \cdot 2} = 1 \in \real\). Wartość ta wynosi \(S\pars{1} = 2 - 4 + 8 = 6\). Zatem, interpretując wynik, optymalny punkt styczności prostej do paraboli to
\begin{equation*}
    A = \pars{1; y\pars{1}} = \pars{1; 3}
\end{equation*}
\subsubsection*{Zadanie~4.}
\begin{mathfigure*}
    \coordinate (A) at (-1.5, -0.5);
    \coordinate (B) at (0.5, -0.5);
    \coordinate (C) at (1.5, 0.5);
    \coordinate (D) at (-0.5, 0.5);
    \coordinate (E) at (-1.5, 3.5);
    \coordinate (F) at (0.5, 3.5);
    \coordinate (G) at (1.5, 4.5);
    \coordinate (H) at (-0.5, 4.5);
    \coordinate (X) at (-0.5, -0.5);
    \coordinate (Y) at (1, 4);
    \coordinate (W) at (1, 0);
    \draw (A) -- (B) -- (C);
    \draw[dashed] (C) -- (D) -- (A);
    \drawrightangle[angle radius=0.25cm]{W--B--X};
    \draw (E) -- (F) -- (G) -- (H) -- cycle;
    \draw[ForestGreen] (Y) -- node[above left]{\(d\)} (X);
    \draw (Y) -- node[right]{\(h\)} (W) -- (X);
    \drawrightangle[angle radius=0.4cm]{Y--W--X};
    \draw (A) -- (E);
    \draw (B) -- (F);
    \draw (C) -- (G);
    \draw[dashed] (D) -- (H);
    \fillpoint*{X}[\(X\)][below];
    \fillpoint*{Y}[\(Y\)][above left];
    \fillpoint*{W}[\(W\)][below right];
    \fillpoint*{B}[\(B\)][below left];
\end{mathfigure*}
Przyjmijmy, że krawędź podstawy tego graniastosłupa ma długość \(a\). Skoro \(X\) jest środkiem krawędzi podstawy, to \(XB = \frac{a}{2}\). Podobnie, skoro \(Y\) jest środkiem krawędzi podstawy, a~wysokość graniastosłupa opada na podstawę pod kątem prostym, to \(WB = \frac{a}{2}\). Zatem \(XW = \frac{a\sqrt{2}}{2} = \frac{a}{\sqrt{2}}\). Z~twierdzenia Pitagorasa w~\(\triangle{XWY}\) mamy
\begin{gather*}
    d^2 = h^2 + \pars{\frac{a}{\sqrt{2}}}^2 = h^2 + \frac{a^2}{2}
    h = \sqrt{d^2 - \frac{a^2}{2}}
\end{gather*}
Zdefiniujmy funkcję powierzchni bocznej graniastosłupa w~zależności od \(a\):
\begin{equation*}
    S\pars{a}
        = 4ah
        = 4a\sqrt{d^2 - \frac{a^2}{2}} \qquad a \in \open{0}{d\sqrt{2}}
\end{equation*}
Wyznaczmy pochodną tej funkcji:
\begin{equation*}
    S'\pars{a}
        = 4\sqrt{d^2 - \frac{a^2}{2}} + \frac{-4a^2}{2\sqrt{d^2 - \frac{a^2}{2}}}
        = \frac{8d^2 - 4a^2 - 4a^2}{2\sqrt{d^2 - \frac{a^2}{2}}}
        = \frac{8d^2 - 8a^2}{2\sqrt{d^2 - \frac{a^2}{2}}}
\end{equation*}
Znak pochodnej zależy tylko od licznika, ponieważ mianownik jest zawsze dodatni:
\begin{gather*}
    8d^2 - 8a^2 = -8\pars{a^2 - d^2} = -8\pars{a + d}\pars{a - d}\\
    \downparabola{-d}{d}[\(a\)]
\end{gather*}
Interesuje nas tylko przedział \(\open{0}{d\sqrt{2}}\). Pochodna jest dodatnia w~przedziale \(\open{0}{d}\), dla \(a = d\) przyjmuje wartość \(0\), a~w~przedziale \(\open{d}{d\sqrt{2}}\) jest ujemna. Oznacza to, że funkcja \(S\) jest rosnąca w~przedziale \(\open{0}{d}\) i~malejąca w~przedziale \(\open{d}{d\sqrt{2}}\), więc dla \(a = d\) przyjmuje globalną wartość największą:
\begin{equation*}
    S_\p{max}
        = S\pars{d}
        = 4d\sqrt{d^2 - \frac{d^2}{2}}
        = 4d \cdot \frac{d}{\sqrt{2}}
        = 2\sqrt{2}d^2
\end{equation*}
Wtedy \(h = \sqrt{d^2 - \frac{d^2}{2}} = \sqrt{\frac{d^2}{2}} = \frac{d}{\sqrt{2}} = \frac{d\sqrt{2}}{2}\).
\subsubsection*{Zadanie~5.}
Jeśli ostrosłup jest prawidłowy, to w~podstawie jest trójkąt równoboczny, a~spodek wysokości znajduje się w~jego środku. Poniżej schemat podstawy:
\begin{mathfigure*}
    \def\rt{\fpeval{sqrt(3)}}
    \coordinate (A) at (-2, 0);
    \coordinate (B) at (2, 0);
    \coordinate (C) at (0, 2*\rt);
    \coordinate (S) at (0, 2*\rt/3);
    \draw (A) -- node[below]{\(a\)} (B) -- (C) -- cycle;
    \draw (S) circle[radius=4*\rt/3];
    \draw (S) -- node[above, sloped]{\(R\)} (B);
    \fillpoint*{S}[\(S\)][above];
\end{mathfigure*}
Ponieważ trójkąt jest równoboczny, to \(R = \frac{2}{3} \cdot \frac{a\sqrt{3}}{2} = \frac{a\sqrt{3}}{3}\). Zatem \(h = 24 - R = 24 - \frac{a\sqrt{3}}{3}\). Zdefiniujmy funkcję objętości ostrosłupa w~zależności od \(a\):
\begin{equation*}
    V\pars{a}
        = \frac{1}{3} \cdot \frac{a^2\sqrt{3}}{4} \cdot h
        = \frac{1}{3} \cdot \frac{a^2\sqrt{3}}{4} \cdot \pars{24 - \frac{a\sqrt{3}}{3}}
        = 2\sqrt{3}a^2 - \frac{a^3\sqrt{3} \cdot \sqrt{3}}{36}
        = 2\sqrt{3}a^2 - \frac{1}{12}a^3 \qquad a \in \open{0}{24\sqrt{3}}
\end{equation*}
Obliczmy jej pochodną:
\begin{equation*}
    V'\pars{a}
        = 4\sqrt{3}a - \frac{a^2}{4}
        = \frac{16\sqrt{3}a - a^2}{4}
        = \frac{a\pars{16\sqrt{3} - a}}{4}
\end{equation*}
Wykres znaku pochodnej:
\begin{equation*}
    \downparabola{0}{16\sqrt{3}}[\(a\)]
\end{equation*}
Interesuje nas tylko przedział \(\open{0}{24\sqrt{3}}\). W~przedziale \(\open{0}{16\sqrt{3}}\) pochodna jest dodatnia, dla \(a = 16\sqrt{3}\) przyjmuje wartość \(0\), a~w~przedziale \(\open{16\sqrt{3}}{24\sqrt{3}}\) jest ujemna. Zatem funkcja \(V\) jest rosnąca w~przedziale \(\open{0}{16\sqrt{3}}\) i~malejąca w~przedziale \(\open{16\sqrt{3}}{24\sqrt{3}}\), więc dla \(a = 16\sqrt{3}\) przyjmuje globalną wartość największą:
\begin{equation*}
    V\pars{16\sqrt{3}}
        = \frac{1}{3} \cdot \frac{758\sqrt{3}}{4} \cdot \pars{24 - \frac{16\sqrt{3} \cdot \sqrt{3}}{3}}
        = 64\sqrt{3} \cdot 8
        = 512\sqrt{3}
\end{equation*}
Ta objętość jest przyjmowana dla \(R = \frac{16\sqrt{3} \cdot \sqrt{3}}{3} = 16\).
\subsubsection*{Zadanie~6.}
\begin{equation*}
    \pi\dm^3 = 1000\pi\cm^3
\end{equation*}
W~tym zadaniu, ilekroć będziemy mówić o~objętości naczynia, będziemy mieć na myśli objętość samej bryły, którą jest naczynie. Objętość jego wnętrza będziemy nazywać pojemnością. Wszystkie wymiary będziemy od teraz podawać w~odpowiednio \(\cm\), \(\cm^2\) i~\(\cm^3\). Jeśli przy stałej gęstości szkła chcemy zminimalizować masę naczynia, musimy zminimalizować jego objętość. Objętość naczynia jest równa objętości całego walca, z~którego wycięto naczynie, pomniejszone o~pojemność tego naczynia. Przyjmijmy, że wewnętrzny promień naczynia wynosi \(r\). Wtedy
\begin{gather*}
    \pi r^2h = 1000\pi\\
    h = \frac{1000}{r^2}
\end{gather*}
gdzie \(h\) to wysokość wewnętrzna. Promień zewnętrzny wynosi więc \(r + 2\), a~wysokość zewnętrzna wynosi \(h + 2 = \frac{1000}{r^2} + 2\). Zdefiniujmy zatem funkcję objętości naczynia w~zależności od \(r\):
\begin{equation*}
    V\pars{r}
        = \pars{r + 2}^2 \cdot h - 1000\pi
        = \pars{r^2 + 2r + 4}\pars{\frac{1000}{r^2} + 2}
        = 1000 + 2r^2 + \frac{2000}{r} + 4r + \frac{4000}{r^2} + 8 - 1000\pi \qquad r \in \open{0}{+\infty}
\end{equation*}
Obliczmy pochodną tej funkcji:
\begin{equation*}
    V'\pars{r}
        = 4r - \frac{2000}{r^2} + 4 - \frac{8000}{r^3}
        = \frac{4r^4 - 2000r + 4r^3 - 8000}{r^3}
\end{equation*}
\subsubsection*{Zadanie~7.}
Rozważmy przekrój płaszczyzną zawierającą średnicę podstawy i~wierzchołek stożka:
\begin{mathfigure*}
    \def\rt{\fpeval{sqrt(3)}}
    \coordinate (A) at (-4, 0);
    \coordinate (B) at (4, 0);
    \coordinate (C) at (0, 4*\rt);
    \coordinate (S) at (0, 0);
    \coordinate (D) at (-1.5, 2.5*\rt);
    \coordinate (E) at (1.5, 2.5*\rt);
    \coordinate (F) at (-1.5, 0);
    \coordinate (G) at (1.5, 0);
    \coordinate (I) at (0, 3*\rt);
    \coordinate (J) at (0, 2.5*\rt);
    \drawrightangle[angle radius=0.4cm]{B--S--C};
    \drawrightangle[angle radius=0.4cm]{B--G--E};
    \drawangle[angle radius=0.4cm, RoyalBlue] {C--B--A};
    \drawangle[angle radius=0.4cm, RoyalBlue] {B--A--C};
    \drawangle[angle radius=0.4cm, RoyalBlue] {A--C--B};
    \drawangle[angle radius=0.4cm, RoyalBlue] {C--E--D};
    \drawangle[angle radius=0.4cm, RoyalBlue] {E--D--C};
    \draw (A) -- (B) -- (C) -- cycle;
    \draw[dashed] (C) -- node[right]{\(23\sqrt{3}\)} (S);
    \path (S) -- node[below]{\(23\)} (B);
    \path (S) -- node[below]{\(r\)} (G);
    \draw[ForestGreen] (F) -- (D) -- (E) -- node[right]{\(h\)} (G);
    \draw[Orange] (I) circle[radius=0.5*\rt];
    \draw[Orange] (I) -- node[right]{\(R\)} (J);
    \fillpoint{I};
    \fillpoint*{S}[\(S\)][below];
    \fillpoint*{B}[\(B\)][below right];
    \fillpoint*{E}[\(E\)][above right];
    \fillpoint*{C}[\(C\)][above];
    \fillpoint*{G}[\(G\)][below];
    \fillpoint*{D}[\(D\)][above left];
    \fillpoint*{A}[\(A\)][below left];
    \fillpoint*{J}[\(J\)][below left];
\end{mathfigure*}
\noindent
Skoro przekrojem jest trójkąt równoboczny, to wysokość stożka wynosi \(23\sqrt{3}\). Zauważmy, że \(\mangle{EGB} = \mangle{CSB} = 90\degree\) i~\(\mangle{EBG} = \mangle{CBS}\). Zatem \(\triangle{EGB} \sim \triangle{CSB}\), czyli
\begin{gather*}
    \frac{EG}{GB} = \frac{CS}{SB}\\
    \frac{h}{23 - r} = \frac{23\sqrt{3}}{23}\\
    \frac{h}{23 - r} = \sqrt{3}\\
    h = 23\sqrt{3} - r\sqrt{3}
\end{gather*}
Ponieważ \(DE \parallel AB\), to \(\triangle{CDE}\) także jest równoboczny. Oznacza to, że promień okręgu wpisanego w~ten trójkąt, czyli promień kuli wpisanej w~stożek odcięty górną podstawą walca, wynosi \(\frac{1}{3}\) wysokości tego trójkąta:
\begin{equation*}
    R = \frac{CJ}{3} = \frac{23\sqrt{3} - h}{3} = \frac{23\sqrt{3} - \pars{23\sqrt{3} - r\sqrt{3}}}{3}
        = \frac{r\sqrt{3}}{3}
\end{equation*}
Możemy już zatem wyrazić objętość walca i~kuli wyłącznie z~użyciem \(r\). Zdefiniujmy zatem funkcję sumy objętości tych dwóch brył w~zależności od \(r\):
\begin{equation*}
    V\pars{r}
        = {\overbrace{\pi r^2h}^{\text{walec}}} + {\overbrace{\frac{4}{3}\pi R^3}^{\text{kula}}}
        = \pi r^2 \cdot \pars{23\sqrt{3} - r\sqrt{3}} + \frac{4}{3}\pi \cdot \pars{\frac{r\sqrt{3}}{3}}^3
        = \pi\pars{23\sqrt{3}r^2 - r^3\sqrt{3} + \frac{4\sqrt{3}r^3}{27}} \qquad r \in \open{0}{23}
\end{equation*}
Obliczmy pochodną tej funkcji:
\begin{equation*}
    V'\pars{r}
        = \pi\pars{46\sqrt{3}r - 3\sqrt{3}r^2 + \frac{4\sqrt{3}r^2}{9}}
        = \pi\sqrt{3}r\pars{46 - 3r + \frac{4r}{9}}
        = \pi\sqrt{3}r\pars{46 - \frac{23r}{9}}
\end{equation*}
Wykres znaku pochodnej:
\begin{equation*}
    \downparabola{0}{18}[\(r\)]
\end{equation*}
Interesuje nas tylko przedział \(\open{0}{23}\). W~przedziale \(\open{0}{18}\) pochodna jest dodatnia, dla \(r = 18\) przyjmuje wartość \(0\), a~przedziale \(\open{18}{23}\) jest ujemna. Oznacza to, że funkcja \(V\) jest rosnąca w~przedziale \(\open{0}{18}\) i~malejąca w~przedziale \(\open{18}{23}\), więc dla \(r = 18\) przyjmuje globalną wartość największą:
\begin{equation*}
    V\pars{18}
        = \pi\pars{23\sqrt{3} \cdot 18^2 - 18^3 \cdot \sqrt{3} + \frac{4\sqrt{3} \cdot 18^3}{27}}
        = \pi\pars{7452\sqrt{3} - 5832\sqrt{3} + 864\sqrt{3}}
        = 2484\pi\sqrt{3}
\end{equation*}
Aby była przyjmowana największa możliwa objętość równa \(2484\pi\sqrt{3}\), bryły muszą mieć następujące wymiary:
\begin{gather*}
    r = 18\\
    h = 23\sqrt{3} - 18\sqrt{3} = 5\sqrt{3}\\
    R = \frac{18\sqrt{3}}{3} = 6\sqrt{3}
\end{gather*}
