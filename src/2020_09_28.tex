\subsection*{Zestaw~VI (zadania otwarte)}
\subsubsection*{Zadanie~1.}
\begin{gather*}
    x^2 - 2x + 2y^2 + 8y + 9 \geq 0\\
    x^2 - 2x + 1 + 2\parens{y^2 + 4y + 4} \geq 0\\
    (x - 1)^2 + 2(y + 2)^2 \geq 0
\end{gather*}
Ostatnia nierówność jest prawdziwa, ponieważ suma kwadratów liczb rzeczywistych jest zawsze nieujemna. Wszystkie przejścia były równoważne, więc teza również jest prawdziwa.
\qed
\subsubsection*{Zadanie~2.}
\begin{gather*}
    W(x) = x^3 - (p + 1)x^2 + (p - 3)x + 3
\end{gather*}
Zauważmy, że współczynniki sumują się do \(0\):
\begin{equation*}
    W(1) = 1 - (p + 1) + (p - 3) + 3 = 1 - p - 1 + p - 3 + 3 = 0
\end{equation*}
Zatem jednym z~pierwiastków jest \(x = 1 \in \integer\), niezależnie od \(p\). Możemy dokończyć rozkład wielomianu na postać iloczynową za pomocą schematu Hornera (Tablica~\ref{2020_09_28:2:table:horner})
\begin{table}[H]
    \centering
    \begin{tabular}{c|c|c|c|c}
        \(c\) & \(x^3\) & \(-(p + 1)x^2\) & \((p - 3)x\) & \(3\)\\
        \hline
        & \(1\) & \(-p - 1\) & \(p - 3\) & \(3\)\\
        \hline
        \(1\) & \(1\) & \(-p - 1 + 1 \cdot 1 = -p\) & \(p - 3 + 1 \cdot (-p) = p - 3 - p = -3\) & \(3 + 1 \cdot (-3) = 3 - 3 = 0\)\\
        \hline
        & \(1\) & \(-p\) & \(-3\) & \(0\)
    \end{tabular}
    \caption{Rozkład wielomianu z~zadania~2. za pomocą schematu Hornera}
    \label{2020_09_28:2:table:horner}
\end{table}
\begin{equation*}
    W(x) = (x - 1)\parens{x^2 - px - 3}
\end{equation*}
Zauważmy, że wielomian \(x^2 - px - 3\) ma zawsze dwa różne pierwiastki, ponieważ \(\Delta > 0\):
\begin{gather*}
    \Delta = (-p)^2 - 4 \cdot 1 \cdot (-3) = p^2 + 12\\
    p^2 + 12 > 0 \text{ dla każdej liczby rzeczywistej \(p\)}
\end{gather*}
Przyjmijmy zatem, że te pierwiastki to \(x_1\) i~\(x_2\):
\begin{gather*}
    x_1 = \frac{-(-p) - \sqrt{\Delta}}{2 \cdot 1} = \frac{p - \sqrt{p^2 + 12}}{2} < \frac{p - \sqrt{p^2}}{2} = \frac{p - p}{2} = 0\\
    x_2 = \frac{-(-p) + \sqrt{\Delta}}{2 \cdot 1} = \frac{p + \sqrt{p^2 + 12}}{2}
\end{gather*}
Ponieważ z~wzorów Viete'a mamy \(x_1 \cdot x_2 = \frac{-3}{1} = -3\) i~\(x_1 < 0\), to musi zachodzić \(x_2 > 0\). Rozważmy teraz trzy sposoby, na jakie pierwiastki wielomianu \(W(x)\) mogą utworzyć ciąg arytmetyczny:
\begin{enumerate}
    \item ciąg \((x_1, 1, x_2)\) lub ciąg \((x_2, 1, x_1)\). Z~warunku, aby trzy liczby tworzyły ciąg arytmetyczny, mamy:
        \begin{gather*}
            2 \cdot 1 = x_1 + x_2\\
            x_1 + x_2 = 2
        \end{gather*}
        Z~wzorów Viete'a mamy zależność \(x_1 + x_2 = \frac{-(-p)}{1} = p\). Zatem \(p = x_1 + x_2 = 2\), czyli dla \(p = 2\) pierwiastki tego wielomianu tworzą ciąg arytmetyczny.
    \item ciąg \((x_1, x_2, 1)\) lub ciąg \((1, x_2, x_1)\). Z~warunku, aby trzy liczby tworzyły ciąg arytmetyczny, mamy:
        \begin{gather*}
            2x_2 = x_1 + 1\\
            2 \cdot \frac{p + \sqrt{p^2 + 12}}{2} = \frac{p - \sqrt{p^2 + 12}}{2} + 1\\
            2p + 2\sqrt{p^2 + 12} = p - \sqrt{p^2 + 12} + 2\\
            2 - p = 3\sqrt{p^2 + 12}\\
            4 - 4p + p^2 = 9p^2 + 108\\
            8p^2 + 4p + 104 = 0\\
            \Delta = 4^2 - 4 \cdot 8 \cdot 104 < 0
        \end{gather*}
        Zatem wśród liczb rzeczywistych nie ma takich \(p\), dla których pierwiastki tworzą ciąg arytmetyczny w~ten sposób.
    \item ciąg \((x_2, x_1, 1)\) lub ciąg \((1, x_1, x_2)\). Z~warunku, aby trzy liczby tworzyły ciąg arytmetyczny, mamy:
        \begin{equation*}
            2x_1 = x_2 + 1
        \end{equation*}
        Jednak to nie jest możliwe, ponieważ \(x_1 < 0\), więc lewa strona jest ujemna, a~\(x_2 > 0\), więc prawa strona jest dodatnia. Okazuje się więc, że w~tym przypadku nie ma rozwiązań.
\end{enumerate}
Zatem jedyną wartością \(p\), dla której pierwiastki wielomianu \(W(x)\) tworzą ciąg arytmetyczny, jest \(p = 2\).
\qed
\subsubsection*{Zadanie~3.}
\begin{gather*}
    x^3 - x^2 - 4 = 0
\end{gather*}
Widać, że dla \(x = 2\) równanie jest prawdziwe. Teraz na pomocą schematu Hornera (Tablica~\ref{2020_09_28:3:table:horner}) możemy rozłożyć wielomian na postać iloczynową:
\begin{table}[H]
    \centering
    \begin{tabular}{c|c|c|c|c}
        \(c\) & \(x^3\) & \(-x^2\) & \(0x\) & \(-4\)\\
        \hline
        & \(1\) & \(-1\) & \(0\) & \(-4\)\\
        \hline
        \(2\) & \(1\) & \(-1 + 2 \cdot 0 = 1\) & \(0 + 2 \cdot 1 = 2\) & \(-4 + 2 \cdot 2 = 0\)\\
        \hline
        & \(1\) & \(2\) & \(2\) & \(0\)
    \end{tabular}
    \caption{Rozkład wielomianu z~zadania~3. za pomocą schematu Hornera}
    \label{2020_09_28:3:table:horner}
\end{table}
\begin{equation*}
    (x - 2) \cdot \underbrace{\parens{x^2 + 2x + 2}}_{\Delta = 2^2 - 4 \cdot 1 \cdot 2 = 4 - 8 = -4 < 0}
\end{equation*}
Jest to jedyny możliwy rozkład wielomianu na postać iloczynową. Zatem jedynym pierwiastkiem jest \(x = 2\).
\qed
\subsubsection*{Zadanie~4.}
\begin{gather*}
    x^3 + y^3 \geq x^2y + xy^2\\
    (x + y)\parens{x^2 - xy + y^2} \geq x^2y + xy^2\\
    (x + y)\parens{x^2 - xy + y^2} - x^2y + xy^2 \geq 0\\
    (x + y)\parens{x^2 - xy + y^2} - xy(x + y) \geq 0\\
    (x + y)\parens{x^2 - xy + y^2 - xy} \geq 0\\
    (x + y)\parens{x^2 - 2xy + y^2} \geq 0\\
    (x + y)(x - y)^2 \geq 0
\end{gather*}
Skoro \(x \geq 0\) i~\(y \geq 0\), to \(x + y \geq 0\). Mamy zatem iloczyn liczby nieujemnej i~kwadratu liczby rzeczywistej, więc jest on na pewno nieujemny.
\qed
\subsubsection*{Zadanie~5.}
\begin{equation*}
    \frac{mx}{m - 2} + \frac{m + 2}{x} = x + 2\sqrt{2}
\end{equation*}
Od razu zauważmy, że \(x \neq 0\) i~\(m - 2 \neq 0 \implies m \neq 2\), ponieważ mianowniki nie mogą się zerować.
\begin{gather*}
    \frac{mx \cdot x}{x(m - 2)} + \frac{(m + 2)(m - 2)}{x(m - 2)} = \frac{\parens{x + 2\sqrt{2}}x(m - 2)}{x(m - 2)}\\
    \frac{mx^2 + (m + 2)(m - 2) - \parens{x + 2\sqrt{2}}x(m - 2)}{x(m - 2)} = 0\\
    \frac{mx^2 + m^2 - 4 - mx^2 + 2x^2 - 2\sqrt{2}mx + 4\sqrt{2}x}{x(m - 2)} = 0\\
    mx^2 + m^2 - 4 - mx^2 + 2x^2 - 2\sqrt{2}mx + 4\sqrt{2}x = 0\\
    2x^2 + \parens{4\sqrt{2} - 2\sqrt{2}m}x + \parens{m^2 - 4} = 0
\end{gather*}
Sprawdźmy na początek, kiedy to równanie ma dwa pierwiastki (być może jeden wielokrotny). Aby tak było, musi zachodzić warunek:
\begin{gather*}
    \Delta \geq 0\\
    \Delta
        = \parens{4\sqrt{2} - 2\sqrt{2}m}^2 - 4 \cdot 2 \cdot \parens{m^2 - 4}
        = 32 - 32m + 8m^2 - 8m^2 + 32
        = 32m + 64\\
    -32m + 64 \geq 0\\
    -32m \geq -64\\
    32m \leq 64\\
    m \leq 2\\
    m \in \rightclosed{-\infty}{2}
\end{gather*}
Kiedy równanie kwadratowe ma dwa pierwiastki \(x_1, x_2\), możemy skorzystać z~wzorów Viete'a:
\begin{gather*}
    x_1 + x_2 = \frac{-\parens{4\sqrt{2} - 2\sqrt{2}m}}{2} = \sqrt{2}m - 2\sqrt{2}\\
    x_1x_2 = \frac{m^2 - 4}{2}
\end{gather*}
Sumę odwrotności pierwiastków możemy zapisać następująco:
\begin{equation*}
    \frac{1}{x_1} + \frac{1}{x_2}
        = \frac{x_1 + x_2}{x_1x_2}
        = \frac{\sqrt{2}m - 2\sqrt{2}}{\frac{m^2 - 4}{2}}
        = \frac{2\sqrt{2}m - 4\sqrt{2}}{m^2 - 4}
        = \frac{2\sqrt{2}\cancel{(m - 2)}}{\cancel{(m - 2)}(m + 2)}
        = \frac{2\sqrt{2}}{m + 2}
\end{equation*}
Chcemy, aby zachodziła nierówność
\begin{gather*}
    \frac{1}{x_1} + \frac{1}{x_2} < \frac{\sqrt{2}}{m - 1}\\
    \frac{2\cancel{\sqrt{2}}}{m + 2} < \frac{\cancel{\sqrt{2}}}{m - 1}\\
    \frac{2}{m + 2} < \frac{1}{m - 1}\\
    \frac{1}{m - 1} - \frac{2}{m + 2} > 0\\
    \frac{m + 2}{(m - 1)(m + 2)} - \frac{2m - 2}{(m - 1)(m + 2)} > 0\\
    \frac{4 - m}{(m - 1)(m + 2)} > 0\\
    -(m - 4)(m - 1)(m + 2) > 0
\end{gather*}
Pierwiastkami tego wielomianu są \(0\), \(1\) i~\(2\), a~współczynnik przy najwyższej potędze jest ujemny. Możemy naszkicować wykres tego wielomianu:
\begin{mathfigure*}
    \drawvec[scale=2] (-1, 0) -- (3, 0);
    \draw[scale=2] (-0.25, 0.5)
        -- (0, 0)
        .. controls (0.37, -0.7) .. (1, 0)
        .. controls (1.6, 0.7) .. (2, 0)
        -- (2.25, -0.5);
    \fillpoint*[2]{0, 0}[\(-2\)][below left];
    \fillpoint*[2]{1, 0}[\(1\)][below right];
    \fillpoint*[2]{2, 0}[\(4\)][below left];
    \node[ForestGreen, scale=2] at (-1, 0.5) {\(+\)};
    \node[ForestGreen, scale=2] at (3.08, 0.5) {\(+\)};
    \node[red, scale=2] at (0.9, -0.5) {\(-\)};
    \node[red, scale=2] at (5, -0.5) {\(-\)};
\end{mathfigure*}
Odczytujemy z~niego, że \(m \in \open{-\infty}{2} \cup \open{1}{2}\).
\subsubsection*{Zadanie~6.}
\begin{gather*}
    \cos 2x + \sin x = 1\\
    1 - 2\sin^2x + \sin x = 1\\
    -2\sin^2x + \sin x = 0\\
    2\sin^2x - \sin x = 0\\
    \sin x(2\sin x - 1) = 0\\
    \sin x = 0 \lor \sin x = \frac{1}{2}\\
    x = k\pi,\quad k \in \integer \lor x = 2\ell\pi + \frac{\pi}{6},\quad \ell \in \integer \lor x = (2m + 1)\pi - \frac{\pi}{6},\quad m \in \integer
\end{gather*}
Wyznaczmy teraz rozwiązania nierówności
\begin{gather*}
    \log_x (x + 2) < 2
\end{gather*}
Aby logarytm po lewej stronie istniał, muszą zachodzić warunki:
\begin{itemize}
    \item \(x > 0\)
    \item \(x \neq 1\)
    \item \(x + 2 > 0\)
\end{itemize}
Zatem \(x \in \real_{> 0} \setminus \set{1}\).
\begin{gather*}
    x^{\log_x (x + 2)} < x^2\\
    x + 2 < x^2\\
    x^2 - x - 2 > 0\\
    (x + 1)(x - 2) > 0\\
    x \in \open{-\infty}{-1} \cup \open{2}{+\infty}
\end{gather*}
Po połączeniu wyniku z~dziedziną:
\begin{equation*}
    x \in \open{2}{+\infty}
\end{equation*}
Po połączeniu tej informacji z~rozwiązaniami równania otrzymujemy:
\begin{equation*}
    x = k\pi,\quad k \in \integer_{> 0} \lor x = 2\ell\pi + \frac{\pi}{6},\quad \ell \in \integer_{> 0} \lor x = (2m + 1)\pi - \frac{\pi}{6},\quad m \in \integer_{\geq 0}
\end{equation*}
\subsubsection*{Zadanie~7.}
\begin{equation*}
    \begin{cases}
        3\abs{x} + 2y = 1\\
        2x - \abs{y} = 4
    \end{cases}
\end{equation*}
Dokonamy analizy przypadków:
\begin{proofcases}
    \item \(x \geq 0 \land y \geq 0\):
        \begin{equation*}
            \begin{cases}
                3x + 2y = 1\\
                2x - y = 4
            \end{cases}
        \end{equation*}
        Możemy pomnożyć drugie równanie stronami przez \(2\) i~dodać stronami równania:
        \begin{gather*}
            \begin{cases}
                3x + 2y = 1\\
                4x - 2y = 8
            \end{cases}\\
            3x + 2y + 4x - 2y = 9\\
            7x = 9\\
            x = \frac{9}{7} \geq 0
        \end{gather*}
        Z drugiego równania mamy
        \begin{gather*}
            2 \cdot \frac{9}{7} - y = 4\\
            y = \frac{18}{7} - 4 = \frac{18}{7} - \frac{28}{7} = -\frac{10}{7} < 0
        \end{gather*}
        Otrzymaliśmy wynik, który nie należy do rozważanego przedziału dla \(y\), więc nie jest to rozwiązanie.
    \item \(x \geq 0 \land y < 0\):
        \begin{equation*}
            \begin{cases}
                3x + 2y = 1\\
                2x + y = 4
            \end{cases}
        \end{equation*}
        Możemy pomnożyć drugie równanie stronami przez \(-2\) i~dodać stronami równania:
        \begin{gather*}
            \begin{cases}
                3x + 2y = 1\\
                -4x - 2y = -8
            \end{cases}\\
            3x + 2y - 4x - 2y = -7\\
            -x = -7\\
            x = 7
        \end{gather*}
        Z~drugiego równania mamy
        \begin{gather*}
            2 \cdot 7 + y = 4\\
            14 + y = 4\\
            y = -10 < 0
        \end{gather*}
        Otrzymane wyniki należą do rozważanych przedziałów dla odpowiednio \(x\) i~\(y\), więc jest to rozwiązanie:
        \begin{equation*}
            \begin{cases}
                x = 7\\
                y = -10
            \end{cases}
        \end{equation*}
    \item \(x < 0 \land y \geq 0\):
        \begin{equation*}
            \begin{cases}
                -3x + 2y = 1\\
                2x - y = 4
            \end{cases}
        \end{equation*}
        Możemy pomnożyć drugie równanie stronami przez \(2\) i~dodać równania stronami:
        \begin{gather*}
            \begin{cases}
                -3x + 2y = 1\\
                4x - 2y = 8
            \end{cases}\\
            -3x + 2y + 4x - 2y = 9\\
            x = 9 \geq 0
        \end{gather*}
        Otrzymaliśmy wynik, który nie należy do rozważanego przedziału dla \(x\), więc na pewno nie będzie to rozwiązanie. Nie ma zatem sensu kontynuować wyznaczanie \(y\).
    \item \(x < 0 \land y < 0\):
        \begin{equation*}
            \begin{cases}
                -3x + 2y = 1\\
                2x + y = 4
            \end{cases}
        \end{equation*}
        Możemy pomnożyć drugie równanie stronami przez \(-2\) i~dodać równania stronami:
        \begin{gather*}
            \begin{cases}
                -3x + 2y = 1\\
                -2x - 2y = -8
            \end{cases}\\
            -3x + 2y - 2x - 2y = -7\\
            -5x = -7\\
            x = \frac{7}{5} \geq 0
        \end{gather*}
        Orzymaliśmy wynik, który nie należy do rozważanego przedziału dla \(x\), więc na pewno nie będzie to rozwiązanie. Nie ma zatem sensu kontynuować wyznaczanie \(y\).
\end{proofcases}
Ostatecznie po rozważeniu wszystkich możliwych przypadków otrzymujemy jedno rozwiązanie:
\begin{equation*}
    \begin{cases}
        x = 7\\
        y = -10
    \end{cases}
\end{equation*}
Możemy również przekształcić początkowe równania i~narysować krzywe, które one opisują, aby rozwiązać układ równań graficznie:
\begin{equation*}
    \begin{cases}
        y = \frac{1 - 3\abs{x}}{2}\\
        x = \frac{4 + \abs{y}}{2}
    \end{cases}
\end{equation*}
\begin{mathfigure*}
    \drawcoordsystem[0.5]{-4, -12}{10, 3};
    \draw[scale=0.5, thick, ForestGreen, domain=-4:8, samples=100] plot (\x, {(1 - 3 * abs(\x)) / 2});
    \draw[scale=0.5, thick, red, domain=-12:3, variable=\x, samples=100] plot ({(4 + abs(\x)) / 2}, \x);
    \fillpoint*[0.5]{7, -10}[\((7; -10)\)][right];
\end{mathfigure*}
Tak jak w~poprzedniej metodzie, otrzymujemy jedno rozwiązanie:
\begin{equation*}
    \begin{cases}
        x = 7\\
        y = -10
    \end{cases}
\end{equation*}
\subsubsection*{Zadanie~8.}
\begin{equation*}
    \frac{1}{1 - x} + \frac{1}{(1 - x)^2} + \frac{1}{(1 - x)^3} + \ldots = 1 - 2x
\end{equation*}
Na pewno \(x \neq 1\), ponieważ mianowniki muszą być niezerowe. Jest to szereg geometryczny o~pierwszym wyrazie równym \(a_1 = \frac{1}{1 - x}\) i~ilorazie równym \(q = \frac{1}{1 - x}\). Szereg jest zbieżny, co daje nam warunek
\begin{gather*}
    \tag{\(\ast\)} \abs{q} < 1 \label{2020_09_28:8:quotient}\\
    \abs{\frac{1}{1 - x}} < 1
\end{gather*}
Przy takim założeniu możemy skorzystać ze wzoru na sumę szeregu geometrycznego:
\begin{equation*}
    \frac{1}{1 - x} + \frac{1}{(1 - x)^2} + \frac{1}{(1 - x)^3} + \ldots
        = \frac{a_1}{1 - q}
        = \frac{\frac{1}{1 - x}}{1 - \frac{1}{1 - x}}
\end{equation*}
Widzimy tu, że \(x \neq 0\), ponieważ inaczej mianownik wynosiłby \(1 - \frac{1}{1 - 0} = 1 - 1 = 0\), co jest niemożliwe.
\begin{gather*}
    \frac{\frac{1}{1 - x}}{1 - \frac{1}{1 - x}}
        = \frac{\frac{1}{1 - x}}{\frac{1 - x - 1}{1 - x}}
        = \frac{\frac{1}{\cancel{1 - x}}}{\frac{-x}{\cancel{1 - x}}}
        = \frac{1}{-x}
        = -\frac{1}{x}\\
    -\frac{1}{x} = 1 - 2x
\end{gather*}
Możemy pomnożyć obie strony równania przez \(x\), ponieważ \(x \neq 0\).
\begin{gather*}
    -1 = x - 2x^2\\
    2x^2 - x - 1 = 0\\
    (x - 1)(2x + 1) = 0\\
    x = 1 \lor x = -\frac{1}{2}
\end{gather*}
Na samym początku rozwiązania wykluczyliśmy \(x = 1\), więc jedynym potencjalnym rozwiązaniem jest \(x = -\frac{1}{2}\). Musimy jednak jeszcze sprawdzić, czy spełniony jest przy nim warunek \refer{2020_09_28:8:quotient}:
\begin{gather*}
    \abs{q} < 1\\
    \abs{\frac{1}{1 - x}} < 1\\
    \abs{\frac{1}{1 - \parens{-\frac{1}{2}}}} = \abs{\frac{1}{\frac{3}{2}}} = \abs{\frac{2}{3}} = \frac{2}{3} < 1
\end{gather*}
Warunek jest spełniony, zatem rozwiązaniem jest \(x = -\frac{1}{2}\).
\subsubsection*{Zadanie~9.}
Oznaczmy przez \(u \in \natural\) cyfrę jedności szukanej liczby, a~przez \(d \in \natural\) cyfrę dziesiątek szukanej liczby. Cyfra dziesiątek nie może być równa \(0\), ponieważ szukana liczba jest dwucyfrowa. Obowiązują zatem warunki:
\begin{gather*}
    \tag{\(\star\)} 1 \leq d \leq 9 \label{2020_09_28:9:decimal}\\
    \tag{\(\ast\)} 0 \leq u \leq 9 \label{2020_09_28:9:unit}\\
\end{gather*}
Zapiszmy warunki, które ma spełniać szukana liczba:
\begin{gather*}
    \begin{cases}
        u + d = 12\\
        u^2 + d^2 = 80
    \end{cases}\\
    u^2 + d^2 = (u + d)^2 - 2ud\\
    80 = 12^2 - 2ud\\
    80 = 144 - 2ud\\
    2ud = 64\\
    ud = 32
\end{gather*}
Wypiszmy wszystkie możliwe iloczyny dwóch liczb naturalnych (z~uwzględnieniem kolejności), które w~wyniku dają \(32\):
\begin{equation*}
    \begin{split}
        32 &= u \cdot d\\
        &= 1 \cdot 32\\
        &= 2 \cdot 16\\
        &= 4 \cdot 8\\
        &= 8 \cdot 4\\
        &= 16 \cdot 2\\
        &= 32 \cdot 1
    \end{split}
\end{equation*}
Jedynymi parami liczb \((u, d)\), które spełniają warunki \refer{2020_09_28:9:decimal} i~\refer{2020_09_28:9:unit} przedstawione na samym początku rozwiązania są
\begin{gather*}
    (u, d) = (4, 8)\\
    (u, d) = (8, 4)
\end{gather*}
Zatem liczby spełniające warunki zadania to \(48\) i~\(84\).
\subsubsection*{Zadanie~10.}
Trzy kolejne liczby nieparzyste możemy zapisać od najmniejszej do największej jako \(2n - 1, 2n + 1, 2n + 3\), gdzie \(n \in \integer\). Zapiszmy dane z~zadania:
\begin{gather*}
    (2n - 1)(2n + 1)(2n + 3) = (2n + 3)^2 - (2n - 1)^2 + 65\\
    \parens{4n^2 - 1}(2n + 3) = 4n^2 + 12n + 9 - 4n^2 + 4n - 1 + 65\\
    8n^3 + 12n^2 - 2n - 3 = 16n + 73\\
    8n^3 + 12n^2 - 18n - 76 = 0\\
    4n^3 + 6n^2 - 9n - 38 = 0
\end{gather*}
Widzimy, że jednym z~pierwiastków jest \(n = 2\). Możemy teraz z~użyciem schematu Hornera rozłożyć wielomian na postać iloczynową:
\begin{gather*}
    (n - 2)\parens{4n^2 + 14n + 19}
\end{gather*}
Drugi z~czynników jest wielomianem stopnia \(2\), w~którym
\begin{equation*}
    \Delta = 14^2 - 4 \cdot 4 \cdot 19 = 196 - 304 = -108 < 0
\end{equation*}
Zatem drugi czynnik nie może być już rozłożony na postać iloczynową, bo skoro \(\Delta < 0\), to nie ma pierwiastków. Zatem jedynym rozwiązaniem równania jest \(n = 2\). W~takim razie, liczby spełniające warunki zadania to:
\begin{gather*}
    2n - 1 = 3\\
    2n + 1 = 5\\
    2n + 3 = 7
\end{gather*}
Rzeczywiście, ich iloczyn wynosi \(105\) i~jest o~\(65\) większy od różnicy kwadratów lizczb największej i~najmniejszej: \(105 = 40 + 65 = 49 - 9 + 65 = 7^2 - 3^2 + 65\). Zatem rozwiązaniem zadania są liczby \(3, 5, 7\).
\subsubsection*{Zadanie~11.}
\begin{equation*}
    \frac{6x}{x - 2} - \sqrt{\frac{12x}{x - 2}} - 2\sqrt[4]{\frac{12x}{x - 2}}
\end{equation*}
Natychmiast zauważamy, że mianowniki nie mogą być równe \(0\), więc \(x \neq 2\). Dokonajmy podstawienia \(t = \sqrt[4]{\frac{12x}{x - 2}}\):
\begin{gather*}
    \frac{t^4}{2} - t^2 - 2t > 0\\
    t^4 - 2t^2 - 4t > 0
\end{gather*}
Zauważamy, że \(0\) i~\(2\) są pierwiastkami wielomianu po lewej stronie. Zatem możemy z~pomocą schematu Hornera rozłożyć wielomian na postać iloczynową:
\begin{equation*}
    t(t - 2)(t^2 + 2t + 2) > 0
\end{equation*}
Ostatni z~czynników nie ma pierwiastków, ponieważ jest to wielomian stopnia \(2\), w~którym
\begin{equation*}
    \Delta = 2^2 - 4 \cdot 1 \cdot 2 = 4 - 8 = -4 < 0
\end{equation*}
Zatem jedynymi pierwiastkami tego wielomianu są \(0\) i~\(2\). Współczynnik przy najwyższej potędze jest dodatni, więc nierówność jest spełniona, gdy
\begin{gather*}
    t \in \open{-\infty}{0} \cup \open{2}{+\infty}\\
    t < 0 \lor t > 2\\
    \sqrt[4]{\frac{12x}{x - 2}} < 0 \text{ (niemożliwe)} \lor \sqrt{4}{\frac{12x}{x - 2}} > 2
\end{gather*}
Pierwsza nierówność nie ma rozwiązań, ponieważ pierwiastek parzystego stopnia jest zawsze nieujemny. Skupmy się zatem na drugiej nierówności. Ponieważ obie strony są nieujemne, możemy podnieść je do czwartej potęgi:
\begin{gather*}
    \sqrt[4]{\frac{12x}{x - 2}} > 2\\
    \frac{12x}{x - 2} > 16\\
    \frac{12x - 16x + 32}{x - 2} > 0\\
    \frac{-4x + 32}{x - 2} > 0\\
    \frac{x - 8}{x - 2} < 0\\
    (x - 2)(x - 8) < 0\\
    x \in \open{2}{8}
\end{gather*}
Zatem rozwiązaniem nierówności jest \(x \in \open{2}{8}\).
\subsubsection*{Zadanie~12.}
\begin{equation*}
    \abs{\frac{3x - 2}{x + 1}} \geq 3\\
\end{equation*}
Zauważamy, że \(x \neq -1\), ponieważ mianownik musi być różny od \(0\).
\begin{gather*}
    \abs{\frac{3(x + 1) - 5}{x + 1}} \geq 3\\
    \abs{3 - \frac{5}{x + 1}} \geq 3\\
    \abs{\frac{5}{x + 1} - 3} \geq 3
\end{gather*}
Możemy rozważyć dwa przypadki:
\begin{proofcases}
    \item
        \begin{gather*}
            \frac{3x - 2}{x + 1} \geq 0\\
            (x + 1)(3x - 2) \geq 0\\
            x \in \open{-\infty}{-1} \cup \leftclosed{\frac{2}{3}}{+\infty}
        \end{gather*}
        W~tym przypadku mamy:
        \begin{gather*}
            \frac{3x - 2}{x + 1} \geq 3\\
            \frac{3x - 2}{x + 1} - \frac{3x + 3}{x + 1} \geq 0\\
            \frac{-5}{x + 1} \geq 0\\
            x + 1 < 0\\
            x \in \rightclosed{-\infty}{-1}
        \end{gather*}
        Po uwzględnieniu faktu, że \(x \neq -1\), otrzymujemy przedział \(\open{-\infty}{-1}\), który zawiera się w~rozważanym przedziale, więc jest rozwiązaniem.
    \item
        \begin{gather*}
            \frac{3x - 2}{x + 1} < 0\\
            (x + 1)(3x - 2) < 0\\
            x \in \open{-1}{\frac{2}{3}}
        \end{gather*}
        W~tym przypadku mamy:
        \begin{gather*}
            \frac{2 - 3x}{x + 1} \geq 3\\
            \frac{2 - 3x}{x + 1} - \frac{3x + 3}{x + 1} \geq 0\\
            \frac{-6x - 1}{x + 1} \geq 0\\
            \frac{6x + 1}{x + 1} \leq 0\\
            (x + 1)(6x + 1) \leq 0\\
            x \in \closed{-1}{-\frac{1}{6}}
        \end{gather*}
        Po uwzględnieniu faktu, że \(x \neq -1\), otrzymujemy przedział \(\rightclosed{-1}{-\frac{1}{6}}\), który zawiera się w~rozważanym przedziale, więc jest rozwiązaniem.
\end{proofcases}
Ostatecznie otrzymujemy przedział rozwiązań będący sumą przedziałów otrzymanych w~rozważanych przypadkach:
\begin{equation*}
    x \in \open{-\infty}{-1} \cup \rightclosed{-1}{-\frac{1}{6}}
\end{equation*}
Zatem największą liczbą spełniającą nierówność \(\abs{\frac{3x - 2}{x + 1}} \geq 3\) jest liczba \(-\frac{1}{6}\).

Możemy również rozwiązać to zadanie graficznie. Najpierw szkicujemy wykres funkcji \(\frac{3x}{x + 1}\). W~tym celu wyznaczamy:
\begin{itemize}
    \item asymptotę pionową: ma ona równanie \(x = -1\), ponieważ \(-1\) nie należy do dziedziny, gdyż mianownik musi być różny od \(0\)
    \item asymptotę poziomą: ma ona równanie \(y = 3\), ponieważ \(\limit[x \to +\infty] \frac{3x - 2}{x + 1} = 3\)
    \item wartość funkcji dla \(x = 0\): \(\frac{3 \cdot 0 - 2}{0 + 1} = -2\). Zatem wykres funkcji przetnie oś \(Oy\) w~punkcie \((0; -2)\)
    \item miejsca zerowe: \(\frac{3x - 2}{x + 1} = 0 \iff 3x - 2 = 0 \iff x = \frac{2}{3}\). Zatem wykres funkcji przetnie oś \(Ox\) w~punkcie \(\parens{\frac{2}{3}; 0}\)
\end{itemize}
\begin{mathfigure*}
    \drawcoordsystem[0.25]{-15, -15}{15, 15};
    \draw[scale=0.25, domain=-15:-1.415, very thick, ForestGreen, smooth, samples=50] plot (\x, {(3 * \x - 2) / (\x + 1)});
    \draw[scale=0.25, domain=-0.723:15, very thick, ForestGreen, smooth, samples=50] plot (\x, {(3 * \x - 2) / (\x + 1)});
    \draw[scale=0.25, dashed] (-1,15) -- (-1, -15);
    \draw[scale=0.25, dashed] (-15, 3) -- (15, 3);
    \fillpoint[0.25]{2 / 3, 0};
    \fillpoint*[0.25]{0, -2}[\tiny\((0; -2)\)][left];
    \draw[->, scale=0.25] (3, -1) node[right]{\tiny\(\parens{\frac{2}{3}; 0}\)} -- (0.77, -0.2);
\end{mathfigure*}
Następnie odbijamy symetrycznie fragment znajdujący się poniżej osi \(Ox\):
\begin{mathfigure*}
    \drawcoordsystem[0.25]{-15, -15}{15, 15};
    \draw[scale=0.25, domain=-15:-1.415, very thick, ForestGreen, smooth, samples=50] plot (\x, {abs((3 * \x - 2) / (\x + 1))});
    \draw[scale=0.25, domain=-0.723:15, very thick, ForestGreen, samples=70] plot (\x, {abs((3 * \x - 2) / (\x + 1))});
    \draw[scale=0.25, domain=-0.723:(2/3), dotted, thick, smooth, samples=50] plot (\x, {(3 * \x - 2) / (\x + 1)});
    \draw[scale=0.25, dashed] (-1,15) -- (-1, -15);
    \draw[scale=0.25, RoyalBlue, thick] (-15, 3) -- (15, 3);
    \fillpoint[0.25]{2 / 3, 0};
    \fillpoint*[0.25]{0, -2}[\tiny\((0; -2)\)][left];
    \fillpoint*[0.25]{0, 2}[\tiny\((0; 2)\)][right];
    \fillpoint*[0.25]{-1/6, 3}[\tiny\(\parens{-\frac{1}{6}; 3}\)][above right];
    \draw[->, scale=0.25] (3, -1) node[right]{\tiny\(\parens{\frac{2}{3}; 0}\)} -- (0.77, -0.2);
\end{mathfigure*}
Widzimy, że rozwiązania powinniśmy poszukwiać wartości \(x\), dla której zachodzi równość, w~przedziale \(\open{-1}{0}\). W~takim razie musimy wziąć liczbę przeciwną do liczby pod modułem:
\begin{gather*}
    \abs{\frac{3x - 2}{x + 1}} = 3\\
    \frac{2 - 3x}{x + 1} = 3\\
    2 - 3x = 3x + 3\\
    6x = -1\\
    x = -\frac{1}{6}
\end{gather*}
Otrzymaliśmy dokładnie taki sam wynik jak z~użyciem poprzedniej metody.
\subsubsection*{Zadanie~1.16.}
\begin{enumerate}[label={\alph*)}]
    \item
        \begin{equation*}
            f(x) = \begin{cases}
                1 + x & \iff x > 0\\
                1 - x & \iff x \leq 0
            \end{cases}
        \end{equation*}
        \begin{description}
            \item[\(x_0 = 0\)]
                \begin{gather*}
                    \limit[x \to 0^-] f(x)
                        = \limit[x \to 0^-] (1 - x)
                        = 1 - 0^-
                        = 1\\
                    \limit[x \to 0^+] f(x)
                        = \limit[x \to 0^+] (1 + x)
                        = 1 + 0^+
                        = 1
                \end{gather*}
            \item[\(x_0 = 1\)]
                \begin{gather*}
                    \limit[x \to 1^{-}] f(x)
                        = \limit[x \to 1^-] (1 + x)
                        = 1 + 1
                        = 2\\
                    \limit[x \to 1^+] f(x)
                        = \limit[x \to 1^+] (1 + x)
                        = 1 + 1
                        = 2
                \end{gather*}
        \end{description}
    \item
        \begin{equation*}
            f(x) = \begin{cases}
                \frac{\sin{x}}{\abs{x}} & \iff x \neq 0\\
                1 & \iff x = 0
            \end{cases}
        \end{equation*}
        \begin{description}
            \item[\(x_0 = 0\)]
                \begin{gather*}
                    \limit[x \to 0^-] f(x)
                        = \limit[x \to 0^{-}] \frac{\sin x}{-x}
                        = -1\\
                    \limit[x \to 0^+] f(x)
                        = \limit[x \to 0^+] \frac{\sin x}{x}
                        = 1
                \end{gather*}
            \item[\(x_0 = \pi\)]
                \begin{gather*}
                    \limit[x \to \pi^-] f(x)
                        = \limit[x \to \pi^-] \frac{\sin x}{x}
                        = \frac{\sin \pi}{\pi}
                        = \frac{0}{\pi}
                        = 0\\
                    \limit[x \to \pi^+] f(x)
                        = \limit[x \to \pi^+] \frac{\sin x}{x}
                        = \frac{\sin \pi}{\pi}
                        = \frac{0}{\pi}
                        = 0
                \end{gather*}
        \end{description}
    \item
        \begin{equation*}
            f(x) = \begin{cases}
                2 - x & \iff x > 0\\
                x^3 - 4 & \iff x \leq 0
            \end{cases}
        \end{equation*}
        \begin{description}
            \item[\(x_0 = 0\)]
                \begin{gather*}
                    \limit[x \to 0^-] f(x)
                        = \limit[x \to 0^-] x^3 - 4
                        = \parens{0^-}^3 - 4
                        = -4\\
                    \limit[x \to 0^+] f(x)
                        = \limit[x \to 0^+] 2 - x
                        = 2 - 0^+
                        = 2
                \end{gather*}
            \item[\(x_0 = 1\)]
                \begin{gather*}
                    \limit[x \to 1^-] f(x)
                        = \limit[x \to 1^-] 2 - x
                        = 2 - 1
                        = 1\\
                    \limit[x \to 1^+] f(x)
                        = \limit[x \to 1^+] 2 - x
                        = 2 - 1
                        = 1
                \end{gather*}
        \end{description}
    \item
        \begin{equation*}
            f(x) = \begin{cases}
                \frac{\sin x}{x} & \iff x > 0\\
                \cos x & \iff x \leq 0
            \end{cases}
        \end{equation*}
        \begin{description}
            \item[\(x_0 = 0\)]
                \begin{gather*}
                    \limit[x \to 0^-] f(x)
                        = \limit[x \to 0^-] \cos{x}
                        = \cos 0^-
                        = 1\\
                    \limit[x \to 0^+] f(x)
                        = \limit[x \to 0^+] \frac{\sin x}{x}
                        = 1
                \end{gather*}
        \end{description}
    \item
        \begin{equation*}
            f(x) = x \cdot \floor{\frac{1}{x}}
        \end{equation*}
        \begin{description}
            \item[\(x_0 = 1\)]
                \begin{gather*}
                    \limit[x \to 1^-] f(x)
                        = \limit[x \to 1^-] x \cdot \floor{\frac{1}{x}}
                        = 1 \cdot \floor{\frac{1}{1}}
                        = 1\\
                    \limit[x \to 1^+] f(x)
                        = \limit[x \to 1^+] x \cdot \floor{\frac{1}{x}}
                        = 1 \cdot \floor{\frac{1}{1}}
                        = 1
                \end{gather*}
        \end{description}
    \item
        \begin{equation*}
            f(x) = \begin{cases}
                x^3 + x^2 + x + 1 & \iff x \geq 1\\
                0 & \iff x < 1
            \end{cases}
        \end{equation*}
        \begin{description}
            \item[\(x_0 = 1\)]
                \begin{gather*}
                    \limit[x \to 1^-] f(x)
                        = \limit[x \to 1^-] 0
                        = 0\\
                    \limit[x \to 1^+] f(x)
                        = \limit[x \to 1^+] \parens{x^3 + x^2 + x + 1}
                        = (1 + 1 + 1 + 1)
                        = 4
                \end{gather*}
        \end{description}
\end{enumerate}
\subsubsection*{Zadanie~1.18.}
\begin{enumerate}[label={\alph*)}]
    \item
        \begin{equation*}
            f(x) = \frac{\abs{x}(x + 2)}{x(x - 1)^3(x - 2)^2}
        \end{equation*}
        \begin{description}
            \item[\(x_0 = 0\)]
                \begin{gather*}
                    \begin{split}
                        \limit[x \to 0^-] f(x)
                            &= \limit[x \to 0^-] \frac{x(x + 2)}{x(x - 1)^3(x - 2)^2}
                            = \indeterminate{\frac{0}{0}}
                            = \limit[x \to 0^-] \frac{-\cancel{x}(x + 2)}{\cancel{x}(x - 1)^3(x - 2)^2}
                            = \limit[x \to 0^-] \frac{-x - 2}{(x - 1)^3(x - 2)^2}\\
                            &= \frac{-0^- - 2}{(0^- - 1)^3(0^- - 2)^2}
                            = \frac{-2}{(-1)^3(-2)^2}
                            = \frac{-2}{-4}
                            = \frac{1}{2}
                    \end{split}\\\
                    \begin{split}
                        \limit[x \to 0^+] f(x)
                            &= \limit[x \to 0^+] \frac{x(x + 2)}{x(x - 1)^3(x - 2)^2}
                            = \indeterminate{\frac{0}{0}}
                            = \limit[x \to 0^+] \frac{\cancel{x}(x + 2)}{\cancel{x}(x - 1)^3(x - 2)^2}
                            = \limit[x \to 0^+] \frac{x + 2}{(x - 1)^3(x - 2)^2}\\
                            &= \frac{0^+ + 2}{(0^+ - 1)^3(0^+ - 2)^2}
                            = \frac{2}{(-1)^3(-2)^2}
                            = \frac{2}{-4}
                            = -\frac{1}{2}
                    \end{split}
                \end{gather*}
            \item[\(x_0 = 1\)]
                \begin{gather*}
                    \limit[x \to 1^-] f(x)
                        = \limit[x \to 1^-] \frac{x(x + 2)}{x(x - 1)^3(x - 2)^2}
                        = \frac{1(1 + 2)}{1\parens{0^-}^3(-1)^2}
                        = -\infty\\
                    \limit[x \to 1^+] f(x)
                        = \limit[x \to 1^+] \frac{x(x + 2)}{x(x - 1)^3(x - 2)^2}
                        = \frac{1(1 + 2)}{1\parens{0^+}^3(-1)^2}
                        = +\infty
                \end{gather*}
            \item[\(x_0 = 2\)]
                \begin{gather*}
                    \limit[x \to 2^-] f(x)
                        = \limit[x \to 2^-] \frac{x(x + 2)}{x(x - 1)^3(x - 2)^2}
                        = \frac{2(2 + 2)}{2(1)^3\parens{0^-}^2}
                        = \infty\\
                    \limit[x \to 2^+] f(x)
                        = \limit[x \to 2^+] \frac{x(x + 2)}{x(x - 1)^3(x - 2)^2}
                        = \frac{2(2 + 2)}{2(1)^3\parens{0^+}^2}
                        = \infty
                \end{gather*}
        \end{description}
    \item
        \begin{equation*}
            f(x) = \frac{x(x + 1)}{(x - 1)^2(x + 2)}
        \end{equation*}
        \begin{description}
            \item[\(x_0 = 0\)]
                \begin{gather*}
                    \limit[x \to 0^-] f(x)
                        = \limit[x \to 0^-] \frac{x(x + 1)}{(x - 1)(x + 2)}
                        = \frac{0^- \cdot 1}{-1 \cdot 2}
                        = 0\\
                    \limit[x \to 0^+] f(x)
                        = \limit[x \to 0^+] \frac{x(x + 1)}{(x - 1)(x + 2)}
                        = \limit[x \to 0^+] \frac{0^+ \cdot 1}{-1 \cdot 2}
                        = 0
                \end{gather*}
            \item[\(x_0 = 1\)]
                \begin{gather*}
                    \limit[x \to 1^-] f(x)
                        = \limit[x \to 1^-] \frac{x(x + 1)}{(x - 1)(x + 2)}
                        = \frac{1 \cdot 2}{0^- \cdot 3}
                        = \frac{2}{0^-}
                        = -\infty\\
                    \limit[x \to 1^+] f(x)
                        = \limit[x \to 1^+] \frac{x(x + 1)}{(x - 1)(x + 2)}
                        = \frac{1 \cdot 2}{0^+ \cdot 3}
                        = \frac{2}{0^+}
                        = +\infty
                \end{gather*}
            \item[\(x_0 = 2\)]
                \begin{gather*}
                    \limit[x \to 2^-]
                        = \limit[x \to 2^-] \frac{x(x + 1)}{(x - 1)(x + 2)}
                        = \frac{2 \cdot 3}{1 \cdot 4}
                        = \frac{6}{4}
                        = \frac{2}{3}\\
                    \limit[x \to 2^+]
                        = \limit[x \to 2^+] \frac{x(x + 1)}{(x - 1)(x + 2)}
                        = \frac{2 \cdot 3}{1 \cdot 4}
                        = \frac{6}{4}
                        = \frac{3}{2}
                \end{gather*}
        \end{description}
    \item
        \begin{equation*}
            f(x) = \frac{x^2 - 4}{\abs{x - 2}x^3(x - 1)^2} = \frac{(x - 2)(x + 2)}{(x - 2)x^3(x - 1)^2}
        \end{equation*}
        \begin{description}
            \item[\(x_0 = 0\)]
                \begin{gather*}
                    \limit[x \to 0^-] f(x)
                        = \limit[x \to 0^-] \frac{\cancel{(x - 2)}(x + 2)}{-\cancel{(x - 2)}x^3(x - 1)^2}
                        = \frac{2}{-\parens{0^-}^3 \cdot (-1)^2}
                        = +\infty\\
                    \limit[x \to 0^+] f(x)
                        = \limit[x \to 0^+] \frac{\cancel{(x - 2)}(x + 2)}{-\cancel{(x - 2)}x^3(x - 1)^2}
                        = \frac{2}{-\parens{0^+}^3 \cdot \parens{-1}^2}
                        = -\infty
                \end{gather*}
            \item[\(x_0 = 1\)]
                \begin{gather*}
                    \limit[x \to 1^-] f(x)
                        = \limit[x \to 1^-] \frac{\cancel{(x - 2)}(x + 2)}{-\cancel{(x - 2)}x^3(x - 1)^2}
                        = \frac{3}{-1 \cdot \parens{0^-}^2}
                        = -\infty\\
                    \limit[x \to 1^+] f(x)
                        = \limit[x \to 1^+] \frac{\cancel{(x - 2)}(x + 2)}{-\cancel{(x - 2)}x^3(x - 1)^2}
                        = \frac{3}{-1 \cdot \parens{0^+}^2}
                        = -\infty
                \end{gather*}
            \item[\(x_0 = 2\)]
                \begin{gather*}
                    \limit[x \to 2^-] f(x)
                        = \limit[x \to 2^-] \frac{\cancel{\parens{x - 2}}\ps{x + 2}}{-\cancel{\pars{x - 2}}x^3\pars{x - 1}^2}
                        = \frac{4}{-2^3 \cdot 1^2}
                        = \frac{4}{-8}
                        = -\frac{1}{2}\\
                    \limit[x \to 2^+] f(x)
                        = \limit[x \to 2^+] \frac{\cancel{\parens{x - 2}}\ps{x + 2}}{\cancel{\pars{x - 2}}x^3\pars{x - 1}^2}
                        = \frac{4}{2^3 \cdot 1^2}
                        = \frac{4}{8}
                        = \frac{1}{2}
                \end{gather*}
        \end{description}
\end{enumerate}
