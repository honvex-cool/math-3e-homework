\subsubsection*{Zadanie~6.1.}
\begin{description}
    \item[doświadczenie losowe:] rzut jedną sześcienną kością
    \item[zbiór zdarzeń elementarnych:]
        \begin{gather*}
            \Omega = \set{1, 2, 3, 4, 5, 6}\\
            \card\Omega = 6
        \end{gather*}
\end{description}
\begin{enumerate}[label={\alph*)}]
    \item
        \begin{description}
            \item[zdarzenie:]
                \begin{gather*}
                    A = \set{4, 5, 6}\\
                    \card A = 3
                \end{gather*}
            \item[prawdopodobieństwo zdarzenia:]
                \begin{equation*}
                    P\pars{A}
                        = \frac{\card A}{\card\Omega}
                        = \frac{3}{6}
                        = \frac{1}{2}
                \end{equation*}
        \end{description}
    \item
        \begin{description}
            \item[zdarzenie:]
                \begin{gather*}
                    A = \set{1, 3, 5}\\
                    \card A = 3
                \end{gather*}
            \item[prawdopodobieństwo zdarzenia:]
                \begin{gather*}
                    P\pars{A}
                        = \frac{\card A}{\card\Omega}
                        = \frac{3}{6}
                        = \frac{1}{2}
                \end{gather*}
        \end{description}
\end{enumerate}
\subsubsection*{Zadanie~6.2.}
\begin{description}
    \item[doświadczenie losowe:] dwa rzuty sześcienną kością
    \item[zbiór zdarzeń elementarnych:]
        \begin{gather*}
            \Omega = \set{\seq{x, y} : x, y \in \set{1, 2, 3, 4, 5, 6}}\\
            \card\Omega = 6^2 = 36
        \end{gather*}
\end{description}
\begin{enumerate}[label={\alph*)}]
    \item
        \begin{description}
            \item[zdarzenie:]
                \begin{gather*}
                    A = \set{\seq{2, y} \in \Omega}\\
                    \card A = 6
                \end{gather*}
            \item[prawdopodobieństwo zdarzenia:]
                \begin{equation*}
                    P\pars{A}
                        = \frac{\card A}{\card\Omega}
                        = \frac{6}{36}
                        = \frac{1}{6}
                \end{equation*}
        \end{description}
    \item
        \begin{description}
            \item[zdarzenie:]
                \begin{gather*}
                    A = \set{\seq{x, x} \in \Omega}\\
                    \card A = 6
                \end{gather*}
            \item[prawdopodobieństwo zdarzenia:]
                \begin{equation*}
                    P\pars{A}
                        = \frac{\card A}{\card\Omega}
                        = \frac{6}{36}
                        = \frac{1}{6}
                \end{equation*}
        \end{description}
    \item
        \begin{description}
            \item[zdarzenie:]
                \begin{gather*}
                    A = \set{\seq{x, y} \in \Omega : x \neq y}\\
                    \card A = 6 \cdot 5 = 30
                \end{gather*}
            \item[prawdopodobieństwo zdarzenia:]
                \begin{equation*}
                    P\pars{A}
                        = \frac{\card A}{\card\Omega}
                        = \frac{30}{36}
                        = \frac{5}{6}
                \end{equation*}
        \end{description}
\end{enumerate}
\subsubsection*{Zadanie~6.3.}
\begin{description}
    \item[doświadczenie losowe:] pięć rzutów sześcienną kością
    \item[zbiór zdarzeń elementarnych:]
        \begin{gather*}
            \Omega = \set{\seq{a_1, a_2, a_3, a_4, a_5} : a_1, a_2, a_3, a_4, a_5 \in \set{1, 2, 3, 4, 5, 6}}\\
            \card\Omega = 6^5
        \end{gather*}
    \item[zdarzenie:]
        \begin{gather*}
            A = \set{\seq{a_1, a_2, a_3, a_4, a_5} \in \Omega : a_1, a_2, a_3, a_4, a_5 \text{ parami różne}}\\
            \card A = \frac{6!}{\pars{6 - 5}!} = 6!
        \end{gather*}
    \item[prawdopodobieństwo zdarzenia:]
        \begin{equation*}
            P\pars{A}
                = \frac{\card A}{\card\Omega}
                = \frac{6!}{6^5}
        \end{equation*}
\end{description}
\subsubsection*{Zadanie~4.}
\begin{description}
    \item[doświadczenie losowe:] rzut trzema sześciennymi kośćmi
    \item[zbiór zdarzeń elementarnych:]
        \begin{gather*}
            \Omega = \set{\seq{x, y, z} : x, y, z \in \set{1, 2, 3, 4, 5, 6}}\\
            \card\Omega = 6^3 = 216
        \end{gather*}
\end{description}
\begin{enumerate}[label={\alph*)}]
    \item 
        \begin{description}
            \item[zdarzenie:] jeśli może być tylko jedna jedynka, to na pozostałych dwóch kostkach muszą być przynajmniej dwójki, czyli suma wynosi przynajmniej \(5\). Zatem
                \begin{gather*}
                    A = \set{\seq{x, y, z} \in \Omega : \pars{x = 1 \nlxor y = 1 \nlxor z = 1} \nland x + y + z < 5} = \emptyset\\
                    \card A = 0
                \end{gather*}
            \item[prawdopodobieństwo zdarzenia:]
                \begin{equation*}
                    P\pars{A}
                        = \frac{\card A}{\card\Omega}
                        = \frac{0}{216}
                        = 0
                \end{equation*}
        \end{description}
    \item
        \begin{description}
            \item[zdarzenie:] Skoro dwie dwójki dają w~sumie \(4\), to na ostatniej kostce musi być \(1\), skoro nie może być \(2\):
                \begin{equation*}
                    A = \set{\seq{x, y, z} \in \Omega : \pars{x \neq 2 \nlxor y \neq 2 \nlxor z \neq 2} \nland x + y + z < 7} = \set{\seq{1, 2, 2}, \seq{2, 1, 2}, \set{2, 2, 1}}\\
                    \card A = 3
                \end{equation*}
            \item[prawdopodobieństwo zdarzenia:]
                \begin{equation*}
                    P\pars{A}
                        = \frac{\card A}{\card\Omega}
                        = \frac{3}{216}
                        = \frac{1}{72}
                \end{equation*}
        \end{description}
\end{enumerate}
\subsubsection*{Zadanie~6.5.}
\begin{description}
    \item[doświadczenie losowe:] rzut dwiema sześciennymi kośćmi
    \item[zbiór zdarzeń elementarnych:]
        \begin{gather*}
            \Omega = \set{\seq{x, y} \in \set{1, 2, 3, 4, 5, 6}}
            \card\Omega = 6^2 = 36
        \end{gather*}
\end{description}
\begin{enumerate}[label={\alph*)}]
    \item
        \begin{description}
            \item[zdarzenie:] aby suma była parzysta, na obydwu kostkach musi wypaść liczba parzysta lub na obydwu nieparzysta:
                \begin{equation*}
                    A = \set{\seq{x, y} \in \Omega : x \equiv y \pars{\bmod\ 2}}
                \end{equation*}
                Liczbę nieparzystą możemy wybrać na \(3\) sposoby, parzystą również na \(3\) sposoby. Zatem
                \begin{equation*}
                    \card A = 3^2 + 3^2 = 18
                \end{equation*}
            \item[prawdopodobieństwo zdarzenia:]
                \begin{equation*}
                    P\pars{A}
                        = \frac{\card A}{\card \Omega}
                        = \frac{18}{36}
                        = \frac{1}{2}
                \end{equation*}
        \end{description}
    \item
        \begin{description}
            \item[zdarzenie:] aby iloczyn był nieparzysty, obydwa czynniki muszą być nieparzyste:
                \begin{equation*}
                    A = \set{\seq{x, y} \in \Omega : x \equiv y \equiv 1 \pars{\bmod\ 2}}
                \end{equation*}
                Liczbę nieparzystą możemy wybrać na \(3\) sposoby. Zatem
                \begin{equation*}
                    \card A = 3^2 = 9
                \end{equation*}
            \item[prawdopodobieństwo zdarzenia:]
                \begin{equation*}
                    P\pars{A}
                        = \frac{\card A}{\card\Omega}
                        = \frac{9}{36}
                        = \frac{1}{4}
                \end{equation*}
        \end{description}
    \item
        \begin{description}
            \item[zdarzenie:] na \(4\) sposoby możemy wybrać środkową liczbę:
                \begin{gather*}
                    A = \set{\seq{x, y} \in \Omega : \abs{x - y} = 2} = \set{\seq{1, 3}, \seq{3, 1}, \seq{2, 4}, \seq{4, 2}, \seq{3, 5}, \seq{5, 3}, \seq{4, 6}, \seq{6, 4}}\\
                    \card A = 8
                \end{gather*}
            \item[prawdopodobieństwo zdarzenia:]
                \begin{equation*}
                    P\pars{A}
                        = \frac{\card A}{\card\Omega}
                        = \frac{8}{36}
                        = \frac{2}{9}
                \end{equation*}
        \end{description}
\end{enumerate}
\subsubsection*{Zadanie~6.6.}
\begin{description}
    \item[doświadczenie losowe:] dwa rzuty sześcienną kością
    \item[zbiór zdarzeń elementarnych:]
        \begin{gather*}
            \Omega = \set{\seq{x, y} : x, y \in \set{1, 2, 3, 4, 5, 6}}\\
            \card\Omega = 6^2 = 36
        \end{gather*}
\end{description}
\begin{enumerate}[label={\alph*)}]
    \item
        \begin{description}
            \item[zdarzenie:]
                \begin{gather*}
                    A = \set{\seq{x, y} \in \Omega : x + y = 5}
                        = \set{\seq{1, 4}, \seq{4, 1}, \seq{2, 3}, \seq{3, 2}}\\
                    \card A = 4
                \end{gather*}
            \item[prawdopodobieństwo zdarzenia:]
                \begin{equation*}
                    P\pars{A}
                        = \frac{\card A}{\card\Omega}
                        = \frac{4}{36}
                        = \frac{1}{9}
                \end{equation*}
        \end{description}
    \item
        \begin{description}
            \item[zdarzenie:]
                \begin{gather*}
                    A = \set{\seq{x, y} \in \Omega : x \cdot y = 12}
                        = \set{\seq{2, 6}, \seq{6, 2}, \seq{3, 4}, \seq{4, 3}}\\
                    \card A = 4
                \end{gather*}
            \item[prawdopodobieństwo zdarzenia:]
                \begin{equation*}
                    P\pars{A}
                        = \frac{\card A}{\card\Omega}
                        = \frac{4}{36}
                        = \frac{1}{9}
                \end{equation*}
        \end{description}
    \item
        \begin{description}
            \item[zdarzenie:]
                \begin{gather*}
                    A = \set{\seq{x, y} \in \Omega : x + y \leq 4}
                        = \set{\seq{x, y} \in \Omega : x \leq 2 \nland y \leq 2} \cup \set{\seq{1, 3}, \seq{3, 1}}\\
                    \card A = 2^2 + 2
                        = 6
                \end{gather*}
            \item[prawdopodobieństwo zdarzenia:]
                \begin{equation*}
                    P\pars{A}
                        = \frac{\card A}{\card\Omega}
                        = \frac{6}{36}
                        = \frac{1}{6}
                \end{equation*}
        \end{description}
    \item
        \begin{description}
            \item[zdarzenie:]
                \begin{gather*}
                    A = \set{\seq{x, x} \in \Omega : x \equiv 1 \pars{\bmod\ 2}}
                        = \set{\seq{1, 1}, \seq{3, 3}, \seq{5, 5}}\\
                    \card A = 3
                \end{gather*}
            \item[prawdopodobieństwo zdarzenia:]
                \begin{equation*}
                    P\pars{A}
                        = \frac{\card A}{\card\Omega}
                        = \frac{3}{36}
                        = \frac{1}{12}
                \end{equation*}
        \end{description}
\end{enumerate}
\subsubsection*{Zadanie~6.7.}
\begin{description}
    \item[doświadczenie losowe:] rzut \(n\) sześciennymi kośćmi
    \item[zbiór zdarzeń elementarnych:]
        \begin{gather*}
            \Omega = \set{\seq{a_1, a_2, \ldots, a_n} : a_1, a_2, \ldots, a_n \in \set{1, 2, 3, 4, 5, 6}}\\
            \card\Omega = 6^n
        \end{gather*}
\end{description}
\begin{enumerate}[label={\alph*)}]
    \item
        \begin{description}
            \item[zdarzenie:]
                \begin{gather*}
                    A = \set{\seq{a_1, a_2, \ldots, a_n} \in \Omega : \summation[k = 1][n]a_k = n}
                        = \set{\underbrace{\seq{1, 1, \ldots, 1}}_{n\text{ jedynek}}}\\
                    \card A = 1
                \end{gather*}
            \item[prawdopodobieństwo zdarzenia:]
                \begin{equation*}
                    P\pars{A}
                        = \frac{\card A}{\card\Omega}
                        = \frac{1}{6^n}
                \end{equation*}
        \end{description}
    \item
        \begin{description}
            \item[zdarzenie:]
                \begin{equation*}
                    A = \set{\seq{a_1, a_2, \ldots, a_n} \in \Omega : \summation[k = 1][n]a_k = n + 1}
                        = \set{\seq{2, 1, 1, \ldots, 1}, \seq{1, 2, 1, \ldots, 1}, \ldots, \seq{1, 1, 1, \ldots, 2}}
                \end{equation*}
                Ciąg wyników rzutów musi składać się z~jednej dwójki i~\(n - 1\) jedynek. Możliwych miejsc dla dwójki jest \(n\):
                \begin{equation*}
                    \card A = n
                \end{equation*}
            \item[prawdopodobieństwo zdarzenia:]
                \begin{equation*}
                    P\pars{A}
                        = \frac{\card A}{\card\Omega}
                        = \frac{n}{6^n}
                \end{equation*}
        \end{description}
    \item
        \begin{description}
            \item[zdarzenie:] suma nigdy nie będzie większa niż \(6n\), ponieważ na każdej z~\(n\) kostek może wypaść co najwyżej \(6\) oczek. Aby osiągnąć sumę \(6n - 1\), trzeba mieć \(n - 1\) szóstek i~jedną piątkę. Możliwych miejsc dla piątki jest \(n\). Zatem:
                \begin{gather*}
                    \begin{split}
                        A &= \set{\seq{a_1, a_2, \ldots, a_n} \in \Omega : \summation[k = 1][n]a_k = 6n - 1}\\
                            &= \set{\underbrace{\seq{6, 6, \ldots, 6}}_{n\text{ szóstek}}, \seq{5, 6, 6, \ldots, 6}, \seq{6, 5, 6, \ldots 6}, \ldots, \seq{6, 6, 6,\ldots, 5}}
                    \end{split}\\
                    \card A = n + 1
                \end{gather*}
            \item[prawdopodobieństwo zdarzenia:]
                \begin{equation*}
                    P\pars{A}
                        = \frac{\card A}{\card \Omega}
                        = \frac{n + 1}{6^n}
                \end{equation*}
        \end{description}
\end{enumerate}
\subsubsection*{Zadanie~6.8.}
\begin{description}
    \item[doświadczenie losowe:] rzut \emph{dwoma} sześciennymi kośćmi (wynik na pierwszej jest ustalony, zastanawiamy się jakie jest prawdopodobieństwo, że suma na pozostałych dwóch jest większa od \(4\))
    \item[zbiór zdarzeń elementarnych:]
        \begin{gather*}
            \Omega = \set{\seq{x, y} : x, y \in \set{1, 2, 3, 4, 5, 6}}
        \end{gather*}
    \item[zdarzenie:]
        \begin{gather*}
            A = \set{\seq{x, y} \in \Omega : x + y > 4}\\
            A' = \set{\seq{x, y} \in \Omega : x + y \leq 4}
                = \set{\seq{x, y} \in \Omega : x \leq 2 \nland y \leq 2} \cup \set{\seq{1, 3}, \seq{3, 1}}\\
            \card A' = 2^2 + 2 = 6\\
            \card A = \card\Omega - \card A'
                = 36 - 6
                = 30
        \end{gather*}
    \item[prawdopodobieństwo zdarzenia:]
        \begin{equation*}
            P\pars{A}
                = \frac{\card A}{\card\Omega}
                = \frac{30}{36}
                = \frac{5}{6}
        \end{equation*}
\end{description}
