\subsection*{Zadania na ekstrema funkcji różniczkowalnych}
\subsubsection*{Zadanie~8.17.}
\begin{equation*}
    f\pars{x} = x^3 - 6x^2 + bx + c \qquad x \in \real
\end{equation*}
Skoro wykres tej funkcji przechodzi przez punkt \(P = \pars{2; -2}\), to znaczy, że
\begin{gather*}
    f\pars{2} = -2\\
    2^3 - 6 \cdot 2^2 + 2b + c = -2\\
    2b + c = 14
\end{gather*}
Natomiast fakt, że współczynnik kierunkowy stycznej do wykresu tej funkcji w~punkcie \(P\) wynosi \(-3\) oznacza, że pochodna funkcji \(f\) przyjmuje wartość \(-3\) dla argumentu \(2\):
\begin{gather*}
    f'\pars{x} = 3x^2 - 12x + b\\
    3 \cdot 2^2 - 12 \cdot 2 + b = -3\\
    b = 9
\end{gather*}
Zatem podstawiając do otrzymanego wcześniej równania otrzymujemy \(c = -4\).
\begin{equation*}
    f\pars{x} = x^3 - 6x^2 + 9x - 4
\end{equation*}
Aby wyznaczyć największą i~najmniejszą wartość funkcji w~przedziale \(\closed{-1}{3}\) skorzystamy z~pochodnej. Znajdziemy jej miejsca zerowe, aby zbadać ekstrema funkcji \(f\):
\begin{gather*}
    f'\pars{x} = 3x^2 - 12x + 9\\
    3x^2 - 12x + 9 = 0\\
    x^2 - 4x + 3 = 0\\
    \pars{x - 1}\pars{x - 3} = 0\\
    \upparabola{1}{3}\\
    \tag{\(1\)} \forall x \in \leftclosed{-1}{1}\colon f'\pars{x} > 0 \label{2020_10_23:8_17:first_increase}\\
    \tag{\(2\)} f'\pars{1} = 0 \label{2020_10_23:8_17:first_zero}\\
    \tag{\(3\)} \forall x \in \open{1}{3}\colon f'\pars{x} < 0 \label{2020_10_23:8_17:decrease}\\
    \tag{\(4\)} f'\pars{3} = 0 \label{2020_10_23:8_17:second_zero}\\
    \tag{\(5\)} \forall x \in \open{3}{+\infty}\colon f'\pars{x} > 0 \label{2020_10_23:8_17:second_increase}
\end{gather*}
Zatem:
\begin{description}
    \item \(\mbox{(\ref{2020_10_23:8_17:first_increase})} \implies\) funkcja \(f\) jest rosnąca w~przedziale \(\open{-1}{1}\)
    \item \(\mbox{(\ref{2020_10_23:8_17:first_increase})} \land \mbox{(\ref{2020_10_23:8_17:first_zero})} \land \mbox{(\ref{2020_10_23:8_17:decrease})} \implies\) funkcja \(f\) ma w~punkcie \(x = 1\) maksimum lokalne:
        \begin{equation*}
            f\pars{1} = 1 - 6 + 9 - 4 = 0
        \end{equation*}
    \item \(\mbox{(\ref{2020_10_23:8_17:decrease})} \implies\) funkcja \(f\) jest malejąca w~przedziale \(\open{1}{3}\)
    \item \(\mbox{(\ref{2020_10_23:8_17:decrease})} \land \mbox{(\ref{2020_10_23:8_17:second_zero})} \land \mbox{(\ref{2020_10_23:8_17:second_increase})} \implies\) funkcja \(f\) ma w~punkcie \(x = 3\) minimum lokalne:
        \begin{equation*}
            f\pars{3} = 3^3 - 6 \cdot 3^2 + 9 \cdot 3 - 4
                = 27 - 54 + 27 - 4
                = -4
        \end{equation*}
    \item \(\mbox{(\ref{2020_10_23:8_17:second_increase})} \implies\) funkcja \(f\) jest rosnąca w~przedziale \(\open{3}{+\infty}\)
\end{description}
Skoro funkcja najpierw rośnie w~przedziale \(\open{-1}{1}\), dla \(x = 1\) osiąga maksimum lokalne, a~nastepnie maleje w~przedziale \(\open{1}{3}\), to na pewno wartością największą w~przedziale \(\closed{-1}{3}\) jest wartość maksimum:
\begin{equation*}
    f_\p{max} = f\pars{1} = 0
\end{equation*}
Natomiast kandydatami na argumenty, dla których funkcja przyjmuje wartość najmniejszą, są \(x = -1\) (które rozważane w~tym przedziale jest minimum lokalnym) i~\(x = 3\) (które jest minimum lokalnym). Wystarczy zatem porównać wartości funkcji dla tych dwóch argumentów:
\begin{gather*}
    f\pars{-1} = -1 - 6 - 9 - 4 = -20\\
    f\pars{3} = -4
\end{gather*}
Zatem w~przedziale \(\closed{-1}{3}\)
\begin{equation*}
    f_\p{min} = f\pars{-1} = -20
\end{equation*}
\subsubsection*{Zadanie~8.19.}
\begin{mathfigure*}
    \def\rt{\fpeval{sqrt(3)}}
    \coordinate (A) at (0, 0);
    \coordinate (B) at (2, 0);
    \coordinate (C) at (0, 2*\rt/3);
    \coordinate (D) at (0.5, \rt/2);
    \draw (A) -- node[right]{\(1\)} (D);
    \draw (A) node[below left]{\(A\)} -- (B) node[below right]{\(B\)} -- node[midway, above right]{\(x\)} (C) node[above]{\(C\)} -- cycle;
    \fillpoint*{D}[\(D\)][above right];
    \drawrightangle[angle radius=0.25cm]{A--D--B};
    \drawangle[ForestGreen, angle radius=0.25cm]{A--C--D};
    \drawangle[ForestGreen, angle radius=0.25cm]{B--A--D};
\end{mathfigure*}
Chcemy zapisać długość przeciwprostokątnej jako funkcję \(y\pars{x}\). Z~twierdzenia Pitagorasa wiemy, że \(AB = \sqrt{x^2 + 1}\). Zauważmy, że \(\mangle{BCA} = \mangle{DAB}\) i~\(\mangle{CAB} = \mangle{ADB} = 90\degree\). Zatem \(\triangle{DAB} \sim \triangle{ABC}\). To oznacza, że
\begin{gather*}
    \frac{BD}{AD} = \frac{AB}{AC}\\
    \frac{x}{1} = \frac{\sqrt{x^2 + 1}}{AC}\\
    AC = \frac{\sqrt{x^2 + 1}}{x}
\end{gather*}
Zatem, ponownie stosując twierdzenie Pitagorasa, mamy
\begin{gather*}
    x \in \open{0}{+\infty}\\
    \begin{split}
        y\pars{x} = \sqrt{AB^2 + AC^2}
            &= \sqrt{\pars{\sqrt{x^2 + 1}}^2 + \pars{\frac{\sqrt{x^2 + 1}}{x}}^2}
            = \sqrt{x^2 + 1 + \frac{x^2 + 1}{x^2}}
            = \sqrt{\frac{x^4 + 2x^2 + 1}{x^2}}\\
            &= \sqrt{\frac{\pars{x^2 + 1}^2}{x^2}}
            = \frac{x^2 + 1}{x}
            = x + \frac{1}{x}
    \end{split}
\end{gather*}
Obliczmy pochodną tej funkcji:
\begin{gather*}
    y'\pars{x} = 1 - \frac{1}{x^2}\\
    1 - \frac{1}{x^2} = 0\\
    \frac{x^2 - 1}{x^2} = 0\\
    x = 1 \wlor x = -1
\end{gather*}
Mianownik jest zawsze nieujemny, więc znak pochodnej zależy jedynie od licznika:
\begin{equation*}
    \upparabola{-1}{1}
\end{equation*}
Interesuje nas tylko przedział \(\open{0}{+\infty}\). Zbadajmy w~nim zachowanie pochodnej:
\begin{gather*}
    \tag{\(1\)} \forall x \in \open{0}{1}\colon y'\pars{x} < 0 \label{2020_10_23:8_19:decrease}\\
    \tag{\(2\)} y'\pars{1} = 0 \label{2020_10_23:8_19:zero}\\
    \tag{\(3\)} \forall x \in \open{1}{+\infty}\colon y'\pars{x} < 0 \label{2020_10_23:8_19:increase}
\end{gather*}
Co z~tego wynika:
\begin{description}
    \item \(\mbox{(\ref{2020_10_23:8_19:decrease})} \implies\) funkcja \(y\) jest malejąca w~przedziale \(\open{0}{1}\)
    \item \(\mbox{(\ref{2020_10_23:8_19:increase})} \implies\) funkcja \(y\) jest rosnąca w~przedziale \(\open{1}{+\infty}\)
    \item \(\mbox{(\ref{2020_10_23:8_19:decrease})} \land \mbox{(\ref{2020_10_23:8_19:zero})} \land \mbox{(\ref{2020_10_23:8_19:increase})} \implies\) funkcja \(y\) ma minimum (globalne!) w~punkcie \(x = 1\):
        \begin{equation*}
            y\pars{1} = 1 + \frac{1}{1} = 2
        \end{equation*}
\end{description}
Zatem najmniejszą możliwą wartością funkcji \(y\pars{x}\) jest \(y\pars{1} = 2\)
