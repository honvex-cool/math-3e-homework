\subsection*{Zestaw XXV --- Rachunek różniczkowy (zadania otwarte)}
\subsubsection*{Zadanie~1.}
\begin{gather*}
    f\pars{x} = \frac{x^2}{2x - 2} \qquad x \in \real \setminus \set{1}\\
    f'\pars{x}
        = \frac{\pars{x^2}'\pars{2x - 2} - x^2\pars{2x - 2}'}{\pars{2x - 2}^2}
        = \frac{2x\pars{2x - 2} - 2x^2}{\pars{2x - 2}^2}
        = \frac{4x^2 - 4x - 2x^2}{\pars{2x - 2}^2}
        = \frac{2x^2 - 4x}{\pars{2x - 2}^2}\\
    f\pars{x} = f'\pars{x}\\
    \frac{x^2}{2x - 2} = \frac{2x^2 - 4x}{\pars{2x - 2}^2}\\
    x^2 = \frac{2x^2 - 4x}{2x - 2}\\
    2x^3 - 2x^2 = 2x^2 - 4x\\
    2x^3 - 4x^2 + 4x = 0\\
    2x\overbrace{\pars{x^2 - 2x + 2}}^{\Delta = 4 - 4 \cdot 2 = -4 < 0} = 0
\end{gather*}
Zatem jedynym rozwiązaniem takiego równania jest \(x = 0\), ponieważ drugi czynnik nie ma pierwiastków.
\qed
\subsubsection*{Zadanie~2.}
\begin{gather*}
    m \geq \frac{4}{3}\\
    f\pars{x} = -x^3 + 2x^2 - mx - 5 \qquad x \in \real
\end{gather*}
Obliczmy pochodną tej funkcji:
\begin{equation*}
    f'\pars{x} = -3x^2 + 4x - m
\end{equation*}
Obliczmy \(\Delta\) takiej funkcji kwadratowej:
\begin{equation*}
    \Delta = 4^2 - 4 \cdot \pars{-3} \cdot \pars{-m}
        = 16 - 12m
\end{equation*}
Skoro \(m \geq \frac{4}{3}\), to \(\Delta = 16 - 12m \leq 16 - 12 \cdot \frac{4}{3} = 0\). Zatem pochodna ma co najwyżej jeden pierwiastek. Współczynnik kierujący jest ujemny, więc ramiona paraboli są skierowane w~stronę malejących współrzędnych \(y\). Oznacza to, że pochodna jest zawsze niedodatnia. Z~tego wszystkiego wyniak, że funkcja \(f\) jest malejąca w~\(\real\) (w~jedynym miejscu zerowym pochodnej funkcja ma punkt przegięcia, co nie wpływa jednak na monotoniczność).
\qed
\subsubsection*{Zadanie~3.}
\begin{equation*}
    f\pars{x} = x^3 - 2x^2 + 8x - 15 \qquad x \in \real
\end{equation*}
Zauważmy, że ta funkcja jest ciągła jako suma funkcji ciągłych. Pokażmy najpierw, że ma miejsce zerowe:
\begin{gather*}
    f\pars{0} = -15 < 0\\
    f\pars{2} = 8 - 8 + 16 - 15 = 1 > 0
\end{gather*}
Skoro funkcja przyjmuje zarówno wartości dodatnie i~ujemne oraz jest ciągła, to na mocy twierdzenia Darboux przyjmuje również wartość \(0\). Teraz obliczmy pochodną funkcji \(f\):
\begin{equation*}
    f'\pars{x} = 3x^2 - 4x + 8
\end{equation*}
Obliczmy \(\Delta\) tej funkcji kwadratowej:
\begin{equation*}
    \Delta = \pars{-4}^2 - 4 \cdot 3 \cdot 8 = 16 - 96 = -80 < 0
\end{equation*}
Zatem ta funkcja kwadratowa nie ma miejsc zerowych, a~współczynnik kierujący jest dodatni. Ramiona paraboli są więc skierowane w~stronę rosnących współrzędnych \(y\), czyli \(f'\) przyjmuje tylko wartości dodatnie. Oznacza to, że funkcja \(f\) jest rosnąca w~\(\real\). Zatem, ponieważ jest ciągła, to nie może przyjmować więcej niż jednego miejsca zerowego, czyli przyjmuje dokładnie jedno.
\qed
\subsubsection*{Zadanie~4.}
\begin{equation*}
    y\pars{x} = ax^2 \qquad x \in \real
\end{equation*}
Obliczmy pochodną tej funkcji, aby uzyskać współczynnik kierunkowy prostej stycznej do paraboli:
\begin{equation*}
    y'\pars{x} = 2ax
\end{equation*}
Zatem współczynnik kierunkowy prostej stycznej do paraboli w~punkcie \(\pars{x_0; y_0}\) wynosi
\begin{equation*}
    y'\pars{x_0} = 2ax_0
\end{equation*}
Prosta styczna do paraboli w~tym punkcie jest więc funkcją liniową o~równaniu
\begin{equation*}
    t\pars{x} = 2ax_0\pars{x - x_0} + y\pars{x_0}
\end{equation*}
Punkt przecięcia tej prostej z~osią \(Oy\) to \(0; t\pars{0}\):
\begin{equation*}
    t\pars{0}
        = 2ax_0\pars{0 - x_0} + y\pars{x_0}
        = -2ax_0^2 + y\pars{x_0}
        = -2y\pars{x_0} + y\pars{x_0}
        = -y\pars{x_0}
        = y_0
\end{equation*}
Zatem punkt przecięcia tej prostej z~osią \(Oy\) to \(\pars{0; -y_0}\).
\qed
\subsubsection*{Zadanie~5.}
\begin{equation*}
    f\pars{x} = ax^3 + bx + 10 \qquad x \in \real
\end{equation*}
Wiemy, że
\begin{gather*}
    f\pars{1} = 8\\
    a + b + 10 = 8\\
    a + b = -2
\end{gather*}
Skoro funkcja \(f\) osiąga ekstremum dla \(x = 1\), to pochodna musi w~tym punkcie mieć miejsce zerowe, w~którym zmienia znak. Obliczmy zatem tę pochodną:
\begin{equation*}
    f'\pars{x}
        = 3ax^2 + b
\end{equation*}
Podstawiając \(x = 1\), otrzymujemy
\begin{equation*}
    3a + b = 0
\end{equation*}
Odejmując stronami uzyskane równania łączące \(a\) i~\(b\), otrzymujemy
\begin{gather*}
    2a = 2\\
    \begin{cases}
        a = 1\\
        b = -3
    \end{cases}
\end{gather*}
Musimy jeszcze sprawdzić, czy to faktycznie jest rozwiązanie, czyli czy pochodna w~punkcie \(x = 1\) zmienia znak:
\begin{gather*}
    f'\pars{1} = 3x^2 - 3
        = 3\pars{x + 1}\pars{x - 1}\\
    \upparabola{-1}{1}
\end{gather*}
Rzeczywiście, pochodna jest ujemna w~przedziale \(\open{-1}{1}\) i~dodatnia w~przedziale \(\open{1}{+\infty}\), zatem faktycznie w~punkcie \(x = 1\), dla którego pochodna ma wartość \(0\), faktycznie jest ekstremum. Zatem para
\begin{equation*}
    \begin{cases}
        a = 1\\
        b = -1
    \end{cases}
\end{equation*}
jest rozwiązaniem.
\subsubsection*{Zadanie~6.}
\begin{equation*}
    f\pars{x} = \frac{2x}{x^2 + 1} + 2 \qquad x \in \real
\end{equation*}
Styczna do wykresu funkcji \(f\) jest równoległa do osi \(Ox\) wtedy, gdy współczynnik takiej prostej jest równy \(0\), czyli gdy pochodna przyjmuje wartość \(0\). Obliczmy więc tę pochodną:
\begin{equation*}
    f'\pars{x}
        = \frac{\pars{2x}'\pars{x^2 + 1} - 2x\pars{x^2 + 1}'}{\pars{x^2 + 1}^2}
        = \frac{2x^2 + 2 - 4x^2}{\pars{x^2 + 1}^2}
        = \frac{2 - 2x^2}{\pars{x^2 + 1}^2}
        = \frac{2\pars{1 - x}\pars{1 + x}}{\pars{x^2 + 1}^2}
\end{equation*}
Aby znaleźć współrzędne \(x\) takich punktów, rozwiązujemy równanie
\begin{gather*}
    f'\pars{x} = 0\\
    \frac{2\pars{1 - x}\pars{1 + x}}{\pars{x^2 + 1}^2} = 0\\
    \pars{1 - x}\pars{1 + x} = 0\\
    x = 1 \wlor x = -1
\end{gather*}
Podstawiając do funkcji \(f\), otrzymujemy współrzędne \(y\) tych punktów:
\begin{gather*}
    f\pars{1} = \frac{2}{1 + 1} + 2
        = \frac{2}{2} + 2
        = 3\\
    f\pars{1} = \frac{-2}{1 + 1} + 2
        = \frac{-2}{2} + 2
        = -1 + 2
        = 1
\end{gather*}
Zatem punkty, w~których styczna do wykresu funkcji \(f\pars{x}\) jest równoległa do osi \(Ox\), to:
\begin{gather*}
    \pars{1; 3}\\
    \pars{-1; 1}
\end{gather*}
\subsubsection*{Zadanie~7.}
\begin{equation*}
    f\pars{x} = \frac{x^3}{3} + \frac{ax^2}{2} \qquad x \in \real
\end{equation*}
Aby zbadać ekstrema, obliczmy pochodną tej funkcji:
\begin{equation*}
    f'\pars{x} = x^2 + ax = x\pars{x + a}
\end{equation*}
Gdy \(a = 0\), to pochodna ma postać \(x^2\), czyli jest zawsze nieujemna, a~zatem funkcja \(f\) nie ma ekstremów. Natomiast gdy \(a \neq 0\), to pochodna jest funkcją kwadratową o~dwóch miejscach zerowych, w~których zmienia znak:
\begin{proofcases}
    \item \(a < 0\)
        \begin{equation*}
            \upparabola{0}{-a}
        \end{equation*}
        Wtedy
            \begin{itemize}
                \item pochodna jest dodatnia w~przedziale \(\open{-\infty}{0}\), dla \(x = 0\) przyjmuje wartość \(0\), a~w~przedziale \(\open{0}{-a}\) jest ujemna, zatem funkcja \(f\) jest rosnąca w~przedziale \(\open{-\infty}{0}\) i~malejąca w~przedziale \(\open{0}{-a}\), czyli dla \(x = 0\) osiąga maksimum lokalne:
                    \begin{equation*}
                        f\pars{0} = 0
                    \end{equation*}
                \item pochodna jest ujemna w~przedziale \(\open{0}{-a}\), dla \(x = -a\) przyjmuje wartość \(0\), a~w~przedziale \(\open{-a}{+\infty}\) jest dodatnia, zatem funkcja \(f\) jest malejąca w~przedziale \(\open{0}{-a}\) i~rosnąca w~przedziale \(\open{-a}{+\infty}\), czyli dla \(x = -a\) osiąga minimum lokalne:
                \begin{equation*}
                    f\pars{-a} = \frac{-a^3}{3} + \frac{a^3}{2} = \frac{a^3}{6}
                \end{equation*}
            \end{itemize}
    \item \(a > 0\)
    \begin{equation*}
        \upparabola{-a}{0}
    \end{equation*}
    Wtedy
        \begin{itemize}
            \item pochodna jest dodatnia w~przedziale \(\open{-\infty}{-a}\), dla \(x = -a\) przyjmuje wartość \(0\), a~w~przedziale \(\open{-a}{0}\) jest ujemna, zatem funkcja \(f\) jest rosnąca w~przedziale \(\open{-\infty}{-a}\) i~malejąca w~przedziale \(\open{-a}{0}\), czyli dla \(x = -a\) osiąga maksimum lokalne:
                \begin{equation*}
                    f\pars{-a} = \frac{-a^3}{3} + \frac{a^3}{2} = \frac{a^3}{6}
                \end{equation*}
            \item pochodna jest ujemna w~przedziale \(\open{-a}{0}\), dla \(x = 0\) przyjmuje wartość \(0\), a~w~przedziale \(\open{0}{+\infty}\) jest dodatnia, zatem funkcja \(f\) jest malejąca w~przedziale \(\open{-a}{0}\) i~rosnąca w~przedziale \(\open{0}{+\infty}\), czyli dla \(x = 0\) osiąga minimum lokalne:
            \begin{equation*}
                f\pars{0} = 0                
            \end{equation*}
        \end{itemize}
\end{proofcases}
\subsubsection*{Zadanie~8.}
Aby zminimalizować odległość możemy równoważnie zminimalizować jej kwadrat. Zdefiniujmy funkcję kwadratu odległości punktu leżącego na paraboli od punktu \(\pars{3; 2}\) w~zależności od współrzędnej \(x\) tego punktu na paraboli:
\begin{equation*}
    \begin{split}
        D\pars{x}
            &= \pars{x - 3}^2 + \pars{3x^2 - 5x + 6 - 2}^2
            = \pars{x - 3}^2 + \pars{3x^2 - 5x + 4}^2\\
            &= x^2 - 6x + 9 + 9x^4 + 25x^2 + 16 + 24x^2 - 30x^3 - 40x
            = 9x^4 - 30x^3 + 50x^2 - 46x + 25 \qquad x \in \real
    \end{split}
\end{equation*}
Obliczmy pochodną tej funkcji, aby zbadać ekstrema:
\begin{equation*}
    D'\pars{x}
        = 36x^3 - 90x^2 + 100x - 46
        = \pars{x - 1}\pars{36x^2 - 54x + 46}
\end{equation*}
Zauważmy, że \(\Delta\) drugiego czynnika jest ujemna:
\begin{equation*}
    \Delta
        = \pars{-54}^2 - 4 \cdot 36 \cdot 46
        = 2916 - 6624 < 0
\end{equation*}
Zatem drugi czynnik nie ma pierwiastków, a~ponieważ współczynnik przy \(x^2\) jest dodatni, to cały czynnik \(36x^2 - 54x + 46\) jest zawsze dodatni. Wykres znaku pochodnej wygląda więc następująco:
\begin{equation*}
    \begin{tikzpicture}
        \drawvec (-2, 0) -- (2, 0) node[below]{\(x\)};
        \draw (-1.5, -1.5) -- (1.5, 1.5);
        \fillpoint*{0, 0}[\(1\)][below right];
    \end{tikzpicture}
\end{equation*}
W~przedziale \(\open{-\infty}{1}\) pochodna jest ujemna, dla \(x = 1\) przyjmuje wartość \(0\), a~w~przedziale \(\open{1}{+\infty}\) jest dodatnia. Oznacza to, że funkcja \(D\) jest malejąca w~przedziale \(\open{-\infty}{1}\) i~rosnąca w~przedziale \(\open{1}{+\infty}\), więc dla \(x = 1\) przyjmuje globalną wartość najmniejszą:
\begin{equation*}
    D_\p{min} = D\pars{1} = 9 - 30 + 50 - 46 + 25 = 8
\end{equation*}
Czyli najmniejszy kwadrat odległości punktu leżącego na paraboli od punktu \(\pars{3; 2}\) wynosi \(8\). Sama odległość wynosi więc \(2\sqrt{2}\). Pierwsza współrzędna punktu wynosi \(1\), a~drugą otrzymujemy przez podstawienie do równania paraboli:
\begin{equation*}
    y = 3x^2 - 5x + 6 = 3 - 5 + 6 = 4
\end{equation*}
Zatem najbliższy do punktu \(\pars{3; 1}\) punkt leżący na paraboli \(y = 3x^2 - 5x + 6\) ma współrzędne
\begin{equation*}
    \pars{1; 4}
\end{equation*}
\subsubsection*{Zadanie~9.}
\begin{gather*}
    m \in \real\\
    f\pars{x} = \frac{4}{3}x^3 - 3\pars{m + 1}x^2 + m^2x - 3m + 1 \qquad x \in \real
\end{gather*}
Aby zbadać ekstrema tej funkcji, obliczmy jej pochodną:
\begin{equation*}
    f'\pars{x} = 4x^2 - 6\pars{m + 1}x + m^2
\end{equation*}
Pochodna jest funkcją kwadratową o~dodatnim współczynniku przy \(x^2\). Aby funkcja \(f\) nie posiadała maksimum, wykres pochodnej nie może wyglądać tak:
\begin{equation*}
    \upparabola{x_1}{x_2}
\end{equation*}
ponieważ wtedy w~przedziale \(\open{-\infty}{x_1}\) pochodna byłaby dodatnia, dla \(x = x_1\) przyjmowałaby wartość \(0\), a~w~przedziale \(\open{x_1}{x_2}\) byłaby ujemna, czyli funkcja \(f\) byłaby rosnąca w~przedziale \(\open{-\infty}{x_1}\) i~malejąca w~przedziale \(\open{x_1}{x_2}\), więc dla \(x = x_1\) przyjmowałaby maksimum lokalne. Zatem pochodna nie może mieć dwóch miejsc zerowych, czyli jej \(\Delta\) musi być niedodatnia:
\begin{gather*}
    \Delta \leq 0\\
    \pars{-6\pars{m + 1}}^2 - 4 \cdot 4m^2 \leq 0\\
    36m^2 + 72m + 36 - 16m^2 \leq 0\\
    20m^2 + 72m + 36 \leq 0\\
    5m^2 + 18m + 9 \leq 0\\
    \Delta_m = 18^2 - 4 \cdot 5 \cdot 9
        = 324 - 180
        = 144\\
    \sqrt{\Delta} = 12\\
    m_1 = \frac{-18 - \sqrt{\Delta}}{2 \cdot 5}
        = \frac{-18 - 12}{10}
        = -3\\
    m_2 = \frac{-18 + \sqrt{\Delta}}{2 \cdot 5}
        = \frac{-18 + 12}{10}
        = -\frac{3}{5}\\
    \upparabola{-3}{-\frac{3}{5}}[\(m\)]
\end{gather*}
Zatem ostatecznie otrzymujemy
\begin{equation*}
    m \in \closed{-3}{-\frac{3}{5}}
\end{equation*}
\subsubsection*{Zadanie~10.}
\begin{equation*}
    4\liter = 4\dm^3
\end{equation*}
W~dalszej części rozwiązania tego zadania będę wszystkie wymiary podawał odpowiednio w~\(\dm\), \(\dm^2\) i~\(\dm^3\). Oznaczmy przez \(x\) długość krawędzi kwadratu znajdującego się w~podstawie prostopadłościanu, a~przez \(h\) wysokość tego prostopadłościanu. Objętość prostopadłościanu wyraża się następująco:
\begin{gather*}
    V = x^2h\\
    4 = x^2h\\
    h = \frac{4}{x^2}
\end{gather*}
Zdefiniujmy funkcję pola powierzchni tego pojemnika (bez pokrywki) w~zależności od \(x\):
\begin{equation*}
    S\pars{x}
        = x^2 + 4hx = x^2 + 4x \cdot \frac{4}{x^2}
        = x^2 + \frac{16}{x}
        = \frac{x^3 + 16}{x} \qquad x \in \open{0}{+\infty}
\end{equation*}
Obliczmy pochodną tej funkcji:
\begin{equation*}
    S'\pars{x}
        = \frac{\pars{x^3 + 16}'x - \pars{x^3 + 16}\pars{x}'}{x^2}
        = \frac{3x^3 - x^3 - 16}{x^2}
        = \frac{2x^3 - 16}{x^2}
        = \frac{2\pars{x^3 - 8}}{x^2}
        = \frac{2\pars{x - 2}\overbrace{\pars{x^2 + 2x + 4}}^{\Delta = 2^2 - 4 \cdot 1 \cdot 4 = -12 < 0}}{x^2}
\end{equation*}
Widzimy, że wyrażenia \(x^2\) oraz \(x^2 + 2x + 4\) są zawsze dodatnie, zatem znak pochodnej zależy tylko od znaku wyrażenia \(x - 2\). Wykres znaku pochodnej wygląda więc tak:
\begin{equation*}
    \begin{tikzpicture}
        \draw (-2, 0) -- (2, 0) node[below]{\(x\)};
        \draw (-1.5, -1.5) -- (1.5, 1.5);
        \fillpoint*{0, 0}[\(2\)][below right];
    \end{tikzpicture}
\end{equation*}
Interesuje nas tylko przedział \(\open{0}{+\infty}\). Pochodna jest ujemna w~przedziale \(\open{0}{2}\), dla \(x = 2\) przyjmuje wartość \(0\), a~w~przedziale \(\open{2}{+\infty}\) jest dodatnia. Oznacza to, że funkcja \(S\) jest malejąca w~przedziale \(\open{0}{2}\) i~rosnąca w~przedziale \(\open{2}{+\infty}\), więc dla \(x = 2\) osiąga globalną wartość największą:
\begin{equation*}
    S_\p{max} = S\pars{2} = \frac{8 + 16}{4} = 6
\end{equation*}
Oznacza to, że wymiary pojemnika wynoszą
\begin{equation*}
    2\dm \times 2\dm \times 1\dm
\end{equation*}
\subsubsection*{Zadanie~11.}
\begin{mathfigure*}
    \coordinate (A) at (-1, 0);
    \coordinate (B) at (0, 1);
    \coordinate (C) at (4, 2);
    \coordinate (D) at (2.5, 3.5);
    \drawcoordsystem{-3, -3}{13, 7};
    \drawrightangle[angle radius=0.4cm]{B--D--C};
    \draw[domain=0:13, thick, ForestGreen, samples=200] plot (\x, {sqrt(\x)}) node[above left]{\(y = \sqrt{x}\)};
    \draw[domain=-3:6, dashed] plot (\x, {\x + 1}) node[below right]{\(y = x + 1\)};
    \draw[RoyalBlue, very thick] (A) -- (B) -- (C) -- cycle;
    \draw[dotted] (C) -- (D);
    \fillpoint*{A}[\(A\)][above left];
    \fillpoint*{B}[\(B\)][above left];
    \fillpoint*{C}[\(C\)][below];
\end{mathfigure*}
\begin{equation*}
    C = \pars{x; \sqrt{x}}
\end{equation*}
Zauważmy, że prosta zawierająca bok \(AB\) ma równanie
\begin{gather*}
    y = x + 1\\
    x - y + 1 = 0
\end{gather*}
Zatem aby obliczyć wysokość \(\triangle{ABC}\) opuszczoną z~wierzchołka \(C\), musimy wyznaczyć odległość punktu \(C\) od tej prostej, czyli
\begin{equation*}
    h
        = \frac{\abs{x - \sqrt{x} + 1}}{\sqrt{1^2 + \pars{-1}^2}}
        = \frac{\abs{x - \sqrt{x} + 1}}{\sqrt{2}}
\end{equation*}
Zauważmy, że
\begin{equation*}
    x - \sqrt{x} + 1
        = x - 2\sqrt{x} + 1 + \sqrt{x}
        = \overbrace{\pars{\sqrt{x} - 1}^2}^{\geq 0} + \overbrace{\sqrt{x}}^{\geq 0}
        \geq 0
\end{equation*}
Zatem we wzorze na odległość punktu od prostej możemy opuścić moduł:
\begin{equation*}
    h = \frac{x - \sqrt{x} + 1}{\sqrt{2}}
\end{equation*}
Odcinek \(AB\) ma długość \(\sqrt{2}\). Zdefiniujmy zatem funkcję pola tego trójkąta w~zależności od \(x\):
\begin{equation*}
    S\pars{x}
        = \frac{AB \cdot h}{2}
        = \frac{\sqrt{2} \cdot \frac{x - \sqrt{x} + 1}{\sqrt{2}}}{2}
        = \frac{1}{2}\pars{x - \sqrt{x} + 1} \qquad x \in \leftclosed{0}{+\infty}
\end{equation*}
Obliczmy pochodną tej funkcji:
\begin{equation*}
    S'\pars{x}
        = \frac{1}{2}\pars{1 - \frac{1}{2\sqrt{x}}}
        = \frac{2\sqrt{x} - 1}{4\sqrt{x}} \qquad x \in \open{0}{+\infty}
\end{equation*}
Zauważmy, że skoro mianownik jest zawsze dodatni, to znak pochodnej zależy tylko od licznika, którego miejscem zerowym jest \(\sqrt{x} = \frac{1}{2}\), czyli \(x = \frac{1}{4}\). Dlatego wykres znaku pochodnej wygląda następująco:
\begin{gather*}
    \begin{tikzpicture}
        \drawvec (-2, 0) -- (2, 0) node[below]{\(\sqrt{x}\)};
        \draw[thick] (-1.5, -1.5) -- (1.5, 1.5);
        \fillpoint*{0, 0}[\(\frac{1}{2}\)][below right];
    \end{tikzpicture}\\
    \begin{tikzpicture}
        \drawvec (-2, 0) -- (2, 0) node[below]{\(x\)};
        \draw[thick] (-1.5, -1.5) -- (1.5, 1.5);
        \fillpoint*{0, 0}[\(\frac{1}{4}\)][below right];
    \end{tikzpicture}
\end{gather*}
Interesuje nas tylko przedział \(\open{0}{+\infty}\). Pochodna jest ujemna w~przedziale \(\open{0}{\frac{1}{4}}\), dla \(x = \frac{1}{4}\) przyjmuje wartość \(0\), a~w~przedziale \(\open{\frac{1}{4}}{+\infty}\) jest dodatnia. Oznacza to, że funkcja \(S\) jest malejąca w~przedziale \(\open{0}{\frac{1}{4}}\) i~rosnąca w~przedziale \(\open{\frac{1}{4}}{+\infty}\), więc dla \(x = \frac{1}{4}\) przyjmowane jest minimum lokalne, które dodatkowo okaże się globalną wartością największą, jeśli w~punkcie \(x = 0\), w~którym funkcja nie jest różniczkowalna, wartość nie okaże się mniejsza:
\begin{gather*}
    S\pars{\frac{1}{4}} = \frac{1}{2}\pars{\frac{1}{4} - \frac{1}{2} + 1} = \frac{1}{2} \cdot \frac{3}{4} = \frac{3}{8}\\
    S\pars{0} = \frac{1}{2}\pars{1 - 1 + 1} = \frac{1}{2}\\
    \frac{3}{8} < \frac{1}{2}
\end{gather*}
Zatem globalna wartość najmniejsza pola \(\triangle{ABC}\) to
\begin{equation*}
    S\pars{\frac{1}{4}} = \frac{3}{8}
\end{equation*}
Optymalny wybór punktu \(C\) to
\begin{equation*}
    C = \pars{\frac{1}{4}; \sqrt{\frac{1}{4}}} = \pars{\frac{1}{4}; \frac{1}{2}}
\end{equation*}
