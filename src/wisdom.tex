\section*{It's dangerous to go alone\ldots{} Take this!}
\subsection*{Pan Dymel rzecze}
\subsubsection*{Sprawy techniczne}
\begin{itemize}
    \item arkusz:
        \begin{itemize}
            \item naklejki --- są na wszystkie egzaminy
            \item ogarniać swój pesel
            \item przeczytać instrukcję
            \item sprawdzić dokładnie kompletność --- czy nic nie brakuje i~czy wszystko jest dobrze wydrukowane
        \end{itemize}
    \item spisywanie rozwiązań:
        \begin{itemize}
            \item do zadań zamkniętych przenosimy odpowiedzi
            \item na polskim zaznaczamy temat
            \item przy obliczeniach robimy przejścia częściowe (żeby nie było tak, że dwie linijki rachunku i~od razu wynik)
            \item brudnopis nie będzie oceniany, chyba że bardzo dokładnie napiszemy co
            \item nie pisać na marginesie
        \end{itemize}
    \item przybory:
        \begin{itemize}
            \item tylko czarny tusz / atrament, nie wolno korektora!
            \item kalkulator prosty --- będzie dostępny, ale można własny
            \item linijka \(\neq\) ekierka
            \item cyrkiel, ale tylko taki, w~który można włożyć długopis
            \item nie wolno żadnych kartek na brudno!
        \end{itemize}
    \item kartę odpowiedzi też kodujemy (jest na samym końcu), czas na to jest przed rozpoczęciem egzaminu
\end{itemize}
\subsubsection*{Strategia}
\begin{enumerate}
    \item Zadania zamknięte
        \begin{itemize}
            \item zaczynamy od nich, żeby mieć je z~głowy
            \item przenosimy rozwiązania na kartę
            \item zazwyczaj szybkie przeliczenie
            \item raczej nie rozwiązujemy jawnie, częściej szybciej będzie po prostu zweryfikować, która odpowiedź jest poprawna
        \end{itemize}
    \item Zadania z~kodowaną odpowiedzią
        \begin{itemize}
            \item miejsce pod spodem to brudnopis
            \item liczy się tylko wynik
            \item tak naprawdę można dostać tylko \(2\) punkty albo \(0\) punktów
        \end{itemize}
    \item Zadania otwarte --- większość rozwiązujemy od razu na czysto, szczególnie jeśli na pierwszy rzut oka wiemy co robić
        \begin{enumerate}[label={\Roman*)}]
            \item zaczynamy od optymalizacyjnego!
                \begin{itemize}
                    \item liczy się za \(12\%\) lub nawet \(14\%\) matury
                    \item piszemy dokładnie, tak jak ćwiczyliśmy
                \end{itemize}
            \item przeglądamy wszystkie zadania
                \begin{itemize}
                    \item grube zadania rozwiązujemy dokładnie
                \end{itemize}
            \item zadania z~dużą liczbą przypadków
                \begin{itemize}
                    \item nie dowodzimy, że to już wszystkie przypadki, po prostu MAJĄ być wszystkie
                \end{itemize}
            \item zadania z~rysunkami
                \begin{itemize}
                    \item zazwyczaj rysunek już jest
                    \item jeśli rysunek nie jest oczywisty, to najpierw w brudnopisie szkicujemy
                    \item rysujemy w~prawym górnym rogu miejsca na rozwiązanie
                \end{itemize}
            \item schemat zapisu
                \begin{itemize}
                    \item starać się jednoznacznie pokazać kolejność etapów rozwiązania, możemy podzielić stronę na dwie części --- mniejszą i~większą, żeby mieć zapas
                    \item jeśli już się zdarzy pomieszać, to dobrze jest ponumerować etapy
                \end{itemize}
        \end{enumerate}
\end{enumerate}
\subsubsection*{Czas}
\begin{itemize}
    \item nie patrzymy co chwilę na zegarek, chociaż kontrolujemy
    \item nie rzucamy się na zadanie, podchodzimy spokojnie
\end{itemize}
\subsubsection*{Sprawdzanie}
\begin{itemize}
    \item nie ma zasady, że odpowiedź ma być ładna --- brzydki wynik nie oznacza błędu, jednak warto sprawdzić
    \item nie przekreślać pochopnie fragmentu, który wydaje się zły
    \item jeśli mamy wątpliwości, a~zostało trochę czasu, możemy spróbować w~brudnopisie rozwiązać jeszcze raz bez posiłkowania się poprzednim rozwiązaniem
\end{itemize}
\subsubsection*{Punktacja}
\begin{itemize}
    \item liczba punktów nie oznacza trudności zadania, tylko liczbę ,,naturalnych kroków'', które podlegają ocenie
\end{itemize}

